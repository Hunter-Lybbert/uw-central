\documentclass[10pt]{amsart}
\include{amsmath}
\usepackage{dsfont}													% gives you \mathds{} font
\usepackage{amssymb}
\usepackage{bbm}
\usepackage{cancel}

\newcommand{\D}{\mathrm{d}}
\DeclareMathOperator{\E}{e}

\begin{document}

\noindent
\text{Hunter Lybbert} \\
\text{Student ID: 2426454} \\
\text{11-06-24} \\
\text{AMATH 561}
\title{Problem Set 5}
\maketitle

\noindent {\bf 1.} Let $X$ and $Y_0, Y_1, Y_2, . . .$ be random variables on a probability space $(\Omega, \mathcal{F},P)$ and suppose $E|X| < \infty$. Define $\mathcal{F}_n=\sigma(Y_0,Y_1,...,Y_n)$ and $X_n = E(X|\mathcal{F}_n)$. Show that the sequence $X_0, X_1, X_2,...$ is a martingale with respect to the filtration $(\mathcal{F}_n)_{n \geq 0}$. \\

\noindent
\textit{Solution:} \\
For my sake, I will review the definition of a martingale then we will show that the sequence $X_0, X_1, X_2, ...$ satisfies all of the necessary conditions and is thus itself a martingale. \\

\noindent
Let $\mathcal F_n$ be a filtration, i.e. an increasing sequence of $\sigma$-algebras.
A sequence of random variables $X_n$ is said to be adapted to $\mathcal F_n$ if $X_n \in \mathcal F_n$
(that is $X_n$ is $\mathcal F_n$-measurable or for all Borel sets $B$ we have $X_n^{-1}(B) = \{\omega \:\: | \:\: X(\omega) \in B\} \in \mathcal F_n$)
for all $n$.
If $X_n$ is a sequence with:
\begin{enumerate}
\item  $E | X_n | < \infty$
\item $X_n$ is adapted to $\mathcal F_n$
\item $E ( X_{n + 1} | \mathcal F_n ) = X_n \text{ for all } n$,
\end{enumerate}
Then $X = (X_n)_{n \in \mathbb N}$ is said to be a martingale with respect to $\mathcal F_n$. \\

\noindent
Now I will begin my actual proof.
Since $X$ and each $Y_n$ are random variables on the probability space $(\Omega, \mathcal F, P)$, then $X \in \mathcal F$ and $Y_n \in \mathcal F$ for each $n \in \mathbb N_0$.
By definition of conditional expectation, we have that $X_n \in \mathcal F_n$ for all $n$ and thus $X_n$ is adapted to $\mathcal F_n$.
Next, let's show $E (X_{n + 1} | \mathcal F_n) = X_n$ for all $n$.
Recall that if $\mathcal F_0 \subset \mathcal F_1$, then
$$
E\big(E(X|\mathcal F_1) \big|\mathcal F_0\big) = E(X|\mathcal F_0).
$$
Therefore, we have,
$$
E (X_{n + 1}|\mathcal F_n) = E \big( E(X | \mathcal F_{n + 1} ) \big| \mathcal F_n\big) = E(X | \mathcal F_n ) = X_n
$$
since
$$
\mathcal F_n = \sigma(Y_0, Y_1, ..., Y_n) \subset \sigma(Y_0, Y_1, ..., Y_n, Y_{n + 1}) = \mathcal F_{n + 1}.
$$
Finally, we want to show that $E | X_n | < \infty$ for all $n$.
Additionally, by our definition of conditional expectation in lecture slides 10 we have that for all $A \in \mathcal F_n$,
$$\int_{A} Y \D P = \int_{A} X \D P.$$
Since, $\mathcal F_n$ is a $\sigma$-algebra we can take $A = \Omega$ and then we have
\begin{align*}
E | X_n |  &= E \big| E (X | \mathcal F_n) \big| \\
	&= \int_\Omega \big|E (X | \mathcal F_n ) \big|\D P \\
	&= \int_\Omega |Y| \D P \\
	&= \int_\Omega |X| \D P \\
	&= E |X| < \infty
\end{align*}
Therefore, $E | X_n | < \infty$. Hence, the sequence of random variables $(X_n)_{n \in \mathbb N_0}$ is a martingale with respect to the filtration $(\mathcal{F}_n)_{n \geq 0}$.\\
\qed \\
\newpage

\noindent {\bf 2.} Let $X_0, X_1, . . .$  be i.i.d Bernoulli random variables with parameter $p$ (i.e., $P(X_i = 1) = p, P(X_i = 0) =1- p$). Define $S_n = \sum_{i=1}^n X_i$ where $S_0 = 0$. Define
$$Z_n = \left(\frac{1-p}{p} \right)^{2S_n-n}, \,\,\,\, n = 0, 1, 2, . . . .$$
Let $\mathcal{F}_n = \sigma(X_0, X_1, . . . , X_n )$. Show that $Z_n$ is a martingale with respect to this filtration. \\

\noindent
\textit{Solution:} \\
We need to verify that the sequence of random variables $(Z_n)_{n \in \mathbb N_0}$ satisfies the requisite criteria to be a martingale.
Let's begin by showing that $E|Z_n| < \infty$.
Let's look more closely at the expected value of the absolute value of $Z_n$.
Since $p \in [0, 1]$, we know that each $Z_n$ is just a nonnegative number raised to some integer power so it is always positive.
Therefore, in calculating the expected value we can drop the absolute value
\begin{align*}
E|Z_n| &= E\Bigg(\left(\frac{1-p}{p} \right)^{2S_n-n}\Bigg) \\
	&= E\bigg(\left(\frac{1-p}{p} \right)^{2(\sum_{i=1}^n X_i)-n}\bigg) \\
	&= \left(\frac{1-p}{p} \right)^{-n} E\Bigg(\left(\frac{1-p}{p} \right)^{2\sum_{i=1}^n X_i}\Bigg) \\
	&= \left(\frac{1-p}{p} \right)^{-n} E\Bigg( \prod_{i = 1}^n \left(\frac{1-p}{p} \right)^{2 X_i}\Bigg) \\
	&= \left(\frac{1-p}{p} \right)^{-n}\prod_{i = 1}^n E\Bigg( \left(\frac{1-p}{p} \right)^{2 X_i}\Bigg) \\
	&= \left(\frac{1-p}{p} \right)^{-n} E\Bigg( \left(\frac{1-p}{p} \right)^{2 X}\Bigg)^n.
\end{align*}
where the final few lines hold due to the fact that the $X_i$ are i.i.d.
It is important to note that we are taking the $n$th power of the expectation of our expression instead of the expectation of the expression  to the $n$th power. The difference is important.
Using the definition of expectation and the fact that the $X$'s are Bernoulli distributed we have
\begin{align*}
E|Z_n| &= \left(\frac{1-p}{p} \right)^{-n} \Bigg(\int_\Omega \left(\frac{1-p}{p} \right)^{2 X} \D P \Bigg)^n \\
	&= \left(\frac{1-p}{p} \right)^{-n} \Bigg(\sum_{x \in \{0, 1\}} \left(\frac{1-p}{p} \right)^{2 x} P(X = x) \Bigg)^n \\
	&= \left(\frac{1-p}{p} \right)^{-n} \Bigg(P(X = 0) + \left(\frac{1-p}{p} \right)^2 P(X = 1)\Bigg)^n \\
	&= \left(\frac{1-p}{p} \right)^{-n} \Bigg((1-p) + \left(\frac{1-p}{p} \right)^2 p\Bigg)^n.
\end{align*}
I suppose we may be able to say something about the finiteness of the expected value at this point but I will continue with the algebra until it is more obvious to me.
Getting common denominators inside the parenthesis on the right, we have
\begin{align*}
E|Z_n| &= \left(\frac{1-p}{p} \right)^{-n} \Bigg(\frac{p(1-p)}{p} + \frac{(1-p)^2}{p}\Bigg)^n \\
	&= \left(\frac{1-p}{p} \right)^{-n} \Bigg(\frac{p(1-p)+ (1-p)^2}{p}\Bigg)^n \\
	&= \frac{(1-p)^{-n}}{p^{-n}} \frac{\bigg((1 - p)\big(p + (1-p)\big)\bigg)^n}{p^n} \\
	&= (1-p)^{-n} \bigg( (1 - p) \big(p + 1 -p \big) \bigg)^n \\
	&= (1-p)^{-n} (1 - p)^n \\
	&= 1
\end{align*}
Thus, $E |Z_n| < \infty$.
Now we need to show that $Z_n$ is adapted to $\mathcal F$ or that $Z_n \in \mathcal F_n$ for all $n$.
Each $X_n \in \mathcal F_n$ for all $n$ and thus, $S_n \in \mathcal F_n$.
Observe that $Z_n$ is a nonnegative real number raised to the power of $S_n$ (an $\mathcal F_n$-measurable random variable).
Since $Z_n$ is of the form $Z_n = g(S_n)$ with the $g$ afore described, then $Z_n \in \mathcal F_n$.
Finally, we need to show, for all $n$, that
$$ E (Z_{n + 1} | \mathcal F_n) = Z_n. $$
Beginning on the right
\begin{align*}
E (Z_{n + 1} | \mathcal F_n) &= E \Bigg(\left(\frac{1-p}{p} \right)^{2S_{n + 1} - n - 1} \Bigg| \sigma(X_0, X_1, ...,  X_n)\Bigg) \\
	&= \left(\frac{1-p}{p} \right)^{- n - 1} E \Bigg(\left(\frac{1-p}{p} \right)^{2S_{n + 1}} \Bigg| \sigma(X_0, X_1, ...,  X_n)\Bigg) \\
	&= \left(\frac{1-p}{p} \right)^{- n - 1} E \Bigg(\left(\frac{1-p}{p} \right)^{2\big(\sum_{i=0}^{n} X_i\big) + 2X_{n + 1}} \Bigg| \sigma(X_0, X_1, ...,  X_n)\Bigg) \\
	&= \left(\frac{1-p}{p} \right)^{- n - 1} E \Bigg(\left(\frac{1-p}{p} \right)^{2\sum_{i=0}^{n} X_i} \left(\frac{1-p}{p} \right)^{2X_{n + 1}} \Bigg| \sigma(X_0, X_1, ...,  X_n)\Bigg).
\end{align*}
Since, we are conditioning on $\sigma(X_0, X_1, ...,  X_n)$ each $X_i$ up to $X_n$ is constant with respect to this given information.
Therefore it can be treated like a constant and pulled out of the expected value because of the linearity of expected value.
We now proceed with this step and can drop the conditioning since $X_{n + 1}$ is independent from the $\sigma$-algebra generated by the collection of $X_i$'s
\begin{align*}
E (Z_{n + 1} | \mathcal F_n) &= \left(\frac{1-p}{p} \right)^{- n - 1} \left(\frac{1-p}{p} \right)^{2\sum_{i=0}^{n} X_i} E \Bigg(\left(\frac{1-p}{p} \right)^{2X_{n + 1}} \Bigg| \sigma(X_0, X_1, ...,  X_n)\Bigg) \\
	&= \left(\frac{1-p}{p} \right)^{- n - 1} \left(\frac{1-p}{p} \right)^{2S_n} E \Bigg(\left(\frac{1-p}{p} \right)^{2X_{n + 1}}\Bigg) \\
	&= \left(\frac{1-p}{p} \right)^{- n - 1} \left(\frac{1-p}{p} \right)^{2S_n} \Bigg(P(X_{n + 1} = 0)  + \left(\frac{1-p}{p} \right)^2P(X_{n + 1} = 1)\Bigg) \\
	&= \left(\frac{1-p}{p} \right)^{- n - 1} \left(\frac{1-p}{p} \right)^{2S_n} \Bigg((1 - p)  + \frac{(1-p)^2}{p}\Bigg) \\
	&= \left(\frac{1-p}{p} \right)^{- n - 1} \left(\frac{1-p}{p} \right)^{2S_n} \Bigg(\frac{(1 - p)\big(p + 1 - p\big)}{p}\Bigg) \\
	&= \left(\frac{1-p}{p} \right)^{- n - 1} \left(\frac{1-p}{p} \right)^{2S_n} \Bigg(\frac{1 - p}{p} \Bigg) \\
	&= \left(\frac{1-p}{p} \right)^{- n} \left(\frac{1-p}{p} \right)^{2S_n} \\
	&=\left(\frac{1-p}{p} \right)^{2S_n - n} \\
	&= Z_n.
\end{align*}
Therefore, $E (Z_{n + 1} | \mathcal F_n)  = Z_n$ for all $n$.
And hence, $Z_n$ is a martingale with respect to $\mathcal{F}_n = \sigma(X_0, X_1, . . . , X_n )$. \\
\qed \\


\newpage

\noindent {\bf 3.} Let $\xi_i$ be a sequence of random variables such that the partial sums 
$$X_n=\xi_0+\xi_1+...+\xi_n, \,\,\,\, n\geq 1,$$
determine a martingale. Show that the summands are mutually uncorrelated, i.e. that $E(\xi_i\xi_j)=E(\xi_i)E(\xi_j)$ for $i\neq j$. \\

\noindent
\textit{Solution:} \\
This means there exists some filtration $\mathcal F_n = \sigma(X_0, X_1, X_2, ..., X_n)$ built out of all the information from previous steps in the martingale such that $X_n$ is $\mathcal F_n$ adapted and both
$$E (X_{n + 1} | \mathcal F_n) = X_n \text{ and }E |X_n| < \infty$$
hold.
Then we also have that
\begin{align*}
E (X_{n + 1} | \mathcal F_n) &= E\bigg( \sum_{i=0}^{n+1} \xi_i \bigg| \mathcal F_n \bigg) \\
	&= E\bigg( \xi_{n + 1} + \sum_{i=0}^n \xi_i \bigg| \mathcal F_n \bigg) \\
	&= E ( \xi_{n + 1} | \mathcal F_n ) + E\bigg( \sum_{i=0}^n \xi_i \bigg| \mathcal F_n \bigg) \\
	&= E ( \xi_{n + 1} | \mathcal F_n ) + \sum_{i=0}^n E( \xi_i | \mathcal F_n).
\end{align*}
Recall, $E (X_{n + 1} | \mathcal F_n) = X_n$ thus
\begin{align*}
E (X_{n + 1} | \mathcal F_n) &= X_n \\
E ( \xi_{n + 1} | \mathcal F_n ) + \sum_{i=0}^n E( \xi_i | \mathcal F_n) &= \sum_{i=0}^n \xi_i \\
E ( \xi_{n + 1} | \mathcal F_n ) + \sum_{i=0}^n \xi_i &= \sum_{i=0}^n \xi_i \\
E ( \xi_{n + 1} | \mathcal F_n ) &= 0 \\
E ( \xi_{n + 1}) &= 0.
\end{align*}
We arrive at the final equality, since $\mathcal F_n$ has no information about $X_{n + 1}$ let alone $\xi_{n + 1}$.
Therefore, without loss of generality let $ i < n + 1$, then 
\begin{align*}
E(\xi_i\xi_{n + 1})
	&= E(\xi_i | \xi_{n + 1}) E(\xi_{n + 1}) \\
	&= E(\xi_i | \xi_{n + 1}) \cdot 0 \\
	&= 0 \\
	&= E(\xi_i) \cdot 0 \\
	&= E(\xi_i)E(\xi_{n + 1}).
\end{align*}
Hence, $\xi_i$ and $\xi_j$ ($i\neq j$) are uncorrelated, since
$$E(\xi_i\xi_j) = E(\xi_i)E(\xi_j) = 0.$$
\qed \\

\newpage

\noindent {\bf 4.} Galton and Watson who invented the process that bears their names were interested in the survival of family names. Suppose each family has exactly 3 children but coin flips determine their sex. In the 1800s, only male children kept the family name so following the male offspring leads to a branching process with $p_0 = 1/8, p_1 = 3/8, p_2 = 3/8, p_3 = 1/8$. Compute the probability $\rho$ that the family name will die out when $Z_0 = 1$. What is $\rho$ if we assume that each family has exactly 2 children? \\

\noindent
\textit{Solution:} \\
Let $Z_0 = 1$.
Furthermore, define $Z_{n+1} = \xi_0^{n + 1} + \xi_1^{n + 1} + \xi_2^{n + 1} + ... + \xi_{Z_n}^{n + 1}$ where this follows the same definition in class.
$Z_{n+1}$ represents the number males in the $n+1$ generation which bear the last name. \\

\noindent
We are going to go about this thinking about the branching process with respect to the birth death process.
We are given that $Z_0 = 1$. 
At the beginning there is a $\frac 1 8$ probability the name dies out right away, a $\frac 3 8$ probability 1 male child is born and can bear the name, a $\frac 3 8$ probability 2 male children are born and can bear the name and finally, a $\frac 1 8$ probability that 3 male children are born and can bear the name.
This gives us
$$
\rho = \frac 1 8 + \frac 3 8 \rho + \frac 3 8 \rho^2 + \frac 1 8 \rho^3.
$$
Solving this for $\rho$ gives us
$$
\rho = 1, \quad \rho = -2 + \sqrt{5}, \quad \rho = -2 - \sqrt{5}.
$$
Since, $$E(\xi_i) = 0 \frac 1 8 + 1 \frac 3 8 + 2 \frac 3 8 + 3 \frac 1 8 = \frac {12} 8 > 1$$
we can say the process is superciritical.
Therefore can determine that we need $0 < \rho < 1$ and thus we have 
$$\rho = -2 + \sqrt{5}.$$
\\
\textbf{TODO: If I have time I will address the case if the family has exactly 2 children.}

\newpage

\end{document}  
