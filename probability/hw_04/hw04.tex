\documentclass[10pt]{amsart}
\include{amsmath}
\usepackage{dsfont}													% gives you \mathds{} font
\usepackage{amssymb}
\usepackage{bbm}

\newcommand{\D}{\mathrm{d}}
\DeclareMathOperator{\E}{e}

\begin{document}

\noindent
\text{Hunter Lybbert} \\
\text{Student ID: 2426454} \\
\text{10-30-24} \\
\text{AMATH 561}
\title{Problem Set 4}
\maketitle

\noindent {\bf 1.} Let $\Omega=\{a,b,c,d\}$ and let $\mathcal{F}=2^{\Omega}$. We define a probability measure $P$ as follows:

$$P({a})=1/6, \,\,\,, P({b})=1/3, \,\,\, P({c})=1/4, \,\,\, P({d})=1/4.$$
Next, define three random variables:

$$X(a)=1, \,\,\, X(b)=1, \,\,\, X(c)=-1, \,\,\,  X(d)=-1,$$
$$Y(a)=1, \,\,\, Y(b)=-1, \,\,\, Y(c)=1, \,\,\,  Y(d)=-1,$$
and $Z=X+Y$. \\

\noindent
(a) List the sets in $\sigma(X)$. \\
\textit{Solution:} \\
The pre-image of X is
$$
X^{-1}(B) =
\begin{cases}
	\{c, d\}, \text{ if } -1 \in B, 1 \not\in B \\
	\{a, b\}, \text{ if } 1 \in B, -1 \not\in B.
\end{cases}
$$
Then we have
$$\sigma(X) = \sigma(\big\{ \{a, b\}, \{c, d\} \big\}) = \bigg\{ \{a, b\}, \{c, d\}, \Omega, \emptyset \bigg\}$$
\qed
\\

\noindent
(b) Calculate $E(Y|X)$. \\
\textit{Solution:} \\
We can calculate this as follows
$$
\mathbb E[Y | X] = \mathbb E [Y | \sigma(X)]
$$
then we have
$$
\mathbb E [Y | \sigma(X)] = \frac {\mathbb E[ Y; \{ a, b \}]}{P(\{ a, b\})} = \frac {1\cdot P(a)  -1\cdot P(b)}{\frac 1 6 + \frac 1 3} = \frac {1\cdot \frac 1 6 - 1\cdot \frac 1 3}{\frac 1 2} = - \frac 1 3
$$
and
$$
\mathbb E [Y | \sigma(X)] = \frac {\mathbb E[ Y; \{ c, d \}]}{P(\{ c, d\})} = \frac {1\cdot P(c) - 1\cdot P(d)}{\frac 1 2} = \frac { 1\cdot \frac 1 4 - 1\cdot \frac 1 4}{\frac 1 2}  = 0
$$
\qed \\

\noindent
(c) Calculate $E(Z|X)$. \\
\textit{Solution:} \\
Let's first look at the values that $Z(\omega)$ takes on for each $\omega \in \{a, b, c, d\}$.
$$
Z(a) = 2,\;Z(b) = 0,\;Z(c) = 0,\;Z(d) = -2
$$
Then we have
$$
\mathbb E[Z|X] = \mathbb E[Z | \sigma(X)]
$$
giving us
$$
\mathbb E [Z | \sigma(X)] = \frac {\mathbb E[ Z; \{ a, b \}]}{P(\{ a, b\})} = \frac {2\cdot P(a) + 0\cdot P(b)}{\frac 1 6 + \frac 1 3} = \frac {2\cdot \frac 1 6 + 0\cdot \frac 1 3}{\frac 1 2} = \frac 2 3
$$
and
$$
\mathbb E [Z | \sigma(X)] = \frac {\mathbb E[ Z; \{ c, d \}]}{P(\{ c, d\})} = \frac {2\cdot P(c) + 0\cdot P(d)}{\frac 1 2} = \frac {2\cdot \frac 1 4 + 0\cdot \frac 1 4}{\frac 1 2}  = \frac 1 2
$$
\qed \\
\newpage

\noindent {\bf 2.} (a) Prove that $E(E(X|\mathcal{F}))=EX$. \\
\textit{Solution:} \\
There is an underlying probability space $(\Omega, \mathcal F_{\star}, P)$.
Let $\mathcal F = \sigma(\{\Omega_1, \Omega_2, ...\})$, then
$$
\mathbb E[X |\mathcal F]
	= \frac{\mathbb E [X; \Omega_i]}{P(\Omega_i)}
	= \frac{\int_{\Omega_i}  X \D P}{P(\Omega_i)}
$$
\begin{align*}
\mathbb E \big[ \mathbb E [X | \mathcal F]\big] = \int_\Omega \mathbb E [X | \mathcal F] \D P =
\end{align*}
\begin{align*}
\mathbb E[X] = \int_\Omega X \D P = \int_{\mathbb R} x \mu(\D x)
\end{align*}

\noindent
(b) Show that if $\mathcal{G}\subset \mathcal{F}$ and $EX^2<\infty$ then
$$E(\{X-E(X|\mathcal{F})\}^2)+ E(\{E(X|\mathcal{F})-E(X|\mathcal{G})\}^2)=E(\{X-E(X|\mathcal{G})\}^2)$$ 

\noindent {\bf 3.} An important special case of the previous result (2b) occurs when $\mathcal{G} = \{\emptyset, \Omega \}$.  Let ${\rm var}(X|\mathcal{F})=E(X^2|\mathcal{F})-E(X|\mathcal{F})^2$. Show that 
$${\rm var}(X)=E({\rm var}(X|\mathcal{F}))+{\rm var}(E(X|\mathcal{F})).$$


\noindent {\bf 4.}  Let $Y_1, Y_2, . . .$ be i.i.d. (independent and identically distributed) random variables with mean $\mu$ and variance $\sigma^2$, $N$ an independent positive integer valued random variable with $EN^2 < \infty$ and $X = Y_1 +...+Y_N$. Show that ${\rm var}(X) = \sigma^2 EN + \mu^2 {\rm var}(N)$. (To understand and help remember the formula, think about the two special cases in which $N$ or $Y$ is constant.)
\end{document}  
