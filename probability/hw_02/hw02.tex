\documentclass[10pt]{amsart}
\include{amsmath}
\usepackage{dsfont}													% gives you \mathds{} font

\begin{document}

\noindent
\text{Hunter Lybbert} \\
\text{Student ID: 2426454} \\
\text{10-07-24} \\
\text{AMATH 561}
\title{Problem Set 2}
\maketitle

\noindent {\bf 1.} Suppose $X$ and $Y$ are random variables on $(\Omega, \mathcal{F},P)$ and let $A\in \mathcal{F}$. Show that if we let $Z(\omega)=X(\omega)$ for $\omega \in A$ and $Z(\omega)=Y(\omega)$ for $\omega \in A^c$, then $Z$ is a random variable. 
\\

\noindent {\bf 2.} Suppose $X$ is a continuous random variable with distribution function $F_X$. Let $g$ be a strictly increasing continuous function. Define $Y=g(X)$. a) What is $F_Y$, the distribution function of $Y$? b) What is $f_Y$, the density function of $Y$?
\\

\noindent {\bf 3.} Suppose $X$ is a continuous random variable with distribution function $F_X$. Find $F_Y$ where $Y$ is given by a) $X^2$ b) $\sqrt{|X|}$ c) $\sin X$ d) $F_X(X)$.
\\

\noindent {\bf 4.}  Let $X: [0,1] \to \mathbf{R}$ be a function that maps every rational number in the interval $[0,1]$ to 0, and every irrational number to 1. We assume that the probability space where $X$ is defined is $([0,1],\mathcal{B}[0,1],P)$, where $\mathcal{B}[0,1]$ is the Borel $\sigma$-algebra on [0,1], and $P$ is the Lebesgue measure. 

(a) Is the set of rational numbers in [0,1] a Borel set? Show using definition of the Borel  $\sigma$-algebra on $[0,1]$. 

(b) Is $X$ a random variable (and why)? If it is, what are its distribution function and expectation? Does $X$ have a density function? Is $X$ discrete?

\end{document}  
