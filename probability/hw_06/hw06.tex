\documentclass[10pt]{amsart}
\include{amsmath}
\usepackage{dsfont}													% gives you \mathds{} font
\usepackage{amssymb}
\usepackage{bbm}
\usepackage{cancel}

\newcommand{\D}{\mathrm{d}}
\DeclareMathOperator{\E}{e}
\DeclareMathOperator{\I}{i}

\begin{document}

\noindent
\text{Hunter Lybbert} \\
\text{Student ID: 2426454} \\
\text{11-13-24} \\
\text{AMATH 561}
\title{Problem Set 6}
\maketitle

\noindent {\bf 1.} Let $X\sim Binomial(n,U)$, where $U\sim Uniform((0,1))$. What is the probability generating function $G_X(s)$ of $X$? What is $P(X=k)$ for $k\in \{0,1,2,...,n\}$? \\

\noindent
\textit{Solution:}
The probability mass function for $X\sim Binomial(n,U)$, is given by
$$ f_X(x) = {n \choose x} U^x (1 - U)^{n - x} \quad \text{ for } x = 0, 1, 2, ..., n$$
And the density for the Uniform distribution is 
$$g_X(x) = \frac 1 {b - a} = \frac 1 {1 - 0} = 1 \quad \text{ for } x \in (0, 1)$$
Then $G_X(s)$ is
\begin{align*}
G_X(s) &= E(s^X) \\
	&= E(E(s^X | U )) \\
	&= E\left(\sum_{x=0}^n {n \choose x} U^x (1 - U)^{n - x}s^x\right) \\
	&= E\left(\sum_{x=0}^n {n \choose x} (Us)^x (1 - U)^{n - x}\right) \\
	&= E\big( (Us + 1 - U)^n \big) \\
	&= \int_0^1 (us + 1 - u)^n du \\
	&= \bigg. \frac {(us + 1 - u)^{n + 1}}{(n + 1)(s - 1)} \Bigg|_0^1 \\
	&= \frac {(1s + 1 - 1)^{n + 1}}{(n + 1)(s - 1)} - \frac {(0s + 1 - 0)^{n + 1}}{(n + 1)(s - 1)} \\
	&= \frac {s^{n + 1}}{(n + 1)(s - 1)} - \frac {1^{n + 1}}{(n + 1)(s - 1)} \\
	&= \frac {s^{n + 1} - 1}{(n + 1)(s - 1)}
\end{align*}
Now to calculate $P(X = k)$, notice
\begin{align*}
G_X(s)
	= \frac {s^{n + 1} - 1}{(n + 1)(s - 1)}
	&= \frac 1 {n + 1} \frac {s^{n + 1} - 1}{s - 1} \\
	&= \frac 1 {n + 1} \frac {1 - s^{n + 1}}{1 - s} \\
	&= \sum_{k=0}^n \frac 1 {n + 1} s^k \\
	&= \sum_{k=0}^n P(X = k) s^k \\
	&= G_X(s)
\end{align*}
Therefore, $$ P(X = k) = \frac 1 {n + 1} \quad \text{ for all } k \in \{0, 1, 2, .., n\}. $$
\qed \\

\newpage

\noindent {\bf 2.} Consider a branching process with immigration

$$Z_0 = 1, \,\,\,\,Z_{n+1} = \sum_{i=1}^{Z_n}  \xi^{n+1}_i + Y_{n+1},$$
where the $(\xi^{n+1}_i)$ are iid with common distribution $\xi$, the $(Y_n)$ are iid with common distribution $Y$ and
the $(\xi^{n+1}_i)$ and $(Y_{n+1})$ are independent. What is $G_{Z_{n+1}}(s)$ in terms of $G_{Z_n}(s)$, $G_{\xi}(s)$ and $G_Y(s)$? Write $G_{Z_2}(s)$ explicitly in terms of $G_\xi(s)$ and $G_Y(s)$. \\

\noindent
\textit{Solution:} \\
We can write the generating function $G_{Z_{n + 1}}(s)$ as follows
\begin{align*}
G_{Z_{n+1}}(s) &=  G_{\sum_{i=1}^{Z_n}  \xi^{n+1}_i + Y_{n+1}}(s) \\
	&=  G_{ \xi^{n+1}_1 + \xi^{n+1}_2 + \xi^{n+1}_3 + ... + \xi^{n+1}_{Z_n} + Y_{n+1}}(s) \\
	&=  G_{ \xi^{n+1}_1 + \xi^{n+1}_2 + \xi^{n+1}_3 + ... + \xi^{n+1}_{Z_n}} G_{Y_{n+1}}(s) \\
	&=  G_{ Z_n}(G_{\xi}(s)) G_{Y_{n+1}}(s).
\end{align*}
The second to third equality comes from the fact that the $Y_{n + 1}$ and $\xi_i^{n + 1}$ are independent.
Finally, the last equality comes from an application of the Theorem 3 from Lecture 15. \\

\noindent
Next to calculate $G_{Z_2}(s)$ explicitly we get
\begin{align*}
G_{Z_2}(s)
	&= G_{Z_1}(G_{\xi}(s))G_Y(s) \\
	&= \bigg(G_{\xi}\Big(G_{\xi}(s)\Big)G_Y\Big(G_{\xi}(s)\Big)\bigg)G_Y(s).
\end{align*}
\qed

\newpage

\noindent {\bf 3.} (a) Let $X$ be exponentially distributed with parameter $\lambda$. Show by elementary integration (not complex integration) that $E(e^{itX}) = \lambda/(\lambda-i t)$.
\\

\noindent
\textit{Solution:}
\begin{align*}
E(\E^{\I tX}) = \int_\Omega \E^{\I tX} dP
	&= \int_{\mathbb R} \E^{\I tx} \lambda \E^{-\lambda x} dx \\
	&= \int_{0}^\infty \E^{\I tx} \lambda \E^{-\lambda x} dx \\
	&= \int_{0}^\infty \lambda \E^{(\I t -\lambda) x} dx \\
	&= \int_{0}^\infty \lambda \E^{-(\lambda- \I t) x} dx
\end{align*}
Notice this integral is off by a scale factor to the density of an exponentially distributed random variable with parameter $\lambda - \I t$.
Additionally, we know the integral of a probability density function is equal to 1, therefore, 
\begin{align*}
\int_{0}^\infty (\lambda - \I t) \E^{-(\lambda- \I t) x} dx = 1 \\
\frac \lambda {\lambda - \I t} \int_{0}^\infty (\lambda - \I t) \E^{-(\lambda- \I t) x} dx = \frac \lambda {\lambda - \I t} \\
\int_{0}^\infty \frac \lambda {\lambda - \I t} (\lambda - \I t) \E^{-(\lambda- \I t) x} dx = \frac \lambda {\lambda - \I t} \\
\int_{0}^\infty \lambda \E^{-(\lambda- \I t) x} dx = \frac \lambda {\lambda - \I t}.
\end{align*}
Which is indeed the integral we wanted to compute. \\
\qed \\

\noindent
(b) Find the characteristic function of the density function $f(x)=\frac{1}{2}e^{-|x|}$ for $x\in \mathds{R}$.
\\

\noindent
\textit{Solution:}
The characteristic function is (skipping directly to the change of variable form of the expectation)
\begin{align*}
\phi_X(t) = E(\E^{\I tX}) &= \int_{-\infty}^\infty \E^{\I tx} \frac 1 2 \E^{-|x|} dx \\
	&= \frac 1 2 \int_{-\infty}^\infty \E^{\I tx - |x|} dx.
\end{align*}
Let's split up the integral into cases in order to handle the absolute value.
Then we have
\begin{align*}
\phi_X(t) = \frac 1 2 \int_{-\infty}^\infty \E^{\I tx - |x|} dx
	&= \frac 1 2 \left[ \int_{-\infty}^0 \E^{\I tx - |x|} dx + \int_0^\infty \E^{\I tx - |x|} dx \right] \\
	&= \frac 1 2 \left[ \int_{-\infty}^0 \E^{\I tx + x} dx + \int_0^\infty \E^{\I tx - x} dx \right] \\
	&= \frac 1 2 \left[ \int_{-\infty}^0 \E^{(\I t + 1)x} dx + \int_0^\infty \E^{-(1- \I t ) x} dx \right] \\
	&= \frac 1 2 \left[ \left( \left. \frac 1 {\I t + 1} \E^{(\I t + 1)x} \right|_{-\infty}^0 \right)
		+ \left( \left. - \frac 1 {1- \I t} \E^{-(1- \I t) x} \right|_0^\infty \right) \right].
\end{align*}
Now, as we evaluate these expressions at the their respective bounds of integration, notice the terms evaluated at $-\infty$ and at $\infty$ in the left and right integrals both go to 0.
Then we have
\begin{align*}
\phi_X(t) &= \frac 1 2 \left[ \left( \frac 1 {\I t + 1} \E^{(\I t + 1)0} - \cancelto{0}{\frac 1 {\I t + 1} \E^{(\I t + 1)(-\infty)}} \right)
		+ \left( \cancelto{0}{- \frac 1 {1- \I t} \E^{-(1- \I t) \infty}} + \frac 1 {1- \I t} \E^{-(1- \I t) 0} \right) \right] \\
	&= \frac 1 2 \left[ \frac 1 {\I t + 1} + \frac 1 {1- \I t} \right] \\
	&= \frac 1 2 \left( \frac {1- \I t + \I t + 1} {(\I t + 1)(1- \I t)} \right) \\
	&= \frac 1 2 \left( \frac {2} {1 - \I^2 t^2} \right) \\
	&= \frac 1 {1 + t^2} .
\end{align*}
Hence, the characteristic function for a random variable with it's density given by $f(x)=\frac{1}{2}e^{-|x|}$ for $x\in \mathds{R}$ is
$$ \phi_X(t) = \frac 1 {1 + t^2}. $$

\newpage

\noindent {\bf 4.} A coin is tossed repeatedly, with heads turning up with probability $p$ on each toss. Let $N$ be the minimum number of tosses required to obtain $k$ heads. Show that, as $p \to 0$, the distribution function of $2Np$ converges to that of a gamma distribution. Note that, if $X \sim \Gamma(\lambda, r )$ then
$$f_X(x) = \frac{1}{\Gamma(r)} \lambda^r x^{r-1} e^{-\lambda x} \, 1_{x \geq 0}.$$

\noindent
\textit{Solution:} \\
Notice our scenario of N being the minimum number of tosses required to obtain $k$ heads (call a heads a ``succcess") is related to the negative binomial distribution.
This distribution describes the probability of $\ell$ failures occurring before $k$ successes, the p.m.f. is given by
$$
P(N = n) = {n - 1 \choose k - 1} p^k (1 - p)^{n - k}.
$$
To further connect this distribution to our scenario $N = \ell + k$.
I think... \\
\begin{align*}
\lim_{p \rightarrow 0} {n - 1 \choose k - 1} p^k (1 - p)^{n - k}
	&= \lim_{p \rightarrow 0} \frac {(n - 1)!}{(n - k)! (k - 1)!} p^k (1 - p)^{n - k} \\
	&= \lim_{p \rightarrow 0} \frac {(n - 1)!}{(n - k)! (k - 1)!} p^k \left[(1 - p)^{1/p}\right]^{(n - k)p} \\
	&= \frac {(n - 1)!}{(n - k)! (k - 1)!} p^k \E^{(n - k)p} \\
	&= \frac {\frac{(n - 1)!}{(k - 1)!}}{\Gamma(n - k + 1)} p^k \E^{(n - k)p} ...
\end{align*} 
\textbf{TODO: Come back to this, you're so close}
We wish to show that as $p \rightarrow 0$ the distribution of $2Np$ converges to that of a gamma distribution.
Recall that the density of the gamma distribution is \\

\end{document}  
