\documentclass[10pt]{amsart}
\usepackage[margin=1.4in]{geometry}
\usepackage[usenames,dvipsnames,cmyk]{xcolor} %load first
\usepackage{cancel}
\usepackage{graphicx,subfig}
\graphicspath{ {./images/} }

\usepackage{amssymb,amsmath,enumitem,url}

\newcommand{\D}{\mathrm{d}}
\newcommand{\I}{\mathrm{i}}
\DeclareMathOperator{\E}{e}
\DeclareMathOperator{\OO}{O}
\DeclareMathOperator{\oo}{o}
\DeclareMathOperator{\erfc}{erfc}
\DeclareMathOperator{\real}{Re}
\DeclareMathOperator{\imag}{Im}
\usepackage{tikz}
\usepackage[framemethod=tikz]{mdframed}
\theoremstyle{nonumberplain}

\mdtheorem[innertopmargin=-5pt]{sol}{Solution}
%\newmdtheoremenv[innertopmargin=-5pt]{sol}{Solution}
\definecolor{MichiganBlue}{HTML}{00274C}
\definecolor{MichiganYellow}{HTML}{FFCB05}  
\definecolor{NicePurple}{RGB}{75,56,76} %PrincePurple
\definecolor{NiceRed}{RGB}{230,37,52}
\definecolor{MidnightBlue}{rgb}{0.1, 0.1, 0.44}
\usepackage[colorlinks=true, linkcolor=MidnightBlue, citecolor=MidnightBlue, urlcolor=MidnightBlue]{hyperref}

\begin{document}
\pagestyle{empty}

\newcommand{\mline}{\vspace{.2in}\hrule\vspace{.2in}}

\noindent
\text{Hunter Lybbert} \\
\text{Student ID: 2426454} \\
\text{10-21-24} \\
\text{AMATH 567} \\

\title{\bf { Homework 4} }


\maketitle
\noindent
Collaborators*: Nate Ward \\
\\
\tiny
\text{*Listed in no particular order. And anyone I discussed at least part of one problem with is considered a collaborator.}
\normalsize
\mline
\begin{enumerate}[label={\bf {\arabic*}:}]
\item From A\&F: 2.4.2 c, e.\\
Evaluate the integral $\oint_Cf(z)\D z$, where C is the unit circle enclosing the origin, and $f(z)$ is given as follows: \\
c) $$f(z) = \frac {1} {\bar z}$$
\textit{Solution:} \\
We want to evaluate
$$
\oint_C\frac {1} {\bar z} \D z
$$
on the parameterized unit circle $z = \E^{i\theta}$ where $\theta \in [0, 2\pi)$, where $\bar z = \E^{-i\theta}$ on the unit circle.
Note, before we do the substitution we need $\D z = i\E^{i\theta} \D \theta$. Now our integral is
\begin{align*}
\oint_C\frac {1} {\bar z} \D z &= \oint_0^{2\pi}\frac {1} {\E^{-i\theta}} i\E^{i \theta}\D \theta \\
					  &= \oint_0^{2\pi} i \E^{2i \theta} \D \theta \\
					  &= \left( \left. \frac 1 2 \E^{2i\theta} \right|_0^{2\pi}\right) \\
					  &= \frac 1 2 \E^{4\pi i} - \frac 1 2 \E^{0} \\
					  &= \frac 1 2 - \frac 1 2 \\
					  &= 0
\end{align*}
\qed
\\
e) $$f(z) = \E^{\bar z}$$
\textit{Solution:} \\
We will use the same substitutions from the previous part \textbf{TODO:} maybe change this back to regular integrals instead of contour integrals after you have parameterized.
\begin{align*}
\oint_C\E^{\bar z} \D z &= \oint_0^{2\pi}\E^{\E^{-i\theta}} i\E^{i \theta}\D \theta \\
				  &= \oint_0^{2\pi} \sum_{j=1}^{\infty} \frac{\left(\E^{-i\theta}\right)^j}{j!} i\E^{i\theta} \D \theta \\
				  &= \sum_{j=1}^{\infty} \oint_0^{2\pi} i \frac{\left(\E^{-i\theta}\right)^j}{j!} {\E^{i\theta}} \D \theta.
\end{align*}
We are justified in reordering the integral of the infinite sum to be the infinite sum of the integrals since the original series converges absolutely.
Notice we can pull out the first term where $j=1$ to get,
$$
\sum_{j=1}^{\infty} \oint_0^{2\pi} i \frac{\left(\E^{-i\theta}\right)^j}{j!} {\E^{i\theta}} \D \theta
= \oint_0^{2\pi} i \frac{\left(\E^{-i\theta}\right)^1}{1!} {\E^{i\theta}} \D \theta
+ \sum_{j=2}^{\infty} \oint_0^{2\pi} i \frac{\left(\E^{-i\theta}\right)^j}{j!} {\E^{i\theta}} \D \theta.
$$
We need to look more closely at this first term, observe
\begin{align*}
\oint_0^{2\pi} i \frac{\left(\E^{-i\theta}\right)^1}{1!} \E^{i\theta} \D \theta = \oint_0^{2\pi} i \frac{\E^{-i\theta}\E^{i\theta}}{1!} \D \theta = \oint_0^{2\pi} i \frac{1}{1} \D \theta = i\oint_0^{2\pi}\D \theta = 2\pi i.
\end{align*}
Now, I will focus on the integral inside the sum where $j = 2, 3, 4, ...$
\begin{align*}
\oint_0^{2\pi} i \frac{\left(\E^{-i\theta}\right)^j}{j!} {\E^{i\theta}} \D \theta &= \oint_0^{2\pi} i \frac{\E^{-i\theta j}\E^{i\theta}}{j!} \D \theta \\
	&= \oint_0^{2\pi} i \frac{\E^{-i\theta j + i\theta}}{j!} \D \theta \\
	&=\oint_0^{2\pi} i \frac{\E^{i\theta\left( - j + 1\right)}}{j!} \D \theta \\
	&=\oint_0^{2\pi} \frac{i\E^{i\theta\left(1 - j\right)}}{j!} \D \theta \\
	&= \left. \frac{1}{1 - j} \frac{i\E^{i\theta\left(1 - j\right)}}{j!} \right|_0^{2 \pi} \\
	&= \frac{1}{1 - j} \frac{i\E^{i2\pi\left(1 - j\right)}}{j!} - \frac{1}{1 - j} \frac{i\E^0}{j!} \\
	&= \frac{i}{\left(1 - j\right)j!}\left(\E^{i2\pi\left(1 - j\right)} - 1 \right) \\
	&= \frac{i}{\left(1 - j\right)j!}\left(1 - 1 \right) \\
	&= 0. \\
\end{align*}
I want to clarify why $\E^{i2\pi\left(1 - j\right)} = 1$.
Since $j \in \{2, 3, ...\}$, then $1 - j$ is an integer and we have $\E^{i2\pi \ell}$ where $\ell \in \mathbb Z$, thus $\E^{i2\pi \ell} = 1$.
Now we return to the original problem
$$
\sum_{j=1}^{\infty} \oint_0^{2\pi} i \frac{\left(\E^{-i\theta}\right)^j}{j!} {\E^{i\theta}} \D \theta =  2\pi i + \sum_{j=2}^{\infty} 0 = 2\pi i.
$$
Now we have completed the requisite task.
\qed
\\

\item From A\&F: 2.4.4 a, b.
Use the principal branch where the argument is in $[-\pi,\pi)$.
Discuss any ambiguities. 
Use the principal branch of $\log(z)$ and $z^{\frac{1}{2}}$ where the argument is in $[-\pi,\pi)$ to evaluate the following: \\
a) $$\int_{-1}^{1}\log z \D z$$
\textit{Solution:} \\
We want to parameterize this once again using $z = r\E^{i\theta}$ where $\theta \in [-\pi,\pi)$. Now our integral is
\begin{align*}
\int_{-1}^{1}\log z \D z &= \int_{-\pi}^{0}\log \left(\E^{i\theta}\right) i \E^{i\theta} \D \theta \\
	&= \int_{-\pi}^{0} i\theta i \E^{i\theta} \D \theta.
\end{align*}
Let's use integration by parts, woohoo! We will assign the substitutions as follows:
\begin{align*}
u &= i\theta \\
\D u &= i \D \theta\\
\\
\D v &= i\E^{i \theta} \D \theta \\
v &= \E^{i \theta}.
\end{align*}
Plugging this in we have
\begin{align*}
\int_{-\pi}^{0} i\theta i \E^{i\theta} \D \theta &= \left. i\theta \E^{i\theta}\right|_{-\pi}^0 - \int_{-\pi}^0 i \E^{i \theta} \D \theta \\
	&= \left(0 - \left(-i\pi \E^{-i\pi}\right)\right) - \left. \E^{i\theta}\right|_{-\pi}^0 \\
	&= 0 + i\pi \E^{-i\pi} - \left. \E^{i\theta}\right|_{-\pi}^0 \\
	&= i\pi \E^{-i\pi} - \left(\E^{0} - \E^{-i\pi} \right) \\
	&= - i\pi - \left(1 - \left( - 1 \right) \right) \\
	&= - i\pi - \left(2\right) \\
	&= - i\pi - 2. \\
\end{align*}
\qed
\\
b) $$\int_{-1}^{1}z^{\frac{1}{2}} \D z$$
\textit{Solution:} \\
We want to parameterize this once again using $z = r\E^{i\theta}$ where $\theta \in [-\pi,\pi)$. Now our integral is
\begin{align*}
\int_{-1}^{1} z^{\frac{1}{2}} \D z &= \int_{-\pi}^{0} \left(\E^{i\theta}\right)^{\frac{1}{2}} i \E^{i\theta} \D \theta \\
	&= \int_{-\pi}^{0} i\E^{\frac{i\theta}{2}} \E^{i\theta} \D \theta \\
	&= \int_{-\pi}^{0} i\E^{\frac{i3}{2}\theta} \D \theta \\
	&= \left. \frac 2 3 \E^{\frac{i3}{2}\theta} \right|_{-\pi}^{0} \\
	&= \frac 2 3 \E^{ - \frac{i3}{2} \pi} - \frac 2 3. \\
\end{align*}
Now remembering our branch cut limits $\theta$ to be within $[-\pi, \pi)$ we change the angle $-\frac 3 2 \pi$ to be $\frac 1 2 \pi$.
Hence,
\begin{align*}
\frac 2 3 \E^{ - \frac{i3}{2} \pi} - \frac 2 3 &= \frac 2 3 \E^{ - \frac{i2}{2} \pi} \E^{ - \frac{i\pi}{2}} - \frac 2 3 \\
	&= \frac 2 3 \E^{ \frac{i\pi}{2}} - \frac 2 3 \\
	&= \frac 2 3 \left( i - 1\right).
\end{align*}
\qed
\\
\item From A\&F: 2.4.7 \\
Let C be an open (upper) semicircle of radius $R$ with its center at the origin, and consider $\int_C f(z) \D z$.
 Let $f(z) = \frac{1}{z^2 + a^2}$ for a real $a > 0$.
Show that $\left| f(z) \right| \leq \frac{1}{R^2 - a^2}$, $R > a$, and
$$
\left| \int_C f(z) dz \right| \leq \frac{\pi R}{R^2 - a^2}, \quad R > a.
$$
\textit{Solution:} \\
First, we want to show
$$
\left| f(z) \right| \leq \frac{1}{R^2 - a^2}
$$
where $R > a > 0$ and a $\in \mathbb R$.
Let's consider the function more closely
\begin{align*}
f(z) =  \frac{1}{z^2 + a^2} &= \frac{1}{(x + \I y)^2 + a^2} \\
	&= \frac{1}{x^2 + 2ixy - y^2 + a^2} \\
	&= \frac{1}{x^2 - y^2 + a^2 + i2xy }. \\
\end{align*}
Notice, we can write the real and imaginary parts of the complex number in the denominator as functions $u(x, y)$ and $v(x, y)$.
Where $u(x, y) = x^2 - y^2 + a^2$ and $v(x, y) = 2xy$.
Now we get 
$$
f(z) = \frac{1}{x^2 - y^2 + a^2 + i2xy }	 = \frac{u - iv }{u - iv }\frac{1}{u + iv } = \frac{u - iv }{u^2 + v^2 }.
$$
Then we calculate
\begin{align*}
\left| f(z) \right| &= \left| \frac{u - iv }{u^2 + v^2 } \right| \\
	&= \left| \frac{u }{u^2 + v^2 } - i\frac{v}{u^2 + v^2 } \right| \\
	&= \sqrt{
		\left(\frac{u }{u^2 + v^2 }\right)^2 + \left(\frac{v }{u^2 + v^2 }\right)^2
	} \\
	&= \sqrt{
		\frac{u^2 }{\left( u^2 + v^2 \right)^2} + \frac{v^2 }{\left( u^2 + v^2 \right)^2}
	} \\
	&= \sqrt{
		\frac{u^2 + v^2 }{\left( u^2 + v^2 \right)^2}
	} \\
	&= \sqrt{
		\frac{1}{u^2 + v^2}
	} \\
	&= \frac{1}{\sqrt{u^2 + v^2}}.
\end{align*}
If we plug our substitution back in we see
\begin{align*}
\frac{1}{\sqrt{u^2 + v^2}} &= \frac{1}{\sqrt{\left(x^2 - y^2 + a^2 \right)^2 + \left(2xy \right)^2}} \\
	&= \frac{1}{\sqrt{\left(x^2 - y^2 + a^2 \right)\left(x^2 - y^2 + a^2 \right) + 4x^2y^2}} \\
	&= \frac{1}{
		\sqrt{
			x^4 - x^2y^2 + x^2a^2 -x^2y^2 + y^4 - y^2a^2 + x^2a^2 - y^2a^2 + a^4 + 4x^2y^2
		}
	} \\
	&= \frac{1}{
		\sqrt{
			x^4 + x^2a^2 + y^4 - y^2a^2 + x^2a^2 - y^2a^2 + a^4 + 2x^2y^2
		}
	}.
\end{align*}
Now we add zero in a particular fashion, namely $- 4x^2a^2 + 4x^2a^2 $, so we can regroup the terms and refactor to get closer to what we desire
\begin{align*}
	&= \frac{1}{
		\sqrt{
			x^4 + y^4 - y^2a^2 + 2x^2a^2 - y^2a^2 + a^4 + 2x^2y^2 +\left( - 4x^2a^2 + 4x^2a^2 \right)
		}
	} \\
	&= \frac{1}{
		\sqrt{
			x^4 + y^4 - 2y^2a^2 + a^4 + 2x^2y^2 - 2x^2a^2 + 4x^2a^2
		}
	} \\
	&= \frac{1}{\sqrt{\left(x^2 + y^2 - a^2\right)^2 + \left(2xa\right)^2}}.
\end{align*}
Using the fact that $\sqrt{a + b} \geq \sqrt{a}$ for $a, b > 0$, in our next step we get a smaller denominator which makes the overall expression greater or equal to the previous step. Note, equality only holds when $x=0$.
\begin{align*}
 \frac{1}{\sqrt{\left(x^2 + y^2 - a^2\right)^2 + \left(2xa\right)^2}}
 	&\leq \frac{1}{\sqrt{\left(x^2 + y^2 - a^2\right)^2}} \\
	&= \frac{1}{x^2 + y^2 - a^2} \\
	&= \frac{1}{|z|^2 - a^2} \\
	&= \frac{1}{R^2 - a^2}
\end{align*}
Therefore $\left| f(z) \right| \leq \frac{1}{R^2 - a^2}$. \\
\qed

\noindent
Next we wish to show that 
$$
\left| \int_C f(z) dz \right| \leq \frac{\pi R}{R^2 - a^2}, \quad R > a.
$$
By Theorem 2.4.2 from A\&F, if $f(z)$ is continuous on contour C, then
$$
\left| \int_C f(z) dz \right| \leq ML
$$
where $L$ is the length of $C$ and $M$ is an upper bound for $\left| f(z) \right|$.
We have that $C$ is continuous, since $a >0$ and $a < R$ and there are no singularities or weirdness with $f(z)$ on the specified contour.
So we have 
$$
M = \frac{1}{R^2 - a^2}
$$
as we calculated in the first part of this problem.
Additionally, we know the arc length of $C$ is easy to calculate because it is half the circumference of the circle with radius $R$.
To convince myself of this I will show the general arc length formula also provides this quick calculation.
Let our parameterization of this contour be $$z(\theta) = R\E^{i\theta}$$ where $\theta \in [0, \pi)$. Then $$z^\prime(\theta) = Ri\E^{i\theta} = -R\sin\theta + iR\cos\theta.$$
Therefore calculating arc length is as follows,
\begin{align*}
L &= \int_a^b |z^\prime(t)| \D t \\
	&= \int_{0}^{\pi} \left| -R\sin\theta + iR\cos\theta \right| \D \theta \\
	&= \int_{0}^{\pi} \sqrt{ R^2\sin^2\theta + R^2\cos^2\theta } \D \theta \\
	&= \int_{0}^{\pi} \sqrt{ R^2\left(\sin^2\theta + \cos^2\theta\right) } \D \theta \\
	&= \int_{0}^{\pi} R \D \theta \\
	&= \pi R.
\end{align*}
Which is the same as half the circumference ($\frac 1 2 2\pi R = \pi R$).
And thus
$$
\left| \int_C f(z) dz \right| \leq ML \leq \frac{1}{R^2 - a^2} \pi R = \frac{\pi R}{R^2 - a^2}.
$$
Hence, 
$$
\left| \int_C f(z) dz \right| \leq \frac{\pi R}{R^2 - a^2}
$$
as desired.
\qed
\\
\item From A\&F: 2.4.8 \\
Let $C$ be an arc of the circle $\left|z\right| = R$ with $(R > 1)$ of angle $\frac{\pi}{3}$.
Show that 
$$
\left| \int_C \frac{\D z}{z^3 + 1} \right| \leq \frac \pi 3 \left( \frac{R}{R^3 - 1} \right)
$$
and deduce
$$
\lim_{R \rightarrow \infty} \int_C \frac{\D z}{z^3 + 1} = 0
$$
\textit{Solution:} \\
Similar to the previous problem, we will utilize Theorem 2.4.2 from A\&F.
We are justified in this, since the contour $C$ is continuous on the arc of the circle $|z| = R$.
This time, our arc length of the contour $C$ is
$$
L = \frac 1 6 2\pi R = \frac{\pi}{3} R.
$$
Next, we need to calculate $M$ as the upper bound for $\left| \frac{1}{z^3 + 1} \right|$.
Let's use a simpler method than in problem 3. Let's get on with it
$$
\left| f(z) \right| = \left| \frac 1 {z^3 + 1} \right| = \frac 1 {\left|z^3 + 1\right|}.
$$
Notice, if we can get a lower bound for the denominator then we will have an upper bound for the whole expression. It falls out quickly using the parameterization $z = R\E^{i\theta}$ and applying the inverse triangle inequality.
\begin{align*}
\left| z^3 + 1\right| &= \left| R^3\E^{i3\theta} + 1\right| \\
	&= \left| R^3\E^{i3\theta} - \left(-1\right)\right| \\
	&\geq \left| \left|R^3\E^{i3\theta}\right| - \left|\left(-1\right)\right|\right| \\
	&= \left| R^3 - 1\right| \\
	&= R^3 - 1
\end{align*}
where the last equality holds because $R > 1$.
Therefore, we have our lower bound for the denominator
$$
\left| z^3 + 1\right| \geq R^3 - 1,
$$
and thus an upper bound for the expression
$$
\left|f(z)\right| = \frac 1 {\left| z^3 + 1\right|} \leq \frac 1 {R^3 - 1} = M.
$$
Finally, applying Theorem 2.4.2 from A\&F we have
$$
\left| \int_C f(z) dz \right| \leq ML \leq \frac {\pi}{3} R\frac{1}{R^3 - 1} = \frac {\pi}{3} \left(\frac{R}{R^3 - 1}\right).
$$
Hence,
$$
\left| \int_C f(z) dz \right| \leq \frac {\pi}{3} \left(\frac{R}{R^3 - 1}\right).
$$
\qed \\
We will now take the limit of both sides of this inequality as $R$ goes to $\infty$
$$
\lim_{R \rightarrow \infty} \left| \int_C f(z) dz \right|
\leq \lim_{R \rightarrow \infty} \frac {\pi}{3} \left(\frac{R}{R^3 - 1}\right) = \frac \infty \infty.
$$
Applying L'Hôpital's rule once, we have
$$
\lim_{R \rightarrow \infty} \frac {\pi}{3} \left(\frac{1}{3R^2 - 1}\right) = 0.
$$
Therefore,
$$
\lim_{R \rightarrow \infty} \left| \int_C f(z) dz \right| \leq 0.
$$
If the limit of the absolute value of something is less than or equal to 0, then the limit of that thing must be zero.
This is the case because the absolute value is a non-negative function, so the ``$\leq 0 $" must be just an equality. Therefore,
$$
\lim_{R \rightarrow \infty} \left| \int_C f(z) dz \right| = 0.
$$
Note this also implies $\lim_{R \rightarrow \infty} -\left| \int_C f(z) dz \right| = 0$.
Now we have that
$$-\left| \int_C f(z) dz \right| \leq \int_C f(z) dz \leq \left| \int_C f(z) dz \right| $$
and thus by the squeeze theorem,
$$\lim_{R \rightarrow \infty} \int_C f(z) dz = 0.$$
\qed
\\

\item From A\&F: 2.5.1 b, e \\
Evaluate $\oint_C f(z)\D z$, where $C$ is the unit circle centered at the origin, and $f(z)$ is given by the following: \\
b) $$f(z) = \E^{z^2}$$ \\
\textit{Solution:} \\
Let's first break $f(z)$ up into real and imaginary parts so we can define $u(x, y)$ and $v(x, y)$ and check if the Cauchy-Riemann (C-R) equations hold.
\begin{align*}
f(z) &= \E^{z^2} \\
	&= \E^{ x^2 + 2ixy - y^2} \\
	&= \E^{x^2}\E^{2ixy}\E^{- y^2} \\
	&= \E^{x^2}\E^{- y^2}\E^{2ixy} \\
	&= \E^{x^2}\E^{- y^2}\left( \cos (2xy) + i \sin (2xy)\right) \\
	&= \E^{x^2}\E^{- y^2}\cos (2xy) + i \E^{x^2}\E^{- y^2}\sin (2xy).
\end{align*}
Then we can assign $u(x, y) = \E^{x^2}\E^{- y^2}\cos (2xy)$ and $v(x, y) = \E^{x^2}\E^{- y^2}\sin (2xy)$.
Now let's calculate the necessary derivatives to verify if $f(z)$ is analytic.
We have
\begin{align*}
\frac {\partial u}{\partial x} &= \E^{-y^2}\left( 2x\E^{x^2}\cos (2xy) - \E^{x^2} \sin (2xy)2y\right) \\
	&= \E^{x^2}\E^{-y^2}\left( 2x\cos (2xy) - \sin (2xy)2y\right) \\
\frac {\partial v}{\partial y} &= \E^{x^2}\left( (-2y)\E^{-y^2}\sin(2xy) + \E^{-y^2}\cos (2xy)2x\right) \\
	&= \E^{x^2}\E^{-y^2}\left( -2y\sin(2xy) + \cos (2xy)2x\right)
\end{align*}
which are equivalent.
Additionally, we get
\begin{align*}
\frac {\partial v}{\partial x} &= \E^{- y^2} \left( 2x\E^{x^2}\sin (2xy) + \E^{x^2}\cos (2xy)2y \right) \\
	&= \E^{x^2}\E^{- y^2} \left( 2x\sin (2xy) + \cos (2xy)2y \right) \\
- \frac {\partial u}{\partial y} &= -\left( \E^{x^2} \left( -2y\E^{- y^2}\cos (2xy) - \E^{- y^2}\sin (2xy)2x \right) \right) \\
	&= - \E^{x^2} \E^{- y^2} \left( -2y\cos (2xy) - \sin (2xy)2x \right) \\
	&= \E^{x^2} \E^{-y^2} \left(2y\cos (2xy) + \sin (2xy)2x \right)
\end{align*}
which are also equal as desired.
Therefore the C-R equations hold and f(z) is analytic everywhere.
Now by the Theorem 2.5.2 from A\&F or Cauchy's Integral formula, we can conclude
$$
\oint_C f(z)\D z = \oint_C \E^{z^2} \D z = 0.
$$
\qed 
\\
Another method of direct verification.
\begin{align*}
\oint_C \E^{z^2} \D z  &= \int_{0}^{2\pi} \E^{ \left( \E^{i\theta}\right)^2} i \E^{i \theta}\D \theta \\
	&= \int_{0}^{2\pi}i \left( \sum_{j=1}^{\infty} \frac{\left(\E^{2i\theta}\right)^j} {j!} \right) \E^{i \theta}\D \theta \\
	&= \int_{0}^{2\pi}i \sum_{j=1}^{\infty} \frac{\E^{i\theta2j + i \theta}} {j!} \D \theta \\
	&= \int_{0}^{2\pi}i \sum_{j=1}^{\infty} \frac{\E^{i\theta\left(2j + 1 \right)}} {j!} \D \theta \\
	&= \sum_{j=1}^{\infty}  \int_{0}^{2\pi}\frac{i\E^{i\theta\left(2j + 1 \right)}} {j!} \D \theta \\
\end{align*}
Again this reordering of the integral and sum is justified by the absolute convergence of the infinite series. We don't need to worry about any terms being undefined by a divide by zero issue. So let's look at the integral inside the sum
\begin{align*}
\int_{0}^{2\pi}\frac{i\E^{i\theta\left(2j + 1 \right)}} {j!} \D \theta
	&= \left. \frac{1}{2j + 1} \frac{\E^{i\theta\left(2j + 1 \right)}} {j!} \right|_0^{2\pi} \\
	&= \frac{1}{2j + 1} \frac{\E^{i2\pi\left(2j + 1 \right)}} {j!}  - \frac{1}{2j + 1} \frac{\E^{i0\left(2j + 1 \right)}} {j!} \\
	&= \frac{1}{2j + 1} \frac{\E^{i2\pi\left(2j + 1 \right)}} {j!}  - \frac{1}{\left(2j + 1\right)j!} \\
	&= \frac{1}{\left(2j + 1\right)j!}  - \frac{1}{\left(2j + 1\right)j!} \\
	&= 0 \\
\end{align*}
\qed
\\
e) $$f(z) = \frac{1}{2z^2 + 1} $$ \\
\textit{Solution:} \\
\textbf{TODO:} Definitely go the route of checking the analyticity of $f(z)$ with the C-R equations.
\\

\item Use the ideas from A\&F: 2.5.5 to evaluate $\int_0^\infty \E^{\I
    z^3 t} \D z$, $t > 0$.  Express the result in terms of $\int_0^\infty \E^{-
    r^3} \D r$. \\
The ideas we might need to use are ... it's actually really long! \\
\textit{Solution:}\\
\item From A\&F: 2.5.6. \\
Consider the integral $$I = \int_{-\infty}^{\infty} \frac{\D x}{x^2 + 1}.$$
Show how to evaluate this integral by considering
$$\oint_{C_{(\mathbb R)}} \frac{\D z}{z^2 + 1},$$
where $C_{(\mathbb R)}$ is closed semicircle in the upper half plane with endpoints at $(-R, 0)$ and $(R, 0)$ plus the $x$-axis.
\textit{Hint:} use
$$\frac{1}{z^2 + 1} = -\frac{1}{2i}\left(\frac{1}{z + i} - \frac{1}{z - i}\right),$$
and show that the integral along the open semicircle in the upper half plane vanishes as $R \rightarrow \infty$.
Verify your answer by usual integration in real variables.
\textit{Solution:}\\
\\

\noindent
Repeat this exercise for
  \begin{align*}
    I_\epsilon = \int_{-\infty}^\infty \frac{\epsilon \D x}{x^2 +
    \epsilon^2}, \quad \epsilon > 0.
  \end{align*}\\
Seems like I am supposed to do 2.5.6 and then for the given integral as well. \\
\textit{Solution:}\\
\item Use a similar method to calculate
  $\int_{-\infty}^{\infty} \frac{d x}{1+x^4}$. \\
\textit{Solution:}\\

\item From A\&F: 2.6.1 a, e.\\
Evaluate the integrals $\oint_C f(z) \D z$, where $C$ is the unit circle centered at the origin and $f(z)$ is given by the following (use Eq. (1.2.19) as necessary): \\
a)
$$
\frac{\sin z}{z}
$$
\\
\textit{Solution:}\\
e)
$$
\E^{z^2}\left(\frac{1}{z^2} - \frac{1}{z^3}\right)
$$
\\
\textit{Solution:}\\
\end{enumerate}

\end{document}

%%% Local Variables:
%%% mode: latex
%%% TeX-master: t
%%% End:
