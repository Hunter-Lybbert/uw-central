\documentclass[10pt]{amsart}
\usepackage[margin=1.4in]{geometry}
\usepackage[usenames,dvipsnames,cmyk]{xcolor} %load first
\usepackage{cancel}
\usepackage{graphicx,subfig}
\graphicspath{ {./images/} }

\usepackage{amssymb,amsmath,enumitem,url}

\newcommand{\D}{\mathrm{d}}
\newcommand{\I}{\mathrm{i}}
\DeclareMathOperator{\E}{e}
\DeclareMathOperator{\OO}{O}
\DeclareMathOperator{\oo}{o}
\DeclareMathOperator{\erfc}{erfc}
\DeclareMathOperator{\real}{Re}
\DeclareMathOperator{\imag}{Im}
\usepackage{tikz}
\usepackage[framemethod=tikz]{mdframed}
\theoremstyle{nonumberplain}

\mdtheorem[innertopmargin=-5pt]{sol}{Solution}
%\newmdtheoremenv[innertopmargin=-5pt]{sol}{Solution}
\definecolor{MichiganBlue}{HTML}{00274C}
\definecolor{MichiganYellow}{HTML}{FFCB05}  
\definecolor{NicePurple}{RGB}{75,56,76} %PrincePurple
\definecolor{NiceRed}{RGB}{230,37,52}
\definecolor{MidnightBlue}{rgb}{0.1, 0.1, 0.44}
\usepackage[colorlinks=true, linkcolor=MidnightBlue, citecolor=MidnightBlue, urlcolor=MidnightBlue]{hyperref}

\begin{document}
\pagestyle{empty}

\newcommand{\mline}{\vspace{.2in}\hrule\vspace{.2in}}

\noindent
\text{Hunter Lybbert} \\
\text{Student ID: 2426454} \\
\text{10-21-24} \\
\text{AMATH 567} \\

\title{\bf { Homework 4} }


\maketitle
\noindent
Collaborators*: Nate Ward \\
\\
\tiny
\text{*Listed in no particular order. And anyone I discussed at least part of one problem with is considered a collaborator.}
\normalsize
\mline
\begin{enumerate}[label={\bf {\arabic*}:}]
\item From A\&F: 2.4.2 c, e.\\
Evaluate the integral $\oint_Cf(z)\D z$, where C is the unit circle enclosing the origin, and $f(z)$ is given as follows: \\
c) $$f(z) = \frac {1} {\bar z}$$
\textit{Solution:} \\
We want to evaluate
$$
\oint_C\frac {1} {\bar z} \D z
$$
on the parameterized unit circle $z = \E^{i\theta}$ where $\theta \in [0, 2\pi)$, where $\bar z = \E^{-i\theta}$ on the unit circle.
Note, before we do the substitution we need $\D z = i\E^{i\theta} \D \theta$. Now our integral is
\begin{align*}
\oint_C\frac {1} {\bar z} \D z &= \oint_0^{2\pi}\frac {1} {\E^{-i\theta}} i\E^{i \theta}\D \theta \\
					  &= \oint_0^{2\pi} i \E^{2i \theta} \D \theta \\
					  &= \left( \left. \frac 1 2 \E^{2i\theta} \right|_0^{2\pi}\right) \\
					  &= \frac 1 2 \E^{4\pi i} - \frac 1 2 \E^{0} \\
					  &= \frac 1 2 - \frac 1 2 \\
					  &= 0
\end{align*}
\qed
\\
e) $$f(z) = \E^{\bar z}$$
\textit{Solution:} \\
We will use the same substitutions from the previous part
\begin{align*}
\oint_C\E^{\bar z} \D z &= \oint_0^{2\pi}\E^{\E^{-i\theta}} i\E^{i \theta}\D \theta \\
				  &= \oint_0^{2\pi} \sum_{j=1}^{\infty} \frac{\left(\E^{-i\theta}\right)^j}{j!} i\E^{i\theta} \D \theta \\
				  &= \sum_{j=1}^{\infty} \oint_0^{2\pi} i \frac{\left(\E^{-i\theta}\right)^j}{j!} {\E^{i\theta}} \D \theta.
\end{align*}
We are justified in reordering the integral of the infinite sum to be the infinite sum of the integrals since the original series converges absolutely.
I will now just look at the integral inside the sum
\begin{align*}
\oint_0^{2\pi} i \frac{\left(\E^{-i\theta}\right)^j}{j!} {\E^{i\theta}} \D \theta &= \oint_0^{2\pi} i \frac{\E^{-i\theta j}\E^{i\theta}}{j!} \D \theta \\
	&= \oint_0^{2\pi} i \frac{\E^{-i\theta j + i\theta}}{j!} \D \theta \\
	&=\oint_0^{2\pi} i \frac{\E^{i\theta\left( - j + 1\right)}}{j!} \D \theta \\
	&=\oint_0^{2\pi} \frac{i\E^{i\theta\left(1 - j\right)}}{j!} \D \theta \\
	&= \left. \frac{1}{1 - j} \frac{i\E^{i\theta\left(1 - j\right)}}{j!} \right|_0^{2 \pi} \\
	&= \frac{1}{1 - j} \frac{i\E^{i2\pi\left(1 - j\right)}}{j!} - \frac{1}{1 - j} \frac{i\E^0}{j!} \\
	&= \frac{i}{\left(1 - j\right)j!}\left(\E^{i2\pi\left(1 - j\right)} - 1 \right) \\
	&= \frac{i}{\left(1 - j\right)j!}\left(1 - 1 \right) \\
	&= 0. \\
\end{align*}
I want to clarify why $\E^{i2\pi\left(1 - j\right)} = 1$.
Since $j \in \{1, 2, 3, ...\}$, then $1 - j$ is an integer and we have $\E^{i2\pi \ell}$ where $\ell \in \mathbb Z$ which is always $1$.
Now we return to the original problem
$$
\sum_{j=1}^{\infty} \oint_0^{2\pi} i \frac{\left(\E^{-i\theta}\right)^j}{j!} {\E^{i\theta}} \D \theta = \sum_{j=1}^{\infty} 0 = 0.
$$
Now we have completed the requisite task.
\qed
\\

\item From A\&F: 2.4.4 a, b.
Use the principal branch where the argument is in $[-\pi,\pi)$.
Discuss any ambiguities. 
Use the principal branch of $\log(z)$ and $z^{\frac{1}{2}}$ where the argument is in $[-\pi,\pi)$ to evaluate the following: \\
a) $$\int_{-1}^{1}\log z \D z$$
\textit{Solution:} \\
We want to parameterize this once again using $z = r\E^{i\theta}$ where $\theta \in [-\pi,\pi)$. Now our integral is
\begin{align*}
\int_{-1}^{1}\log z \D z &= \int_{-\pi}^{0}\log \left(\E^{i\theta}\right) i \E^{i\theta} \D \theta \\
	&= \int_{-\pi}^{0} i\theta i \E^{i\theta} \D \theta.
\end{align*}
Let's use integration by parts, woohoo! We will assign the substitutions as follows:
\begin{align*}
u &= i\theta \\
\D u &= i \D \theta\\
\\
\D v &= i\E^{i \theta} \D \theta \\
v &= \E^{i \theta}.
\end{align*}
Plugging this in we have
\begin{align*}
\int_{-\pi}^{0} i\theta i \E^{i\theta} \D \theta &= \left. i\theta \E^{i\theta}\right|_{-\pi}^0 - \int_{-\pi}^0 i \E^{i \theta} \D \theta \\
	&= \left(0 - \left(-i\pi \E^{-i\pi}\right)\right) - \left. \E^{i\theta}\right|_{-\pi}^0 \\
	&= 0 + i\pi \E^{-i\pi} - \left. \E^{i\theta}\right|_{-\pi}^0 \\
	&= i\pi \E^{-i\pi} - \left(\E^{0} - \E^{-i\pi} \right) \\
	&= - i\pi - \left(1 - \left( - 1 \right) \right) \\
	&= - i\pi - \left(2\right) \\
	&= - i\pi - 2. \\
\end{align*}
\qed
\\
b) $$\int_{-1}^{1}z^{\frac{1}{2}} \D z$$
\textit{Solution:} \\
We want to parameterize this once again using $z = r\E^{i\theta}$ where $\theta \in [-\pi,\pi)$. Now our integral is
\begin{align*}
\int_{-1}^{1} z^{\frac{1}{2}} \D z &= \int_{-\pi}^{0} \left(\E^{i\theta}\right)^{\frac{1}{2}} i \E^{i\theta} \D \theta \\
	&= \int_{-\pi}^{0} i\E^{\frac{i\theta}{2}} \E^{i\theta} \D \theta \\
	&= \int_{-\pi}^{0} i\E^{\frac{i3}{2}\theta} \D \theta \\
	&= \left. \frac 2 3 \E^{\frac{i3}{2}\theta} \right|_{-\pi}^{0} \\
	&= \frac 2 3 \E^{ - \frac{i3}{2} \pi} - \frac 2 3. \\
\end{align*}
Now remembering our branch cut limits $\theta$ to be within $[-\pi, \pi)$ we change the angle $-\frac 3 2 \pi$ to be $\frac 1 2 \pi$.
Hence,
\begin{align*}
\frac 2 3 \E^{ - \frac{i3}{2} \pi} - \frac 2 3 &= \frac 2 3 \E^{ - \frac{i2}{2} \pi} \E^{ - \frac{i\pi}{2}} - \frac 2 3 \\
	&= \frac 2 3 \E^{ \frac{i\pi}{2}} - \frac 2 3 \\
	&= \frac 2 3 \left( i - 1\right).
\end{align*}
\qed
\\
\item From A\&F: 2.4.7 \\
Let C be an open (upper) semicircle of radius $R$ with its center at the origin, and consider $\int_C f(z) \D z$.
 Let $f(z) = \frac{1}{z^2 + a^2}$ for a real $a > 0$.
Show that $\left| f(z) \right| \leq \frac{1}{R^2 - a^2}$, $R > a$, and
$$
\left| \int_C f(z) dz \right| \leq \frac{\pi R}{R^2 - a^2}, \quad R > a.
$$
\textit{Solution:} \\
First, we want to show
$$
\left| f(z) \right| \leq \frac{1}{R^2 - a^2}
$$
where $R > a > 0$ and a $\in \mathbb R$.
Let's consider the function more closely
\begin{align*}
f(z) =  \frac{1}{z^2 + a^2} &= \frac{1}{(x + \I y)^2 + a^2} \\
	&= \frac{1}{x^2 + 2ixy - y^2 + a^2} \\
	&= \frac{1}{x^2 - y^2 + a^2 + i2xy }. \\
\end{align*}
Notice, we can write the real and imaginary parts of the complex number in the denominator as functions $u(x, y)$ and $v(x, y)$.
Where $u(x, y) = x^2 - y^2 + a^2$ and $v(x, y) = 2xy$.
Now we get 
$$
f(z) = \frac{1}{x^2 - y^2 + a^2 + i2xy }	 = \frac{u - iv }{u - iv }\frac{1}{u + iv } = \frac{u - iv }{u^2 + v^2 }
$$
Then we calculate
\begin{align*}
\left| f(z) \right| &= \left| \frac{u - iv }{u^2 + v^2 } \right| \\
	&= \left| \frac{u }{u^2 + v^2 } - i\frac{v}{u^2 + v^2 } \right| \\
	&= \sqrt{
		\left(\frac{u }{u^2 + v^2 }\right)^2 + \left(\frac{v }{u^2 + v^2 }\right)^2
	} \\
	&= \sqrt{
		\frac{u^2 }{\left( u^2 + v^2 \right)^2} + \frac{v^2 }{\left( u^2 + v^2 \right)^2}
	} \\
	&= \sqrt{
		\frac{u^2 + v^2 }{\left( u^2 + v^2 \right)^2}
	} \\
	&= \sqrt{
		\frac{1}{u^2 + v^2}
	} \\
	&= \frac{1}{\sqrt{u^2 + v^2}}.
\end{align*}
If we plug our substitution back in we see
\begin{align*}
\frac{1}{\sqrt{u^2 + v^2}} &= \frac{1}{\sqrt{\left(x^2 - y^2 + a^2 \right)^2 + \left(2xy \right)^2}} \\
	&= \frac{1}{\sqrt{\left(x^2 - y^2 + a^2 \right)\left(x^2 - y^2 + a^2 \right) + 4x^2y^2}} \\
	&= \frac{1}{
		\sqrt{
			x^4 - x^2y^2 + x^2a^2 -x^2y^2 + y^4 - y^2a^2 + x^2a^2 - y^2a^2 + a^4 + 4x^2y^2
		}
	} \\
	&= \frac{1}{
		\sqrt{
			x^4 + x^2a^2 + y^4 - y^2a^2 + x^2a^2 - y^2a^2 + a^4 + 2x^2y^2
		}
	}.
\end{align*}
Now we add zero in a particular fashion, namely $- 4x^2a^2 + 4x^2a^2 $, so we can regroup the terms and refactor to get closer to what we desire
\begin{align*}
	&= \frac{1}{
		\sqrt{
			x^4 + x^2a^2 + y^4 - y^2a^2 + x^2a^2 - y^2a^2 + a^4 + 2x^2y^2 +\left( - 4x^2a^2 + 4x^2a^2 \right)
		}
	} \\
	&= \frac{1}{
		\sqrt{
			x^4 + y^4 - y^2a^2 - y^2a^2 + a^4 + 2x^2y^2 - x^2a^2 - x^2a^2 + 4x^2a^2
		}
	} \\
	&= \frac{1}{\sqrt{\left(x^2 + y^2 - a^2\right)^2 + \left(2xa\right)^2}}.
\end{align*}
Using the fact that $\sqrt{a + b} \geq \sqrt{a}$ for $a, b > 0$, in our next step we get a smaller denominator which makes the overall expression greater or equal to the previous step. Note, equality only holds when $x=0$.
\begin{align*}
 \frac{1}{\sqrt{\left(x^2 + y^2 - a^2\right)^2 + \left(2xa\right)^2}}
 	&\leq \frac{1}{\sqrt{\left(x^2 + y^2 - a^2\right)^2}} \\
	&= \frac{1}{x^2 + y^2 - a^2} \\
	&= \frac{1}{|z|^2 - a^2} \\
	&= \frac{1}{R^2 - a^2}
\end{align*}
Therefore $\left| f(z) \right| \leq \frac{1}{R^2 - a^2}$. \\
\qed

\noindent
Next we wish to show that 
$$
\left| \int_C f(z) dz \right| \leq \frac{\pi R}{R^2 - a^2}, \quad R > a.
$$
By Theorem 2.4.2 from A\&F, if $f(z)$ is continuous on contour C, then
$$
\left| \int_C f(z) dz \right| \leq ML
$$
where $L$ is the length of $C$ and $M$ is an upper bound for $\left| f(z) \right|$.
We have that $C$ is continuous, since $a >0$ and $a < R$ there are no singularities or weirdness with $f(z)$ on the specified contour.
So we have 
$$
M = \frac{1}{R^2 - a^2}
$$
as we calculated in the first part of this problem.
Additionally, we know the arc length of $C$ is easy to calculate because it is half the circumference of the circle with radius $R$. Therefore,
$$
L = \int_a^b |z^\prime(t)| \D t = \frac 1 2 2 \pi R = \pi R.
$$
And thus
$$
\left| \int_C f(z) dz \right| \leq ML \leq \pi R\frac{1}{R^2 - a^2} = \frac{\pi R}{R^2 - a^2}.
$$
Hence, 
$$
\left| \int_C f(z) dz \right| \leq \frac{\pi R}{R^2 - a^2}
$$
as desired.
\qed
\\
\item From A\&F: 2.4.8 \\
Let $C$ be an arc of the circle $\left|z\right| = R$ with $(R > 1)$ of angle $\frac{\pi}{3}$.
Show that 
$$
\left| \int_C \frac{\D z}{z^3 + 1} \right| \leq \frac \pi 3 \left( \frac{R}{R^3 - 1} \right)
$$
and deduce
$$
\lim_{R \rightarrow \infty} \int_C \frac{\D z}{z^3 + 1} = 0
$$
\textit{Solution:} \\
Similar to the previous problem we will utilize Theorem 2.4.2 from A\&F. This time, our arc length of the contour $C$ is
$$
L = \frac 1 6 2\pi R = \frac{\pi}{3} R.
$$
Next, we need to calculate $M$ as the upper bound for $\left| \frac{1}{z^3 + 1} \right|$.
Let's follow a similar path as the previous problem
\begin{align*}
\left| f(z) \right| &= \left| \frac 1 {z^3 + 1} \right| \\
	&= \left| \frac 1 {\left( x + iy \right)^3 + 1} \right| \\
	&= \left| \frac 1 {x^3 -3y^2x + i3x^2y - iy^3 + 1} \right| \\
	&= \left| \frac 1 {\left(x^3 -3y^2x + 1\right) + i\left(3x^2y - y^3\right)} \right|. \\
\end{align*}
Using $u(x, y) = x^3 -3y^2x + 1$ and $v(x, y) = 3x^2y - y^3$ we have,
\begin{align*}
	&= \left| \frac 1 {u + iv} \right| \\
	&= \left| \frac{u - iv }{u^2 + v^2 } \right| \\
	&= \frac{1}{\sqrt{u^2 + v^2}}.
\end{align*}
Substituting back in we have
\begin{align*}
& \frac{1}{\sqrt{\left( x^3 -3y^2x + 1 \right)^2 + \left( 3x^2y - y^3 \right)^2}} \\
	&= \frac{1}{\sqrt{x^6 -3y^2x^4 + x^3 -3y^2x^4 + 9y^4x^2 - 3y^2x +  x^3 -3y^2x + 1 + \left( 3x^2y - y^3 \right)^2}} \\
	&= \frac{1}{\sqrt{x^6 -3y^2x^4 + x^3 -3y^2x^4 + 9y^4x^2 - 3y^2x +  x^3 -3y^2x + 1 + 9x^4y^2 - 6x^2y^4 + y^6 }} \\
	&= \frac{1}{\sqrt{x^6 + x^3 + 9y^4x^2 - 3y^2x +  x^3 -3y^2x + 1 -3y^2x^4 -3y^2x^4 + 9x^4y^2 - 6x^2y^4 + y^6 }} \\
	&= \frac{1}{\sqrt{x^6 + 2x^3 + 9x^2y^4 - 6xy^2 + 1 +3x^4y^2 - 6x^2y^4 + y^6 }} \\
	&= \frac{1}{\sqrt{x^6 + 2x^3 + 3x^2y^4 - 6xy^2 + 1 +3x^4y^2 + y^6 }} \\
	&= ...
\end{align*}
\textbf{TODO} Figure out another way, I don't think this is it.
\qed
\\

\item From A\&F: 2.5.1 b, e \\
Evaluate $\oint_C f(z)\D z$, where $C$ is the unit circle centered at the origin, and $f(z)$ is given by the following: \\
b) $$f(z) = \E^{z^2}$$ \\
\textit{Solution:} \\
e) $$f(z) = \frac{1}{2z^2 + 1} $$ \\
\textit{Solution:} \\
\item Use the ideas from A\&F: 2.5.5 to evaluate $\int_0^\infty \E^{\I
    z^3 t} \D z$, $t > 0$.  Express the result in terms of $\int_0^\infty \E^{-
    r^3} \D r$. \\
The ideas we might need to use are ... it's actually really long! \\
\textit{Solution:}\\
\item From A\&F: 2.5.6. \\
Consider the integral $$I = \int_{-\infty}^{\infty} \frac{\D x}{x^2 + 1}.$$
Show how to evaluate this integral by considering
$$\oint_{C_{(\mathbb R)}} \frac{\D z}{z^2 + 1},$$
where $C_{(\mathbb R)}$ is closed semicircle in the upper half plane with endpoints at $(-R, 0)$ and $(R, 0)$ plus the $x$-axis.
\textit{Hint:} use
$$\frac{1}{z^2 + 1} = -\frac{1}{2i}\left(\frac{1}{z + i} - \frac{1}{z - i}\right),$$
and show that the integral along the open semicircle in the upper half plane vanishes as $R \rightarrow \infty$.
Verify your answer by usual integration in real variables.
\textit{Solution:}\\
\\

\noindent
Repeat this exercise for
  \begin{align*}
    I_\epsilon = \int_{-\infty}^\infty \frac{\epsilon \D x}{x^2 +
    \epsilon^2}, \quad \epsilon > 0.
  \end{align*}\\
Seems like I am supposed to do 2.5.6 and then for the given integral as well. \\
\textit{Solution:}\\
\item Use a similar method to calculate
  $\int_{-\infty}^{\infty} \frac{d x}{1+x^4}$. \\
\textit{Solution:}\\

\item From A\&F: 2.6.1 a, e.\\
Evaluate the integrals $\oint_C f(z) \D z$, where $C$ is the unit circle centered at the origin and $f(z)$ is given by the following (use Eq. (1.2.19) as necessary): \\
a)
$$
\frac{\sin z}{z}
$$
\\
\textit{Solution:}\\
e)
$$
\E^{z^2}\left(\frac{1}{z^2} - \frac{1}{z^3}\right)
$$
\\
\textit{Solution:}\\
\end{enumerate}

\end{document}

%%% Local Variables:
%%% mode: latex
%%% TeX-master: t
%%% End:
