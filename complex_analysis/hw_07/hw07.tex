\documentclass[10pt]{amsart}
\usepackage[margin=1.4in]{geometry}
\usepackage[usenames,dvipsnames,cmyk]{xcolor} %load first
\usepackage{cancel}
\usepackage{graphicx,subfig}
\usepackage{mathtools}

\graphicspath{ {./images/} }

\usepackage{amssymb,amsmath,enumitem,url}

\newcommand{\D}{\mathrm{d}}
\newcommand{\I}{\mathrm{i}}
\DeclareMathOperator{\E}{e}
\DeclareMathOperator{\OO}{O}
\DeclareMathOperator{\oo}{o}
\DeclareMathOperator{\erfc}{erfc}
\DeclareMathOperator{\real}{Re}
\DeclareMathOperator{\imag}{Im}
\usepackage{tikz}
\usepackage[framemethod=tikz]{mdframed}
\theoremstyle{nonumberplain}

\mdtheorem[innertopmargin=5pt]{lemma}{Lemma}
\mdtheorem[innertopmargin=-5pt]{sol}{Solution}
%\newmdtheoremenv[innertopmargin=-5pt]{sol}{Solution}
\definecolor{MichiganBlue}{HTML}{00274C}
\definecolor{MichiganYellow}{HTML}{FFCB05}  
\definecolor{NicePurple}{RGB}{75,56,76} %PrincePurple
\definecolor{NiceRed}{RGB}{230,37,52}
\definecolor{MidnightBlue}{rgb}{0.1, 0.1, 0.44}
\usepackage[colorlinks=true, linkcolor=MidnightBlue, citecolor=MidnightBlue, urlcolor=MidnightBlue]{hyperref}

\begin{document}
\pagestyle{empty}

\newcommand{\mline}{\vspace{.2in}\hrule\vspace{.2in}}


\noindent
\text{Hunter Lybbert} \\
\text{Student ID: 2426454} \\
\text{11-12-24} \\
\text{AMATH 567} \\

\title{\bf { Homework 7} }


\maketitle
\noindent
Collaborators*: Nate Ward, Sophia Kamien, Laura Thomas, Cooper Simpson,  \\
\\
\tiny
\text{*Listed in no particular order. And anyone I discussed at least part of one problem with is considered a collaborator.}
\normalsize


\mline
\begin{enumerate}[label={\bf {\arabic*}:}]
\item  From A\&F: 3.5.1 b, c, d (Only consider singularities in the finite
  complex plane) \\
Discuss the type of singularity (removable, pole and order, essential, branch cluster, natural barrier, etc.); if the type is a pole give the strength of the pole,  and give the nature (isolated or not) of all singular points associated with the following functions. \\

\noindent
(b)
$$
f(z) = \frac {\E^{2z} - 1} {z^2}
$$
\textit{Solution:} \\
Begin by writing the Taylor series expansion of the numerator, simplifying, and reindexing
\begin{align*}
f(z) = \frac {\E^{2z} - 1} {z^2} &= \frac 1 {z^2} \sum_{j=0}^\infty \frac{(2z)^j} {j!} - 1 \\
	&= \frac 1 {z^2} \sum_{j=1}^\infty \frac{2^jz^j} {j!} \\
	&= \frac 1 {z^2} \sum_{j=0}^\infty \frac{2^{j + 1}z^{j + 1}} {(j + 1)!} \\
	&= \sum_{j=0}^\infty \frac{2^{j + 1}} {(j + 1)!} \frac {z^{j + 1}}{z^2} \\
	&= \sum_{j=0}^\infty \frac{2^{j + 1}} {(j + 1)!} z^{j - 1} \\
	&= \sum_{j=-1}^\infty \frac{2^{j + 2}z^j} {(j + 2)!}.
\end{align*}
Therefore, $f(z)$ has an isolated simple pole at z = 0.
Calculating the strength of the pole we get
$$C_{-1} = \frac{2^{-1 + 2}} {(-1 + 2)!} = 2$$
The strength of the pole is 2.
\qed \\
\newpage

\noindent
(c)
$$
f(z) = \E^{\tan z}
$$
\textit{Solution:} \\
Let's first look for where the denominator is undefined.
Notice,
\begin{align*}
f(z) &= \E^{\tan z} \\
	&= \sum_{n=0}^\infty \frac {\tan^n z} {n!} \\
	&= \sum_{n=0}^\infty \frac 1 {n!} \frac {\sin^n z} {\cos^n z}.
\end{align*}
Therefore, $f(z)$ has singularities at $z = \pi/2 + \pi k$ for $k \in \mathbb Z$.
Further expanding $\tan z$ is enlightening.
We can do this quickly utilizing the expansion given in A \& F in example 3.5.3 this gives us
\begin{align*}
f(z) &= \E^{\tan z} \\
	&= \sum_{n=0}^\infty \frac {\tan^n z} {n!} \\
	&= \sum_{n=0}^\infty \frac 1 {n!} \left( \frac 1 {z - (\pi/2 + m\pi)} + \frac 1 3 (z - (\pi/2 + m \pi) + ...\right)^n.
\end{align*}
Foiling this out and since $n$ goes to $\infty$ there will be an infinite number of negative powers of $z$ so these singularities are all essential. \\
\qed \\

\noindent
(d)
$$
f(z) = \frac {z^3}{z^2 + z + 1}
$$
\textit{Solution:} \\
We can write $f(z)$ in the form
\begin{align*}
f(z) = \frac {z^3}{z^2 + z + 1} = \frac {z^3}{\left( z - \left(-1/2 + \I\frac{\sqrt{3}}{2} \right) \right)\left( z - \left(-1/2 - \I\frac{\sqrt{3}}{2} \right)\right)}
\end{align*}
This reveals the two poles at $z=z_1,z_2$ where $z_1= -1/2 + \I\frac{\sqrt{3}}{2}$ and $z_2= -1/2 - \I\frac{\sqrt{3}}{2}$.
Now let's compute the strength of each of these poles in turn.
We begin by expanding $f(z)$ around $z=z_1$
\begin{align*}
f(z) &= \frac {1}{ z - z_1}z^3( z - z_2)^{-1} \\
	&= \frac {1}{ z - z_1}
		\left(\frac{z_1^3}{0!} + \frac{3z_1^2}{1!} z + \frac{6z_1}{2!}z^2 + \frac 6 {3!} z^3 + ... \right)
		\left((z_1 - z_2)^{-1} - \frac {( z_1 - z_2)^{-2}}{1!}z + \frac {2 ( z_1 - z_2)^{-3}}{2!}z^2  + ...\right) \\
	&= \frac {1}{ z - z_1}
		\left( z_1^3 \left((z_1 - z_2)^{-1} - ( z_1 - z_2)^{-2}z + ( z_1 - z_2)^{-3}z^2  + ...\right) + ... \right) \\
	&= \frac {1}{ z - z_1}
		\left( z_1^3(z_1 - z_2)^{-1} - z_1^3( z_1 - z_2)^{-2}z + z_1^3( z_1 - z_2)^{-3}z^2  + ... \right) \\
	&= \frac {1}{ z - z_1} z_1^3(z_1 - z_2)^{-1} - \frac {1}{ z - z_1} z_1^3( z_1 - z_2)^{-2}z + \frac {1}{ z - z_1} z_1^3( z_1 - z_2)^{-3}z^2  + ...
\end{align*}
Then we can calculate the strength of the pole at $z_1$ as
\begin{align*}
z_1^3(z_1 - z_2)^{-1}
	&= \left(-1/2 + \I\frac{\sqrt{3}}{2}\right)^3 \left(-1/2 + \I\frac{\sqrt{3}}{2} - \bigg(-1/2 - \I\frac{\sqrt{3}}{2}\bigg) \right)^{-1} \\
	&= \left(-1/2 + \I\frac{\sqrt{3}}{2}\right)^3 \left(\cancel{-1/2} + \I\frac{\sqrt{3}}{2} + \cancel{1/2} + \I\frac{\sqrt{3}}{2}\right)^{-1} \\
	&= \left(-1/2 + \I\frac{\sqrt{3}}{2}\right)^3 \left(\I\sqrt{3}\right)^{-1} \\
	&= \left( (-1/2)^3 + 3(-1/2)^2\I\frac{\sqrt{3}}{2} + 3(-1/2)\bigg(\I\frac{\sqrt{3}}{2}\bigg)^2 + \bigg(\I\frac{\sqrt{3}}{2}\bigg)^3 \right) \left(\I\sqrt{3}\right)^{-1} \\
	&=\left(  -\frac 1 8 + \cancel{\I\frac{3\sqrt{3}}{8}} + \frac{9}{8} - \cancel{ \I \frac{3\sqrt{3}}{8}} \right) \left(\I\sqrt{3}\right)^{-1} \\
	&= \frac 1 {\I \sqrt{3}}.
\end{align*}

Furthermore, the strength of the pole at $z_2$ is
\begin{align*}
z_2^3(z_2 - z_1)^{-1}
	&= \left(-1/2 - \I\frac{\sqrt{3}}{2}\right)^3 \left(-1/2 - \I\frac{\sqrt{3}}{2} - \bigg(-1/2 + \I\frac{\sqrt{3}}{2}\bigg) \right)^{-1} \\
	&= \left(-1/2 - \I\frac{\sqrt{3}}{2}\right)^3 \left(\cancel{-1/2} - \I\frac{\sqrt{3}}{2} + \cancel{1/2} - \I\frac{\sqrt{3}}{2}\right)^{-1} \\
	&= \left(-1/2 - \I\frac{\sqrt{3}}{2}\right)^3 \left(- \I\sqrt{3}\right)^{-1} \\
	&= \left( (-1/2)^3 - 3(-1/2)^2\I\frac{\sqrt{3}}{2} + 3(-1/2)\bigg(- \I\frac{\sqrt{3}}{2}\bigg)^2 + \bigg( -\I\frac{\sqrt{3}}{2}\bigg)^3 \right) \left(- \I\sqrt{3}\right)^{-1} \\
	&=\left(  -\frac 1 8 - \cancel{\I\frac{3\sqrt{3}}{8}} + \frac{9}{8} + \cancel{ \I \frac{3\sqrt{3}}{8}} \right) \left(-\I\sqrt{3}\right)^{-1} \\
	&= -\frac 1 {\I \sqrt{3}}.
\end{align*}
Therefore the function $f(z) = \frac {z^3}{z^2 + z + 1}$
has two simple poles at
\begin{align*}
z_1 &= -1/2 + \I\frac{\sqrt{3}}{2} \text{ and } \\
z_2 &= -1/2 - \I\frac{\sqrt{3}}{2}.
\end{align*}
With strengths of 
$$\frac 1 {\I \sqrt{3}} \text{ and } -\frac 1 {\I \sqrt{3}},$$
respectively. \\
\qed \\

\newpage

\item From A\&F: 3.5.3 a, c, d \\
Show that the functions below are meromorphic; that is, the only singularities in the finite $z$ plane are poles.
Determine the location, order and strength of the poles. 

\noindent
(a)
$$
f(z) = \frac z {z^4 + 2}
$$
\textit{Solution:} \\
In order to compute the partial fractions, I make use of the partial fractions calculated in homework set 4 problem 8 with some slight modifications due to the 2 that is present and the z in the numerator.
Let's begin
\begin{align*}
f(z) &= \frac z {z^4 + 2} \\
	&= \frac z {(z^2 - \I\sqrt{2})(z^2 + \I\sqrt{2})} \\
	&= \frac z {
		(z - \sqrt{\I}\sqrt[4]{2})
		(z + \sqrt{\I}\sqrt[4]{2})
		(z - \I\sqrt{\I}\sqrt[4]{2})
		(z + \I\sqrt{\I}\sqrt[4]{2})
	} \\
	&= - \frac {z \sqrt[4]2} {8\sqrt{\I}} \left( \frac{1}{z + \I\sqrt{\I} \sqrt[4]{2}} \right)
		+ \frac {z \sqrt[4]2} {8\sqrt{\I}} \left(\frac{1}{z - \I\sqrt{\I} \sqrt[4]{2}} \right) \\
		& \quad \quad - \frac {z \sqrt[4]2} {8\I\sqrt{\I}} \left( \frac{1}{z + \sqrt{\I} \sqrt[4]{2}} \right)
		+ \frac {z \sqrt[4]2} {8\I\sqrt{\I}} \left( \frac{1}{z - \sqrt{\I} \sqrt[4]{2}} \right).
\end{align*}
Therefore, $f(z)$ has { \bf four}  { \bf simple } (order 1) poles at the following locations with given { \bf strengths }
\begin{align*}
z_1 &= - \I\sqrt{\I} \sqrt[4]{2}
	& \text{ with strength }
	& - \frac {\left( - \I\sqrt{\I} \sqrt[4]{2} \right) \sqrt[4]2} {8\sqrt{\I}}
		= \frac {\I \sqrt{2} } {8} \\
z_2 &= \I\sqrt{\I} \sqrt[4]{2}
	& \text{ with strength }
	& \frac {\left( i\sqrt{i} \sqrt[4]{2} \right) \sqrt[4]2} {8\sqrt{i}}
		= \frac {i \sqrt{2} } {8} \\
z_3 &= - \sqrt{\I} \sqrt[4]{2}
	& \text{ with strength }
	& - \frac {\left(- \sqrt{\I} \sqrt[4]{2} \right) \sqrt[4]2} {8 \I\sqrt{\I}}
		= \frac {\sqrt{2} } {8 \I} 
		= - \frac { \I \sqrt{2} } {8} \\
z_4 &= \sqrt{\I} \sqrt[4]{2}
	& \text{ with strength }
	& \frac {\left(\sqrt{\I} \sqrt[4]{2} \right) \sqrt[4]2} {8 \I \sqrt{\I}}
		= \frac {\sqrt{2} } {8 \I} 
		= - \frac { \I \sqrt{2} } {8} \\
\end{align*} \qed \\
\newpage

\noindent
(c)
$$
f(z) = \frac z {\sin^2 z}
$$
\textit{Solution:} \\
We first do a Taylor series expansion around $z= 0$
\begin{align*}
f(z) = \frac z {\sin^2 z}
	&= \frac z {\left(\sum_{j=0}^{\infty} \frac {(-1)^j z^{2j + 1}}{(2j + 1)!} \right)^2} \\
	&= \frac z {z^2 \left(\sum_{j=0}^{\infty} \frac {(-1)^j z^{2j}}{(2j + 1)!} \right)^2} \\
	&= \frac 1 {z \left(\sum_{j=0}^{\infty} \frac {(-1)^j z^{2j}}{(2j + 1)!} \right)^2} \\
	&= \frac {\frac{1}{\left(\sum_{j=0}^{\infty} \frac {(-1)^j z^{2j}}{(2j + 1)!} \right)^2}} {z }.
\end{align*}
The strength of this simple pole at $z = 0$
$$
\frac{1}{\left(\sum_{j=0}^{\infty} \frac {(-1)^j 0^{2j}}{(2j + 1)!} \right)^2}
	= \frac{1}{\left( 1 + \cancel{\sum_{j=1}^{\infty} \frac {(-1)^j 0^{2j}}{(2j + 1)!}} \right)^2}
	= 1.
$$


\noindent
Next we do a Taylor series expansion around $z= n\pi$
\begin{align*}
f(z) = \frac z {\sin^2 z}
	&= \frac z {\left(\sum_{j=0}^{\infty} \frac {(-1)^j (z - n\pi)^{2j + 1}}{(2j + 1)!} \right)^2} \\
	&= \frac z {(z - n\pi)^2 \left(\sum_{j=0}^{\infty} \frac {(-1)^j (z - n\pi)^{2j}}{(2j + 1)!} \right)^2} \\
	&= \frac {\frac z {\left(\sum_{j=0}^{\infty} \frac {(-1)^j (z - n\pi)^{2j}}{(2j + 1)!} \right)^2}} {(z - n\pi)^2}.
\end{align*}
The strength of the $2^{\rm nd}$ order pole at $z = n \pi$ for all $n \in \mathbb Z$ is
$$
\frac z {\left(\sum_{j=0}^{\infty} \frac {(-1)^j (z - n\pi)^{2j}}{(2j + 1)!} \right)^2}
	= \frac {n \pi} {\left( 1 + \cancel{\sum_{j=1}^{\infty} \frac {(-1)^j (n \pi - n\pi)^{2j}}{(2j + 1)!}} \right)^2}
	= n \pi.
$$
\qed \\
\newpage

\noindent
(d)
$$
f(z) = \frac {\E^z - 1 - z}{z^4}
$$
\textit{Solution:} \\
Taylor expand the numerator, simplify, and reindex
\begin{align*}
f(z) = \frac {\E^z - 1 - z}{z^4} &= \frac {\sum_{j=0}^\infty \frac {z^j} {j!} - 1 - z}{z^4} \\
	&= \frac 1 {z^4} \sum_{j=2}^\infty \frac {z^j} {j!} \\
	&= \frac 1 {z^4} \sum_{j=0}^\infty \frac {z^{j + 2}} {(j + 2)!} \\
	&= \sum_{j=0}^\infty \frac {z^{j - 2}} {(j + 2)!} \\
	&= \sum_{j=-2}^\infty \frac {z^{j}} {(j + 4)!}
\end{align*}
The pole at $z = 0$ is of order 2 and the strength is
$$
C_{-2} = \frac 1 {(-2 + 4)!} = \frac 1 2.
$$
\qed \\
\newpage

\item Introducing the Gamma function: Do A\&F: 3.6.6. \\

\noindent
Let $\Gamma(z)$ be given by
$$
\frac 1 {\Gamma(z)} = z \E^{\gamma z} \prod_{n=1}^\infty \left( 1 + \frac z n \right) \E^{-z / n}
$$
for $z \neq 0, -1, -2, ...$ and $\gamma = $ constant (probably real). \\

\noindent
(a) Show that 
$$
\frac {\Gamma^\prime(z)}{\Gamma(z)} = - \frac 1 z - \gamma - \sum_{n = 1}^\infty \left( \frac 1 {z + n} - \frac 1 n\right).
$$
\textit{Solution:} \\
Let's begin by looking at the $\log$ of each side (let the right hand side be $g(z)$)
\begin{align*}
\log \left( \frac 1 {\Gamma(z)} \right) &= \log g(z) \\
\log 1 - \log \Gamma(z)  &= \log g(z) \\
\frac d {d z} - \log\Gamma(z) &= \frac d {dz} \log g(z) \\
\frac d {d z} \log \Gamma(z) &=  - \frac d {dz} \log g(z) \\
\frac {\Gamma^\prime(z)}{\Gamma(z)} &=  - \frac d {dz} \log g(z).
\end{align*}
Now, simplify $\log g(z)$, plugging back in the expression $g(z)$ represents
\begin{align*}
\log g(z) &= \log \left( z \E^{\gamma z} \prod_{n=1}^\infty \left( 1 + \frac z n \right) \E^{-z / n} \right) \\
	&= \log z + \gamma z + \sum_{n=1}^\infty \left( \log \left(  \frac {n + z} n \right)  - \frac z  n \right) \\
	&= \log z + \gamma z + \sum_{n=1}^\infty \bigg( \log (n + z) - \log n  - \frac z  n \bigg).
\end{align*}
Finally, take the derivative and negate the resulting expression
\begin{align*}
\frac {\Gamma^\prime(z)}{\Gamma(z)} &= - \frac d {dz} \log g(z) \\
	&= - \frac d {dz}\left( \log z + \gamma z + \sum_{n=1}^\infty \bigg( \log (n + z) - \log n  - \frac z  n \bigg) \right) \\
	&= - \frac 1 z - \gamma - \frac d {dz}\sum_{n=1}^\infty \bigg( \log (n + z) - \log n  - \frac z  n \bigg) \\
	&= - \frac 1 z - \gamma - \sum_{n=1}^\infty \bigg( \frac 1 {z + n} - \frac 1 n \bigg).
\end{align*}
\qed \\

\noindent
(b) Show that
\begin{align}
\frac {\Gamma^\prime(z + 1)} {\Gamma(z + 1)} - \frac {\Gamma^\prime(z)} {\Gamma(z)} - \frac 1 z = 0.
\label{eq:gamma_thing}
\end{align}
\textit{Solution:} \\
We can plug in $z + 1$ to the formula we just computed and simplify
\begin{align*}
&\frac {\Gamma^\prime(z + 1)} {\Gamma(z + 1)} - \frac {\Gamma^\prime(z)} {\Gamma(z)} - \frac 1 z \\
	& \quad = - \frac 1 {z + 1} - \gamma - \sum_{n=1}^\infty \bigg( \frac 1 {z + 1 + n} - \frac 1 n \bigg) - \left(- \frac 1 z - \gamma - \sum_{n=1}^\infty \bigg( \frac 1 {z + n} - \frac 1 n \bigg) \right) - \frac 1 z \\
	& \quad = - \frac 1 {z + 1} - \cancel{\gamma} - \sum_{n=1}^\infty \bigg( \frac 1 {z + 1 + n} - \frac 1 n \bigg) + \cancel{\frac 1 z} + \cancel{\gamma} + \sum_{n=1}^\infty \bigg( \frac 1 {z + n} - \frac 1 n \bigg) - \cancel{\frac 1 z} \\
	& \quad = - \frac 1 {z + 1} + \sum_{n=1}^\infty \bigg( - \frac 1 {z + 1 + n} + \frac 1 n \bigg) + \sum_{n=1}^\infty \bigg( \frac 1 {z + n} - \frac 1 n \bigg) \\
\end{align*}
Now, we want to combine these two sums but in order to do so we need to ensure that they are each convergent.
We demonstrate this briefly using the comparison test.
For the series on the left we have
\begin{align*}
& - \frac 1 {z + 1 + n} + \frac 1 n = \frac {z + 1} {nz + n + n^2} \\
& \quad \implies \left| \frac {z + 1} {nz + n + n^2}\right| \leq \frac {|z| + 1} {n^2} \leq \frac 1 {n^2}
\end{align*}
which gives us the series $\sum_{n=0}^\infty \frac 1 {n^2} < \infty$.
Therefore our series we are concerned with is convergent as well.
Furthermore, the series on the right we have
\begin{align*}
& \frac 1 {z + n} - \frac 1 n = \frac {- z} {nz + n^2} \\
& \quad \implies \left| \frac {-z} {nz + n^2}\right| \leq \frac {|z|} {n^2}
\end{align*}
which gives us the series $|z| \sum_{n=0}^\infty \frac 1 {n^2} < \infty$.
Therefore our series we are concerned with is convergent as well.
Finally, we are justified to combine these.
Let's verify that their combination is also convergent
\begin{align*}
& - \frac 1 {z + 1 + n} + \frac 1 {z + n} = \frac {- z - n + z + 1 + n} {(z + 1 + n)(z + n)} = \frac {1} {(z + 1 + n)(z + n)} \\
& \quad \implies \left| \frac {1} {(z + 1 + n)(z + n)} \right| = \left| \frac {1} {z^2 + z + 2nz + n + n^2} \right| \leq \frac 1 {n^2}
\end{align*}
which gives us the series $\sum_{n=0}^\infty \frac 1 {n^2} < \infty$.

\noindent
Picking up where we left off we have
\begin{align*}
&\frac {\Gamma^\prime(z + 1)} {\Gamma(z + 1)} - \frac {\Gamma^\prime(z)} {\Gamma(z)} - \frac 1 z \\
	& \quad = - \frac 1 {z + 1} + \sum_{n=1}^\infty \bigg( - \frac 1 {z + 1 + n} + \frac 1 n \bigg) + \sum_{n=1}^\infty \bigg( \frac 1 {z + n} - \frac 1 n \bigg) \\
	& \quad = - \frac 1 {z + 1} + \sum_{n=1}^\infty \bigg( - \frac 1 {z + 1 + n} + \cancel{\frac 1 n} + \frac 1 {z + n} - \cancel{\frac 1 n} \bigg) \\
	& \quad = - \frac 1 {z + 1} + \sum_{n=1}^\infty \bigg(\frac 1 {z + n}  - \frac 1 {z + 1 + n} \bigg) \\
	& \quad = - \frac 1 {z + 1} + \bigg(\frac 1 {z + 1}  - \cancel{\frac 1 {z + 2}} + \cancel{\frac 1 {z + 2}}  - \cancel{\frac 1 {z + 3}} + \cancel{\frac 1 {z + 3}}  - \frac 1 {z + 4} + ... \bigg) \\
	& \quad = - \frac 1 {z + 1} + \frac 1 {z + 1} \\
	& \quad = 0
\end{align*}
\qed \\
Whereupon
$$ \Gamma(z + 1) = Cz\Gamma(z), \quad \text{for a constant $C$}. $$
\textit{Solution:} \\
We can show this by integrating equation (\ref{eq:gamma_thing})
\begin{align*}
\int \frac {\Gamma^\prime(z + 1)} {\Gamma(z + 1)} - \frac {\Gamma^\prime(z)} {\Gamma(z)} - \frac 1 z &= \int 0 \\
\log \Gamma(z + 1) - \log {\Gamma(z)} - \log z - C &= 0 \\
\log \Gamma(z + 1) &= C + \log z + \log {\Gamma(z)} \\
\log \Gamma(z + 1) &= C + \log (z{\Gamma(z)}) \\
\E^{\log \Gamma(z + 1)} &= \E^{C + \log (z{\Gamma(z)})} \\
\Gamma(z + 1) &= \E^C z\Gamma(z) \\
\Gamma(z + 1) &= C^\prime z\Gamma(z)
\end{align*}
where $C^\prime = \E^C$ is a constant. \\
\qed \\
\newpage

\noindent
(c) Show that $\lim_{z\rightarrow 0} z \Gamma(z) = 1$ to find that $C = \Gamma(1)$. \\

\noindent
\textit{Solution:} \\
If I can show the limit of the reciprocal is 1, then the limit of the original function will also be 1.
Now actually taking the limit we have
\begin{align*}
\lim_{z\rightarrow 0} \frac 1 {z \Gamma(z)}
	&= \lim_{z\rightarrow 0} \frac 1 z z \E^{\gamma z} \prod_{n=1}^\infty \left( 1 + \frac z n \right) \E^{-z / n} \\
	&= \lim_{z\rightarrow 0} \E^{\gamma z} \prod_{n=1}^\infty \left( 1 + \frac z n \right) \E^{-z / n} \\
	&= \E^{0} \prod_{n=1}^\infty \left( 1 + \frac 0 n \right) \E^{-0 / n} \\
	&= \prod_{n=1}^\infty \left( 1\right) \\
	&= 1.
\end{align*}
Now if we take the limit of our previous expression we have from part (c)
\begin{align*}
\lim_{z \rightarrow 0} \Gamma(z + 1) &= \lim_{z \rightarrow 0} Cz\Gamma(z) \\
\Gamma(1) &= C
\end{align*}
\qed \\

\noindent
(d) Determine the following representation of the constant $\gamma$ so that $\Gamma(1) = 1$
$$
\E^{-\gamma} = \prod_{n = 1}^\infty \left(1 + \frac 1 n\right)\E^{-1/n}
$$
\textit{Solution:} \\
We want to find $\E^{-\gamma}$ s.t. $\Gamma(1) = 1$.
Therefore we need
\begin{align*}
\frac 1 {\Gamma(z)} &= z \E^{\gamma z} \prod_{n=1}^\infty \left( 1 + \frac z n \right) \E^{-z / n} \\
\frac 1 {\Gamma(1)} &= \E^\gamma \prod_{n=1}^\infty \left( 1 + \frac 1 n \right) \E^{-1 / n} \\
1 &= \E^\gamma \prod_{n=1}^\infty \left( 1 + \frac 1 n \right) \E^{-1 / n} \\
\frac 1 {\E^\gamma} &= \prod_{n=1}^\infty \left( 1 + \frac 1 n \right) \E^{-1 / n} \\
\E^{-\gamma} &= \prod_{n=1}^\infty \left( 1 + \frac 1 n \right) \E^{-1 / n}
\end{align*}
\qed \\
\newpage

\noindent
(e) Show that
\begin{align*}
\prod_{n=1}^\infty \left(1 + \frac 1 n \right)\E^{-1/n}
	&= \lim_{n\rightarrow\infty} \frac 2 1 \frac 3 2 \frac 4 3 ... \frac {n + 1} n \E^{-S(n)} \\
	&= \lim_{n\rightarrow\infty} (n + 1) \E^{-S(n)}
\end{align*}
where $S(n) = 1 + \frac 1 2 + \frac 1 3 + \frac 1 4 + ... + \frac 1 n = \sum_{\ell=1}^n \frac 1 \ell$. \\

\noindent
\textit{Solution:} \\
We begin by truncating the infinite product to $n$ and take a limit as $n \rightarrow \infty$, 
\begin{align*}
\prod_{n=1}^\infty \left(1 + \frac 1 n \right)\E^{-1/n} 
	&= \prod_{n=1}^\infty \left(\frac {n + 1} n \right)\E^{-1/n} \\
	&= \lim_{n\rightarrow\infty}
		\left(\frac {2} 1 \E^{-1} \right)
		\left(\frac {3} 2 \E^{-1/2} \right)
		\left(\frac {4} 3 \E^{-1/3} \right) ...
		\left(\frac {n} {n - 1} \E^{-1/(n - 1)} \right)
		\left(\frac {n + 1} n \E^{-1/n} \right) \\
	&= \lim_{n\rightarrow\infty}
		\left(\frac {2} 1 \frac {3} 2 \frac {4} 3 ... \frac {n} {n - 1}\frac {n + 1} n \right)
		\left( \E^{-1} \E^{-1/2}\E^{-1/3} ... \E^{-1/n} \right) \\
	&= \lim_{n\rightarrow\infty}
		\left(\frac {2} 1 \frac {3} 2 \frac {4} 3 ... \frac {n} {n - 1}\frac {n + 1} n \right)
		\E^{-1 -1/2 -1/3 ... -1/n} \\
	&= \lim_{n\rightarrow\infty}
		\left(\frac {2} 1 \frac {3} 2 \frac {4} 3 ... \frac {n} {n - 1}\frac {n + 1} n \right)
		\E^{-\sum_{\ell=0}^n 1/\ell} \\
	&= \lim_{n\rightarrow\infty}
		\left(\frac {2} 1 \frac {3} 2 \frac {4} 3 ... \frac {n} {n - 1}\frac {n + 1} n \right) \E^{-S(n)}.
\end{align*}
Notice, we can pair up and cancel out each term in the product in the numerator and in the denominator except for a $1$ in the denominator and the $n + 1$ in the numerator
\begin{align*}
\lim_{n\rightarrow\infty}
	\left(\frac {\cancel{2}} 1 \frac {\cancel{3}} {\cancel{2}} \frac {\cancel{4}} {\cancel{3}} ... \frac {\cancel n} {\cancel{n - 1}}
	\frac {n + 1} {\cancel n} \right) \E^{-S(n)}
&= \lim_{n\rightarrow\infty} (n + 1) \E^{-S(n)}.
\end{align*} \qed \\

\noindent
Consequently, obtain the limit
$$
\gamma = \lim_{n\rightarrow\infty} \Bigg(  \sum_{k=1}^n \frac 1 k - \log(n + 1)  \Bigg).
$$
\textit{Solution:} \\
This comes from taking the log of both sides of the following:
\begin{align*}
\E^{-\gamma} &= \lim_{n\rightarrow\infty} (n + 1) \E^{-S(n)} \\
\log \E^{-\gamma} &= \log\big( \lim_{n\rightarrow\infty} (n + 1) \E^{-S(n)} \big) \\
-\gamma &= \lim_{n\rightarrow\infty} \log\big( (n + 1) \E^{-S(n)} \big) \\
\gamma &= - \bigg(\lim_{n\rightarrow\infty} \log (n + 1) + \log \big(\E^{-S(n)}\big) \bigg) \\
\gamma &= - \bigg(\lim_{n\rightarrow\infty} \log (n + 1) - S(n) \bigg) \\
\gamma &= \lim_{n\rightarrow\infty} \Bigg (\sum_{\ell = 0}^n \frac 1 \ell - \log (n + 1) \Bigg ).
\end{align*} \qed \\

\noindent
This is the same Gamma function you may have seen defined as
$$
\Gamma(z)=\int_0^{\infty} t^{z-1} e^{-t} d t
$$
This better known representation is only valid for
$\operatorname{Re}(z)>0$. The representation given here is valid in
all of $\mathbb{C}$. It takes a bit of work to show that our
representation is an analytic continuation of the integral
representation (this requires the Dominated Convergence Theorem), but
it is quite doable. Not now though. \\
\newpage

\item Consider a sequence of numbers $(a_n)_{n \geq 0}$ such that $|a_n| < 1$ and
$$ \sum_{n = 0}^\infty (1 - |a_n|) < \infty. $$
Define a Blaschke factor
\begin{align*}
B(a,z) =
	\begin{cases}
		\frac{|a|}{a} \frac{ a - z}{ 1 - \bar a z} & a \neq 0,\\
  		z & a  =0.
  	\end{cases}
\end{align*}
\begin{itemize}
\item Show that
\begin{align*}
H(z) = \prod_{n=0}^\infty B(a_n,z),
\end{align*}
defines an analytic function in the open unit disk $|z| < 1$. \\

\noindent
\textit{Solution:} \\
As we are working through the problem let's add a little notation.
Let $|z| \leq R < 1$ and $a_n = (1 - r_n)\E^{\I \theta_n}$.
Additionally, from our setup we can deduce that $a_n \rightarrow 1$ and $r_n \rightarrow 0$ as $n$ goes to $\infty$.
Suppose, some of the terms in the sequence of $a_n$'s are 0 for which $B(0, z) = z$.
Since eventually the sequence converges to 1, then there must exist some $N$ such that when $n > N$ every $a_n$ is non zero.
Thus we can look at
$$
H(z) = \prod_{n=0}^\infty B(a_n,z) = \Bigg(\prod_{n=0}^N B(a_n,z) \Bigg) \Bigg(\prod_{n = N + 1}^\infty B(a_n,z) \Bigg).
$$
Let's rewrite the infinite product portion of the right hand side in a helpful way
\begin{align*}
H(z) = \prod_{n=N + 1}^\infty B(a_n,z)
	&= \prod_{n=N + 1}^\infty \frac{|a_n|}{a_n} \frac{ a_n - z}{ 1 - \bar a_n z} \\
	&= \prod_{n=N + 1}^\infty \frac{|(1 - r_n)\E^{\I \theta_n}|}{(1 - r_n)\E^{\I \theta_n}} \frac{ (1 - r_n)\E^{\I \theta_n} - z}{ 1 - (1 - r_n)\E^{-\I \theta_n} z} \\
	&= \prod_{n=N + 1}^\infty \frac{\cancel{(1 - r_n)}}{\cancel{(1 - r_n)}\E^{\I \theta_n}} \frac{ (1 - r_n)\E^{\I \theta_n} - z}{ 1 -(1 - r_n)\E^{-\I \theta_n} z} \\
	&= \prod_{n=N + 1}^\infty \frac{ (1 - r_n)\E^{\I \theta_n} - z}{ \E^{\I \theta_n} - (1 - r_n)\E^{\I \theta_n}\E^{-\I \theta_n} z} \\
	&= \prod_{n=N + 1}^\infty \frac{ (1 - r_n)\E^{\I \theta_n} - z}{ \E^{\I \theta_n} - (1 - r_n) z} \\
	&= \prod_{n=N + 1}^\infty \bigg(1 + \frac{ (1 - r_n)\E^{\I \theta_n} - z}{ \E^{\I \theta_n} - (1 - r_n) z} - 1\bigg).
\end{align*}
Let $b_n(z) = \frac{ (1 - r_n)\E^{\I \theta_n} - z}{ \E^{\I \theta_n} - (1 - r_n) z} - 1$.
Then we have 
$$
H(z) = \prod_{n=0}^\infty \big(1 + b_n(z)\big).
$$
Now that our infinite product is in this form, we can utilize Theorem 3.6.1 from A \& F in order to show $H(z)$ is an analytic function.
This can sometimes be referred to as the Weierstrass $M$ test for infinite products (restating for my comprehension and reference but slightly applied to our context). \\

\noindent
Let $b_n(z)$ be analytic in a domain $D$ for all $n$.
Suppose for all $z \in D$ and $n \geq N$ either
\begin{enumerate}
\item $\big|\log \big(1 + b_n(z)\big)\big| \leq M_n$, or
\item $\big|b_n(z)\big| \leq M_n$
\end{enumerate}
where $\sum_{n = 1}^\infty M_n < \infty$ and $M_n$ are constants.
Then the product
$$P (z) = \prod_{n=1}^\infty \big(1 + b_n(z)\big)$$
is uniformly convergent to an analytic function $P(z)$ in $D$.
Furthermore $P(z)$ is zero only when a finite number of it's factors $1 + b_n(z)$ are zero in $D$. \\

\indent
First we need to determine each $b_n(z)$ in the sequence is analytic.
Consider that 
$$
b_n(z)
	= \frac{ (1 - r_n)\E^{\I \theta_n} - z}{ \E^{\I \theta_n} - (1 - r_n) z} - 1
	= \frac 1 {\E^{\I \theta_n}} \frac{ (1 - r_n) \E^{\I \theta_n} - z} {1 - (1 - r_n)\E^{-\I \theta_n} z} - 1.
$$
Since the numerator and denominator of the rational portion of this expression are analytic, it is up to whether or not the denominator will ever be 0.
This would only occur if $(1 - r_n)\E^{-\I \theta_n} z = 1$, however, since $|(1 - r_n)\E^{-\I \theta_n}| < 1$ and $|z| < 1$ this will never occur.
Hence, $b_n(z)$ is analytic. \\

\noindent
Next, we will use the second option of the necessary assumption in Theorem 3.6.1, which is bounding the $|b_n(z)|$.
\begin{align*}
\left|\frac{ (1 - r_n)\E^{\I \theta_n} - z}{ \E^{\I \theta_n} - (1 - r_n) z} - 1 \right| 
	&= \left|\frac{ (1 - r_n)\E^{\I \theta_n} - z}{ \E^{\I \theta_n} - (1 - r_n) z} - \frac{\E^{\I \theta_n} - (1 - r_n) z}{\E^{\I \theta_n} - (1 - r_n) z}\right| \\
	&= \left|\frac{ (1 - r_n)\E^{\I \theta_n} - z - (\E^{\I \theta_n} - (1 - r_n) z)}{ \E^{\I \theta_n} - (1 - r_n) z} \right| \\
	&= \left|\frac{ \cancel{\E^{\I \theta_n}} - r_n\E^{\I \theta_n} - \cancel{z} - \cancel{\E^{\I \theta_n}} + \cancel{z} + r_n z}{ \E^{\I \theta_n} - (1 - r_n) z} \right| \\
	&= \left|\frac{ - r_n\E^{\I \theta_n} + r_n z}{ \E^{\I \theta_n} - (1 - r_n) z} \right|.
\end{align*}
Applying the triangle inequality in the numerator followed by the inverse triangle inequality in the denominator gives us,
\begin{align*}
\left|\frac{ - r_n\E^{\I \theta_n} + r_n z}{ \E^{\I \theta_n} - (1 - r_n) z} \right|
	&\leq \frac{ |-r_n\E^{\I \theta_n}| + |r_n z|}{ \left|\E^{\I \theta_n} - (1 - r_n) z \right|} \\
	&\leq \frac{ r_n + r_n |z|}{ \left|\E^{\I \theta_n} - (1 - r_n) z \right|} \\
	&\leq \frac{ r_n + r_n |z|}{ \big| |\E^{\I \theta_n}| - |(1 - r_n) z| \big|}.
\end{align*}
Next we can utilize that fact that
$$1 - |1 -  r_n| |z| = 1 - (1 - r_n) |z| = 1 - |z| + r_n|z| \geq 1 - |z|$$
to deduce
\begin{align*}
r_n \frac{1 + |z|}{ \big| 1 - |1 - r_n| |z| \big|}
	\leq r_n \frac{1 + |z|}{ \big| 1 - |z| \big|} 
	\leq r_n \frac{1 + |z|}{1 - |z|} 
	\leq r_n \frac{1 + R}{1 - R}.
\end{align*}
Therefore, $M_n = r_n \frac{1 + R}{1 - R}$ and
$$
\sum_{n = N + 1}^ \infty r_n \frac{1 + R}{1 - R} = \frac{1 + R}{1 - R} \sum_{n = N + 1}^ \infty r_n < \infty,
$$
since $r_n \rightarrow 0$ as $n\rightarrow\infty$.
Thus, by Theorem 3.6.1 from A \& F, we can conclude that the product
$$ \prod_{n=N + 1}^\infty \big(1 + b_n(z)\big)$$
is uniformly convergent to an analytic function on the interior of the unit disc.
Thus the whole product 
$$H(z) = \Bigg(\prod_{n=0}^N B(a_n,z) \Bigg) \Bigg(\prod_{n = N + 1}^\infty B(a_n,z) \Bigg) = \prod_{n=0}^\infty B(a_n,z)$$
uniformly converges to an analytic function in the same domain. \\
\qed \\

Consider a sequence of numbers $(a_n)_{n \geq 0}$ such that $|a_n| < 1$ and
$$ \sum_{n = 0}^\infty (1 - |a_n|) < \infty. $$
Define a Blaschke factor
\begin{align*}
B(a,z) =
	\begin{cases}
		\frac{|a|}{a} \frac{ a - z}{ 1 - \bar a z} & a \neq 0,\\
  		z & a  = 0.
  	\end{cases}
\end{align*}
  
\item Show that $H(z)$ has zeros at $z = a_n$ for every $n$.  It
  might seem that this construction of an analytic function with an
  infinite number of zeros in a bounded region implies that $H(z) =
  0$ for all $z$.  Why is this not the case? \\

\noindent
\textit{Solution:} \\
By definition of the Blaschke factor each term is zero if $z = a_n$.
Notice,
$$
B(a_n, a_n)
	= \frac{|a_n|}{a_n} \frac{ a_n - a_n}{ 1 - \bar a_n z}
	= \frac{|a_n|}{a_n} \frac{ 0}{ 1 - \bar a_n z} = 0.
$$
Though this implies the function $H(z)$ has an infinite number of zeros in a bounded region, it does not however,  mean $H(z) = 0$ for all $z$.
This can be concluded from the following.
The sequence of $|a_n| \rightarrow 1$ as $n \rightarrow \infty$ implies that for every $\epsilon > 0$ there exists some $N$ such that when $n > N$ we have $\big| |a_n| - 1\big| < \epsilon$.
Therefore, only a finite number of $a_n$'s (zeros of our function $H(z)$) are within the circle centered at 0 of radius $1 - \epsilon$.
For any $\epsilon^\prime < \epsilon$ you choose, we can choose another $N^\prime$ such that the statement from convergence of a sequence holds once again.
And there would only be a finite number of $a_n$'s within the circle with a new radius of $1-\epsilon^\prime$.
This means there is really only a finite number of zeros and thus $H(z)$ is not 0 everywhere on the interior of the unit disc. \\
\qed
\end{itemize}
\newpage

\item We define the Weierstrass $\wp$-function as
$$
\wp(z)=\frac{1}{z^2}+\sum_{j, k=-\infty}^{\infty}\left(\frac{1}{\left(z-j \omega_1-k \omega_2\right)^2}-\frac{1}{\left(j \omega_1+k \omega_2\right)^2}\right),
$$
where $(j, k)=(0,0)$ is excluded from the double sum. Also, you may
assume that $\omega_1$ is a positive real number, and that $\omega_2$
is on the positive imaginary axis. All considerations below are meant
for the entire complex plane, except the poles of $\wp(z)$.
\begin{enumerate}
\item Show that $\wp\left(z+M \omega_1+N \omega_2\right)=\wp(z)$, for any two integers $M, N$. In other words, $\wp(z)$ is a doubly-periodic function: it has two independent periods in the complex plane. Doubly periodic functions are called elliptic functions. \\

\noindent
\textit{Solution:} \\
Let's begin with $\wp(z + N \omega_1)$, where $N \in \mathbb Z$
\begin{align*}
	&= \frac{1}{(z + N \omega_1)^2}
		+ \sum_{\substack{j, k=-\infty \\ j,k \neq (0,0)}}^{\infty} \left(
			\frac{1}{\left(z + N \omega_1 - j \omega_1 - k \omega_2 \right)^2}
			- \frac{1}{\left(j \omega_1+k \omega_2\right)^2}
		\right) \\
	&= \frac{1}{(z + N\omega_1)^2}
		+ \sum_{\substack{j, k=-\infty \\ j,k \neq (0,0)}}^{\infty} \left(
			\frac{1}{\left(z - (j - N) \omega_1 - k \omega_2 \right)^2}
			- \frac{1}{\left(j \omega_1+k \omega_2\right)^2}
		\right) \\
	&= \frac{1}{(z + N\omega_1)^2}
		+ \frac{1}{\left(z - (N - N) \omega_1 - 0 \omega_2 \right)^2}
			- \frac{1}{\left(N \omega_1+0 \omega_2\right)^2}
		+ \sum_{\substack{j, k=-\infty \\ j,k \neq (0,0) \\ j,k \neq (N,0)}}^{\infty} \left(
			\frac{1}{\left(z - (j - N) \omega_1 - k \omega_2 \right)^2}
			- \frac{1}{\left(j \omega_1+k \omega_2\right)^2}
		\right) \\
	&= \frac{1}{(z + N\omega_1)^2}
		+ \frac 1 {z^2}
			- \frac 1 {N^2\omega_1^2}
		+ \sum_{\substack{j, k=-\infty \\ j,k \neq (0,0) \\ j,k \neq (N,0)}}^{\infty} \left(
			\frac{1}{\left(z - (j - N) \omega_1 - k \omega_2 \right)^2}
			- \frac{1}{\left(j \omega_1+k \omega_2\right)^2}
		\right) \\
	&= \frac 1 {z^2} 
		\underbrace{+ \frac{1}{(z + N\omega_1)^2} - \frac 1 {N^2\omega_1^2}}_\text{j,k = (N, 0)}
		+ \sum_{\substack{j, k=-\infty \\ j,k \neq (0,0) \\ j,k \neq (N,0)}}^{\infty} \left(
			\frac{1}{\left(z - (j - N) \omega_1 - k \omega_2 \right)^2}
			- \frac{1}{\left(j \omega_1+k \omega_2\right)^2}
		\right).
\end{align*}
Notice the term in front of the summation is actually the $j=N,k=0$ term in the sum.
Moving it inside we get
\begin{align*}
&= \frac 1 {z^2} 
	+ \frac{1}{(z + N\omega_1)^2} - \frac 1 {N^2\omega_1^2}
	+ \sum_{\substack{j, k=-\infty \\ j,k \neq (0,0) \\ j,k \neq (N,0)}}^{\infty} \left(
		\frac{1}{\left(z - (j - N) \omega_1 - k \omega_2 \right)^2}
		- \frac{1}{\left(j \omega_1+k \omega_2\right)^2}
	\right) \\
&= \frac 1 {z^2} 
	+ \sum_{\substack{j, k=-\infty \\ j,k \neq (0,0)}}^{\infty} \left(
		\frac{1}{\left(z - j\omega_1 - k\omega_2 \right)^2}
		- \frac{1}{\left(j \omega_1+k \omega_2\right)^2}
	\right) \\
&= \wp(z).
\end{align*}
Therefore, $\wp(z + N \omega_1) = \wp(z)$ for all $z \in \mathbb C$.
Notice, at no point did my argument depend on $\omega_1$ being a positive real number.
Hence, if we did the same algebra with $z + M \omega_2$ we would be able to shuffle things around using the $k$ index instead of the $j$ index to conclude $\wp(z + M \omega_2) = \wp(z)$ for all $z \in \mathbb C$.
Let $z^\prime = z + N \omega_1$.
Then we have $$\wp(z^\prime + M \omega_2) = \wp(z^\prime) = \wp(z + N\omega_1) = \wp(z).$$
Thus
$$\wp(z + N \omega_1 + M \omega_2) = \wp(z).$$
\qed \\

\item Establish that $\wp(z)$ is an even function: $\wp(-z)=\wp(z)$. \\

\noindent
\textit{Solution:} \\
Before I plugin the $z = -z$ into the Weierstrass $\wp$-function, let's pay close attention to one detail.
Notice that the $j,k$ indices index over all combinations integers except the term $j,k = (0,1)$.
This means, as written, the first fraction in the summand the negative signs in front of $j$ and $k$ flip the sign of that particular integer that those indices are taking on for that term in the sum.
$$ \frac{1}{\left(z-j \omega_1-k \omega_2\right)^2} $$
That means if we changed the sign in front of the indices $j$, $k$ we would just be ``reordering" the sum but it will still sum over the same terms.
Therefore, 
\begin{align*}
\wp(-z)
	&= \frac{1}{(-z)^2} + \sum_{\substack{j, k=-\infty \\ j,k \neq (0,0)}}^{\infty} \left(\frac{1}{\left(-z-j \omega_1-k \omega_2\right)^2}-\frac{1}{\left(j \omega_1+k \omega_2\right)^2}\right) \\
	&= \frac{1}{z^2} + \sum_{\substack{j, k=-\infty \\ j,k \neq (0,0)}}^{\infty} \left(\frac{1}{\left(-(z + j \omega_1 + k \omega_2)\right)^2}-\frac{1}{\left(j \omega_1+k \omega_2\right)^2}\right) \\
	&= \frac{1}{z^2} + \sum_{\substack{j, k=-\infty \\ j,k \neq (0,0)}}^{\infty} \left(\frac{1}{\left(z + j \omega_1 + k \omega_2\right)^2}-\frac{1}{\left(j \omega_1+k \omega_2\right)^2}\right).
\end{align*}
We can change the sign in front of the indices in the summand and reverse the order of summation to preserve both equality and the order of summation.
However, since the order in which something is summed does not change it's value we can revert the summation order in our final step.
\begin{align*}
	&= \frac{1}{z^2} + \sum_{\substack{j, k=\infty \\ j,k \neq (0,0)}}^{-\infty} \left(\frac{1}{\left(z - j \omega_1 - k \omega_2\right)^2}-\frac{1}{\left(j \omega_1+k \omega_2\right)^2}\right) \\
	&= \frac{1}{z^2} + \sum_{\substack{j, k=-\infty \\ j,k \neq (0,0)}}^{\infty} \left(\frac{1}{\left(z - j \omega_1 - k \omega_2\right)^2}-\frac{1}{\left(j \omega_1+k \omega_2\right)^2}\right) \\
	&= \wp(z)
\end{align*}
Hence, $\wp(-z) = \wp(z)$. \\
\qed \\

\item Find Laurent expansions for $\wp(z)$ and $\wp^{\prime}(z)$ in a neighborhood of the origin in the form
$$ \wp(z)=\frac{1}{z^2}+\alpha_0+\alpha_2 z^2+\alpha_4 z^4+\ldots $$
and
$$ \wp^{\prime}(z)=-\frac{2}{z^3}+\beta_1 z+\beta_3 z^3+\ldots $$
Give expressions for the coefficients introduced above. \\

\noindent
\textit{Solution:} \\
Let's rewrite the terms inside the sum very carefully
\begin{align*}
& \frac 1 {\left(z-j \omega_1-k \omega_2\right)^2}-\frac{1}{\left(j \omega_1+k \omega_2\right)^2} \\
	&\quad= \frac{1}{\left(j \omega_1+k \omega_2\right)^2}
		\left(
			\frac 1 { \frac{\left(z - (j \omega_1 + k \omega_2) \right)^2}{\left(j \omega_1+k \omega_2\right)^2}} - 1
		\right) \\
	&\quad= \frac{1}{\left(j \omega_1+k \omega_2\right)^2}
		\left(
			\frac 1 { \left(\frac z {j \omega_1 + k \omega_2} 
			- \frac {j \omega_1 + k \omega_2}{j \omega_1 + k \omega_2} \right)^2} - 1
		\right) \\
	&\quad= \frac{1}{\left(j \omega_1+k \omega_2\right)^2}
		\left( \frac 1 { \left(\frac z {j \omega_1 + k \omega_2} - 1 \right)^2} - 1\right) \\
	&\quad= \frac{1}{\left(j \omega_1+k \omega_2\right)^2}
		\left( \frac 1 { \left(1 -  \frac z {j \omega_1 + k \omega_2} \right)^2} - 1 \right) \\
	&\quad= \frac{1}{\left(j \omega_1+k \omega_2\right)^2}
		\left( \left(  \frac 1 { 1 -  \frac z {j \omega_1 + k \omega_2} }\right)^2 - 1 \right) .
\end{align*}
Now looking specifically at the expression being squared.
Assuming
$$|z| < |j \omega_1 + k \omega_2|$$ 
which holds when $|z|$ is near the origin.
Thus, we have a geometric series
$$
\frac 1 { 1 -  \frac z {j \omega_1 + k \omega_2} } = \sum_{m=0}^{\infty} \left( \frac z {j \omega_1 + k \omega_2} \right)^m.
$$
However, notice we can arrive at a clever representation of the squared term inside the first set of parentheses 
\begin{align*}
\frac d {dz} \left[ \frac 1 { 1 -  \frac z {j \omega_1 + k \omega_2} } \right]
	&= \frac d {dz} \left[ \sum_{m=0}^{\infty} \left( \frac z {j \omega_1 + k \omega_2} \right)^m \right] \\
- \frac 1 {(j \omega_1 + k \omega_2)} \frac {-1} {\left(1 -  \frac z {(j \omega_1 + k \omega_2)} \right)^2}
	&= \sum_{m=0}^{\infty} \frac d {dz} \left[ \left( \frac z {j \omega_1 + k \omega_2} \right)^m \right] \\
\frac 1 {(j \omega_1 + k \omega_2)} \frac {1} {\left(1 -  \frac z {(j \omega_1 + k \omega_2)} \right)^2}
	&= \frac 1 {j \omega_1 + k \omega_2} \sum_{m=0}^{\infty} m \left( \frac z {j \omega_1 + k \omega_2} \right)^{m - 1} \\
\frac {1} {\left(1 -  \frac z {(j \omega_1 + k \omega_2)} \right)^2}
	&= \sum_{m=1}^{\infty} m \left( \frac z {j \omega_1 + k \omega_2} \right)^{m - 1} \\
\left( \frac {1} {1 -  \frac z {(j \omega_1 + k \omega_2)} }\right)^2
	&= \sum_{m=0}^{\infty} (m + 1) \left( \frac z {j \omega_1 + k \omega_2} \right)^{m}.
\end{align*}
Combining this with our representation of the summand we have
\begin{align*}
&\frac{1}{\left(j \omega_1+k \omega_2\right)^2}
		\left( \left(  \frac 1 { 1 -  \frac z {j \omega_1 + k \omega_2} }\right)^2 - 1 \right) \\
	&= \frac{1}{\left(j \omega_1+k \omega_2\right)^2}
		\left( \sum_{m=0}^{\infty} (m + 1) \left( \frac z {j \omega_1 + k \omega_2} \right)^{m} - 1 \right) \\
	&= \frac{1}{\left(j \omega_1+k \omega_2\right)^2}
		\left( 1 + \sum_{m=1}^{\infty} (m + 1) \left( \frac z {j \omega_1 + k \omega_2} \right)^{m} - 1 \right) \\
	&= \frac{1}{\left(j \omega_1+k \omega_2\right)^2}
		\sum_{m=0}^{\infty} (m + 2) \frac {z^{m + 1}} {\left( j \omega_1 + k \omega_2\right)^{m + 1}} \\
	&= \sum_{m=0}^{\infty} (m + 2) \frac {z^{m + 1}} {\left( j \omega_1 + k \omega_2\right)^{m + 3}}
\end{align*}
Thus we have,
$$\wp(z) = \frac 1 {z^2} + \sum_{\substack{j, k=-\infty \\ j,k \neq (0,0)}}^{\infty} \sum_{m=0}^{\infty} (m + 2) \frac {z^{m + 1}} {\left( j \omega_1 + k \omega_2\right)^{m + 3}}.$$
Since, $\wp$ is an even function we can let $m = 2\ell - 1$, then
\begin{align*}
\wp(z) &= \frac 1 {z^2} + \sum_{\substack{j, k=-\infty \\ j,k \neq (0,0)}}^{\infty} \sum_{\ell=0}^{\infty} (2\ell - 1 + 2) \frac {z^{2\ell - 1 + 1}} {\left( j \omega_1 + k \omega_2\right)^{2\ell - 1 + 3}} \\
	&= \frac 1 {z^2} + \sum_{\substack{j, k=-\infty \\ j,k \neq (0,0)}}^{\infty} \sum_{\ell=0}^{\infty} (2\ell + 1) \frac {z^{2\ell}} {\left( j \omega_1 + k \omega_2\right)^{2\ell + 2}}
\end{align*}
We can reorder these sums since we have a form of our original geometric series and there are no issues with convergence at the moment
\begin{align*}
\wp(z) &= \frac 1 {z^2} + \sum_{\ell=0}^{\infty} \sum_{\substack{j, k=-\infty \\ j,k \neq (0,0)}}^{\infty} (2\ell + 1) \frac {z^{2\ell}} {\left( j \omega_1 + k \omega_2\right)^{2\ell + 2}} \\
	&= \frac 1 {z^2} + \sum_{\ell=0}^{\infty} \sum_{\substack{j, k=-\infty \\ j,k \neq (0,0)}}^{\infty} (2\ell + 1) \frac {z^{2\ell}} {\left( j \omega_1 + k \omega_2\right)^{2\ell + 2}}.
\end{align*}
Thus our $\alpha_\ell$'s are given by
$$
\alpha_\ell = \sum_{\substack{j, k=-\infty \\ j,k \neq (0,0)}}^{\infty} (2\ell + 1) \frac 1 {\left( j \omega_1 + k \omega_2\right)^{2\ell + 2}} \quad \text{ for $\ell$ = 0, 1, 2, ...}.
$$

\noindent
Furthermore, to calculate the beta's for $\wp^\prime(z)$ we have
\begin{align*}
\wp^\prime(z)
	&= \frac d {dz} \left[ \frac 1 {z^2} + \sum_{\ell=0}^{\infty} \sum_{\substack{j, k=-\infty \\ j,k \neq (0,0)}}^{\infty} (2\ell + 1) \frac {z^{2\ell}} {\left( j \omega_1 + k \omega_2\right)^{2\ell + 2}} \right] \\
	&= - \frac 2 {z^3} + \frac d {dz} \left[ \sum_{\ell=0}^{\infty} \sum_{\substack{j, k=-\infty \\ j,k \neq (0,0)}}^{\infty} (2\ell + 1) \frac {z^{2\ell}} {\left( j \omega_1 + k \omega_2\right)^{2\ell + 2}} \right].
\end{align*}
Because our series is convergent we can differentiate term by term to get
\begin{align*}
\wp^\prime(z)
	&= - \frac 2 {z^3} + \sum_{\ell=0}^{\infty} \sum_{\substack{j, k=-\infty \\ j,k \neq (0,0)}}^{\infty} \frac d {dz} \left[  (2\ell + 1) \frac {z^{2\ell}} {\left( j \omega_1 + k \omega_2\right)^{2\ell + 2}} \right] \\
	&= - \frac 2 {z^3} + \sum_{\ell=0}^{\infty} \sum_{\substack{j, k=-\infty \\ j,k \neq (0,0)}}^{\infty} (2\ell + 1)(2\ell) \frac {z^{2\ell - 1}} {\left( j \omega_1 + k \omega_2\right)^{2\ell + 2}}
\end{align*}
which implies
$$
\beta_\ell = \sum_{\substack{j, k=-\infty \\ j,k \neq (0,0)}}^{\infty} (2\ell + 1)(2\ell) \frac 1 {\left( j \omega_1 + k \omega_2\right)^{2\ell + 2}}.
$$
\qed \\
\newpage

\item Show that $\wp(z)$ satisfies the differential equation
$$ \left(\wp^{\prime}\right)^2=a \wp^3+b \wp^2+c \wp+d, $$

for suitable choices of $a, b, c, d$. Find these constants. You may need to invoke Liouville's theorem to obtain this final result. It turns out that the function $\wp(z)$ is determined by the coefficients $c$ and $d$, implying that it is possible to recover $\omega_1$ and $\omega_2$ from the knowledge of $c$ and $d$. \\

\noindent
\textit{Solution:} \\
Begin by explicitly writing out the first few terms of the Laurent expansions for both $\wp(z)$ and $\wp^\prime(z)$.
For $\wp(z)$ we have
\begin{align*}
\wp(z) &= \frac{1}{z^2}
	+ \sum_{\substack{j, k=-\infty \\ j,k \neq (0,0)}}^{\infty} \frac 1 {\left( j \omega_1 + k \omega_2\right)^2}
	+ \left( \sum_{\substack{j, k=-\infty \\ j,k \neq (0,0)}}^{\infty} \frac 5 {\left( j \omega_1 + k \omega_2\right)^6} \right) z^2
	+ \left( \sum_{\substack{j, k=-\infty \\ j,k \neq (0,0)}}^{\infty} \frac 9 {\left( j \omega_1 + k \omega_2\right)^{10}} \right) z^4
	+ \mathcal O (z^6) \\
\end{align*}
We also have
\begin{align*}
\wp^\prime(z) &= -\frac{2}{z^3}
	+ \left( \sum_{\substack{j, k=-\infty \\ j,k \neq (0,0)}}^{\infty} \frac 6 {\left( j \omega_1 + k \omega_2\right)^4} \right) z
	+ \left( \sum_{\substack{j, k=-\infty \\ j,k \neq (0,0)}}^{\infty} \frac {42} {\left( j \omega_1 + k \omega_2\right)^8} \right) z^3
	+ \left( \sum_{\substack{j, k=-\infty \\ j,k \neq (0,0)}}^{\infty} \frac {110} {\left( j \omega_1 + k \omega_2\right)^{12}} \right) z^5 
	+ \mathcal O (z^7) \\
\end{align*}
\textbf{TODO: Only come back to this tedious algebra if you have time.}
\\

\end{enumerate}
\end{enumerate}
\end{document}

%%% Local Variables:
%%% mode: latex
%%% TeX-master: t
%%% End:
