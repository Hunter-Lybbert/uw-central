\documentclass[10pt]{amsart}
\usepackage[margin=1.4in]{geometry}
\usepackage[usenames,dvipsnames,cmyk]{xcolor} %load first
\usepackage{cancel}
\usepackage{graphicx,subfig}
\usepackage{mathtools}

\graphicspath{ {./images/} }

\usepackage{amssymb,amsmath,enumitem,url}

\newcommand{\D}{\mathrm{d}}
\newcommand{\I}{\mathrm{i}}
\DeclareMathOperator{\E}{e}
\DeclareMathOperator{\OO}{O}
\DeclareMathOperator{\oo}{o}
\DeclareMathOperator{\erfc}{erfc}
\DeclareMathOperator{\real}{Re}
\DeclareMathOperator{\imag}{Im}
\usepackage{tikz}
\usepackage[framemethod=tikz]{mdframed}
\theoremstyle{nonumberplain}

\mdtheorem[innertopmargin=5pt]{lemma}{Lemma}
\mdtheorem[innertopmargin=-5pt]{sol}{Solution}
%\newmdtheoremenv[innertopmargin=-5pt]{sol}{Solution}
\definecolor{MichiganBlue}{HTML}{00274C}
\definecolor{MichiganYellow}{HTML}{FFCB05}  
\definecolor{NicePurple}{RGB}{75,56,76} %PrincePurple
\definecolor{NiceRed}{RGB}{230,37,52}
\definecolor{MidnightBlue}{rgb}{0.1, 0.1, 0.44}
\usepackage[colorlinks=true, linkcolor=MidnightBlue, citecolor=MidnightBlue, urlcolor=MidnightBlue]{hyperref}

\begin{document}
\pagestyle{empty}

\newcommand{\mline}{\vspace{.2in}\hrule\vspace{.2in}}


\noindent
\text{Hunter Lybbert} \\
\text{Student ID: 2426454} \\
\text{11-11-24} \\
\text{AMATH 567} \\

\title{\bf { Homework 7} }


\maketitle
\noindent
Collaborators*: Nate Ward, Sophia Kamien, Laura Thomas, Cooper Simpson,  \\
\\
\tiny
\text{*Listed in no particular order. And anyone I discussed at least part of one problem with is considered a collaborator.}
\normalsize


\mline
\begin{enumerate}[label={\bf {\arabic*}:}]
\item  From A\&F: 3.5.1 b, c, d (Only consider singularities in the finite
  complex plane) \\
Discuss the type of singularity (removable, pole and order, essential, branch cluster, natural barrier, etc.); if the type is a pole give the strength of the pole,  and give the nature (isolated or not) of all singular points associated with the following functions. \\

\noindent
For my reference, here are the definitions of types of singularities.
A point $z=z_0$ is a {\bf singularity} of $f(z)$ if $f^\prime(z_0)$ does not exist.
Suppose $f(z)$ is analytic in a region $0 < |z - z_0| < R$ (a neighborhood of $z = z_0$) but not at the point $z_0$, then $z_0$ is called an {\bf isolated singular point} of $f(z)$.
An isolated singularity $z=z_0$ is {\bf removable}, if $f(z)$ is bounded in some neighborhood of $z=z_0$ s.t. $|f(z)| < M$.
Additionally, an isolated singularity at $z = z_0$ of $f(z)$ is called a {\bf pole} if $f(z)$ has the following representation
$$
f(z) = \frac {\phi (z)}{(z - z_0)^N}
$$
where $N$ is a positive integer, $N \geq 1$, $\phi(z)$ is analytic in a neighborhood of $z_0$, and $\phi(z_0) \neq 0$.
If $N \geq 2$ we say the pole is an {\bf ${\bf N}^{\rm th}$ order pole} and if $N = 1$ it is a {\bf simple pole}.
Therefore, the Laurent expansion of $f(z)$ takes the form
$$
f(z) = \sum_{n=-N}^\infty C_n(z - z_0)^n.
$$
The first coefficient $C_{-N}$ is the {\bf strength of the pole}.

\noindent
(b)
$$
f(z) = \frac {\E^{2z} - 1} {z^2}
$$
\textit{Solution:} \\
\textbf{TODO} \\

\noindent
(c)
$$
f(z) = \E^{\tan z}
$$
\textit{Solution:} \\
Let's first look for where the denominator is undefined.
Notice,
\begin{align*}
f(z) &= \E^{\tan z} \\
	&= \sum_{n=0}^\infty \frac {\tan^n z} {n!} \\
	&= \sum_{n=0}^\infty \frac 1 {n!} \frac {\sin^n z} {\cos^n z}.
\end{align*}
Therefore, $f(z)$ has singularities at $z = \pi/2 + \pi k$ for $k \in \mathbb Z$.
These singularities are essential since the limit of $f(z)$ at teach of these points does not exist.
Let $w$ be an arbitrary singularity of $f(z)$ of the form $w = \pi/2 + \pi k$
\begin{align*}
\lim_{z\rightarrow w} \E^{\tan z}
	&= \lim_{z\rightarrow w} \sum_{n=0}^\infty \frac 1 {n!} \frac {\sin^n z} {\cos^n z} \\
	&= \sum_{n=0}^\infty \lim_{z\rightarrow w} \frac 1 {n!} \frac {\sin^n z} {\cos^n z}
\end{align*}
Looking at just one term we have,
\begin{align*}
\lim_{z\rightarrow w} \frac 1 {n!} \frac {\sin^n z} {\cos^n z}
	&= \frac 1 {n!} \frac {\sin^n (\pi/2 + \pi k)} {\cos^n (\pi/2 + \pi k)} \\
	&= \frac 1 {n!} \frac {(-1)^{kn}} {0}.
\end{align*}
Using L'Hôpital's rule we have
\begin{align*}
\lim_{z\rightarrow w} \frac 1 {n!} \frac {\sin^n z} {\cos^n z}
	&= \lim_{z\rightarrow w} \frac 1 {n!} \frac {n\cos^{n - 1} z} {-n\sin^{n - 1} z} \\
	&= \lim_{z\rightarrow w} -\frac 1 {n!} \frac {\cos^{n - 1} z} {\sin^{n - 1} z} \\
	&= -\frac 1 {n!} \frac {0} {(-1)^{k(n - 1)}} \\
	&= 0
\end{align*}
Attempting to apply L'Hôpital's rule will not help us since $\sin$ and $\cos$ are infinitely differentiable and won't vanish at any point.
\textbf{TODO} \\

\noindent
(d)
$$
f(z) = \frac {z^3}{z^2 + z + 1}
$$
\textit{Solution:} \\
\textbf{TODO} \\
\newpage

\item From A\&F: 3.5.3 a, c, d \\
Show that the functions below are meromorphic; that is, the only singularities in the finite $z$ plane are poles.
Determine the location, order and strength of the poles. 

\noindent
(a)
$$
f(z) = \frac z {z^4 + 2}
$$
\textit{Solution:} \\
\textbf{TODO} \\
Find the Laurent series in a clever way to find the strength (partial fractions).

\noindent
(c)
$$
f(z) = \frac z {\sin^2 z}
$$
\textit{Solution:} \\
\textbf{TODO} \\

\noindent
(d)
$$
f(z) = \frac {\E^z - 1 - z}{z^4}
$$
\textit{Solution:} \\
\textbf{TODO} \\
\newpage

\item Introducing the Gamma function: Do A\&F: 3.6.6. \\

\noindent
Let $\Gamma(z)$ be given by
$$
\frac 1 {\Gamma(z)} = z \E^{\gamma z} \prod_{n=1}^\infty \left( 1 + \frac z n \right) \E^{-z / n}
$$
for $z \neq 0, -1, -2, ...$ and $\gamma = $ constant (probably real). \\

\noindent
(a) Show that 
$$
\frac {\Gamma^\prime(z)}{\Gamma(z)} = - \frac 1 z - \gamma - \sum_{n = 1}^\infty \left( \frac 1 {z + n} - \frac 1 n\right).
$$
\textit{Solution:} \\
Let's begin by looking at the $\log$ of each side (let the right hand side be $g(z)$)
\begin{align*}
\log \left( \frac 1 {\Gamma(z)} \right) &= \log g(z) \\
\log 1 - \log \Gamma(z)  &= \log g(z) \\
\frac d {d z} - \log\Gamma(z) &= \frac d {dz} \log g(z) \\
\frac d {d z} \log \Gamma(z) &=  - \frac d {dz} \log g(z) \\
\frac {\Gamma^\prime(z)}{\Gamma(z)} &=  - \frac d {dz} \log g(z).
\end{align*}
Now, simplify $\log g(z)$, plugging back in the expression $g(z)$ represents
\begin{align*}
\log g(z) &= \log \left( z \E^{\gamma z} \prod_{n=1}^\infty \left( 1 + \frac z n \right) \E^{-z / n} \right) \\
	&= \log z + \gamma z + \sum_{n=1}^\infty \left( \log \left(  \frac {n + z} n \right)  - \frac z  n \right) \\
	&= \log z + \gamma z + \sum_{n=1}^\infty \bigg( \log (n + z) - \log n  - \frac z  n \bigg).
\end{align*}
Finally, take the derivative and negate the resulting expression
\begin{align*}
\frac {\Gamma^\prime(z)}{\Gamma(z)} &= - \frac d {dz} \log g(z) \\
	&= - \frac d {dz}\left( \log z + \gamma z + \sum_{n=1}^\infty \bigg( \log (n + z) - \log n  - \frac z  n \bigg) \right) \\
	&= - \frac 1 z - \gamma - \frac d {dz}\sum_{n=1}^\infty \bigg( \log (n + z) - \log n  - \frac z  n \bigg) \\
	&= - \frac 1 z - \gamma - \sum_{n=1}^\infty \bigg( \frac 1 {z + n} - \frac 1 n \bigg).
\end{align*}
\qed \\

\noindent
(b) Show that
\begin{align}
\frac {\Gamma^\prime(z + 1)} {\Gamma(z + 1)} - \frac {\Gamma^\prime(z)} {\Gamma(z)} - \frac 1 z = 0.
\label{eq:gamma_thing}
\end{align}
\textit{Solution:} \\
We can plug in $z + 1$ to the formula we just computed and simplify
\begin{align*}
&\frac {\Gamma^\prime(z + 1)} {\Gamma(z + 1)} - \frac {\Gamma^\prime(z)} {\Gamma(z)} - \frac 1 z \\
	& \quad = - \frac 1 {z + 1} - \gamma - \sum_{n=1}^\infty \bigg( \frac 1 {z + 1 + n} - \frac 1 n \bigg) - \left(- \frac 1 z - \gamma - \sum_{n=1}^\infty \bigg( \frac 1 {z + n} - \frac 1 n \bigg) \right) - \frac 1 z \\
	& \quad = - \frac 1 {z + 1} - \cancel{\gamma} - \sum_{n=1}^\infty \bigg( \frac 1 {z + 1 + n} - \frac 1 n \bigg) + \cancel{\frac 1 z} + \cancel{\gamma} + \sum_{n=1}^\infty \bigg( \frac 1 {z + n} - \frac 1 n \bigg) - \cancel{\frac 1 z} \\
	& \quad = - \frac 1 {z + 1} + \sum_{n=1}^\infty \bigg( - \frac 1 {z + 1 + n} + \frac 1 n \bigg) + \sum_{n=1}^\infty \bigg( \frac 1 {z + n} - \frac 1 n \bigg) \\
	& \quad = - \frac 1 {z + 1} + \sum_{n=1}^\infty \bigg( - \frac 1 {z + 1 + n} + \cancel{\frac 1 n} + \frac 1 {z + n} - \cancel{\frac 1 n} \bigg) \\
	& \quad = - \frac 1 {z + 1} + \sum_{n=1}^\infty \bigg(\frac 1 {z + n}  - \frac 1 {z + 1 + n} \bigg) \\
	& \quad = - \frac 1 {z + 1} + \bigg(\frac 1 {z + 1}  - \cancel{\frac 1 {z + 2}} + \cancel{\frac 1 {z + 2}}  - \cancel{\frac 1 {z + 3}} + \cancel{\frac 1 {z + 3}}  - \frac 1 {z + 4} + ... \bigg) \\
	& \quad = - \frac 1 {z + 1} + \frac 1 {z + 1} \\
	& \quad = 0
\end{align*}
\qed \\
Whereupon
$$ \Gamma(z + 1) = Cz\Gamma(z), \quad \text{for a constant $C$}. $$
\textit{Solution:} \\
We can show this by integrating equation (\ref{eq:gamma_thing})
\begin{align*}
\int \frac {\Gamma^\prime(z + 1)} {\Gamma(z + 1)} - \frac {\Gamma^\prime(z)} {\Gamma(z)} - \frac 1 z &= \int 0 \\
\log \Gamma(z + 1) - \log {\Gamma(z)} - \log z - C &= 0 \\
\log \Gamma(z + 1) &= C + \log z + \log {\Gamma(z)} \\
\log \Gamma(z + 1) &= C + \log (z{\Gamma(z)}) \\
\E^{\log \Gamma(z + 1)} &= \E^{C + \log (z{\Gamma(z)})} \\
\Gamma(z + 1) &= \E^C z\Gamma(z) \\
\Gamma(z + 1) &= C^\prime z\Gamma(z)
\end{align*}
where $C^\prime = \E^C$ is a constant. \\
\qed \\
\newpage

\noindent
(c) Show that $\lim_{z\rightarrow 0} z \Gamma(z) = 1$ to find that $C = \Gamma(1)$. \\

\noindent
\textit{Solution:} \\
If I can show the limit of the reciprocal is 1, then the limit of the original function will also be 1.
Now actually taking the limit we have
\begin{align*}
\lim_{z\rightarrow 0} \frac 1 {z \Gamma(z)}
	&= \lim_{z\rightarrow 0} \frac 1 z z \E^{\gamma z} \prod_{n=1}^\infty \left( 1 + \frac z n \right) \E^{-z / n} \\
	&= \lim_{z\rightarrow 0} \E^{\gamma z} \prod_{n=1}^\infty \left( 1 + \frac z n \right) \E^{-z / n} \\
	&= \E^{0} \prod_{n=1}^\infty \left( 1 + \frac 0 n \right) \E^{-0 / n} \\
	&= \prod_{n=1}^\infty \left( 1\right) \\
	&= 1.
\end{align*}
Now if we take the limit of our previous expression we have from part (c)
\begin{align*}
\lim_{z \rightarrow 0} \Gamma(z + 1) &= \lim_{z \rightarrow 0} Cz\Gamma(z) \\
\Gamma(1) &= C
\end{align*}
\qed \\

\noindent
(d) Determine the following representation of the constant $\gamma$ so that $\Gamma(1) = 1$
$$
\E^{-\gamma} = \prod_{n = 1}^\infty \left(1 + \frac 1 n\right)\E^{-1/n}
$$
\textit{Solution:} \\
We want to find $\E^{-\gamma}$ s.t. $\Gamma(1) = 1$.
Therefore we need
\begin{align*}
\frac 1 {\Gamma(z)} &= z \E^{\gamma z} \prod_{n=1}^\infty \left( 1 + \frac z n \right) \E^{-z / n} \\
\frac 1 {\Gamma(1)} &= \E^\gamma \prod_{n=1}^\infty \left( 1 + \frac 1 n \right) \E^{-1 / n} \\
1 &= \E^\gamma \prod_{n=1}^\infty \left( 1 + \frac 1 n \right) \E^{-1 / n} \\
\frac 1 {\E^\gamma} &= \prod_{n=1}^\infty \left( 1 + \frac 1 n \right) \E^{-1 / n} \\
\E^{-\gamma} &= \prod_{n=1}^\infty \left( 1 + \frac 1 n \right) \E^{-1 / n}
\end{align*}
\qed \\
\newpage

\noindent
(e) Show that
\begin{align*}
\prod_{n=1}^\infty \left(1 + \frac 1 n \right)\E^{-1/n}
	&= \lim_{n\rightarrow\infty} \frac 2 1 \frac 3 2 \frac 4 3 ... \frac {n + 1} n \E^{-S(n)} \\
	&= \lim_{n\rightarrow\infty} (n + 1) \E^{-S(n)}
\end{align*}
where $S(n) = 1 + \frac 1 2 + \frac 1 3 + \frac 1 4 + ... + \frac 1 n = \sum_{\ell=1}^n \frac 1 \ell$. \\

\noindent
\textit{Solution:} \\
We begin by truncating the infinite product to $n$ and take a limit as $n \rightarrow \infty$, 
\begin{align*}
\prod_{n=1}^\infty \left(1 + \frac 1 n \right)\E^{-1/n} 
	&= \prod_{n=1}^\infty \left(\frac {n + 1} n \right)\E^{-1/n} \\
	&= \lim_{n\rightarrow\infty}
		\left(\frac {2} 1 \E^{-1} \right)
		\left(\frac {3} 2 \E^{-1/2} \right)
		\left(\frac {4} 3 \E^{-1/3} \right) ...
		\left(\frac {n} {n - 1} \E^{-1/(n - 1)} \right)
		\left(\frac {n + 1} n \E^{-1/n} \right) \\
	&= \lim_{n\rightarrow\infty}
		\left(\frac {2} 1 \frac {3} 2 \frac {4} 3 ... \frac {n} {n - 1}\frac {n + 1} n \right)
		\left( \E^{-1} \E^{-1/2}\E^{-1/3} ... \E^{-1/n} \right) \\
	&= \lim_{n\rightarrow\infty}
		\left(\frac {2} 1 \frac {3} 2 \frac {4} 3 ... \frac {n} {n - 1}\frac {n + 1} n \right)
		\E^{-1 -1/2 -1/3 ... -1/n} \\
	&= \lim_{n\rightarrow\infty}
		\left(\frac {2} 1 \frac {3} 2 \frac {4} 3 ... \frac {n} {n - 1}\frac {n + 1} n \right)
		\E^{-\sum_{\ell=0}^n 1/\ell} \\
	&= \lim_{n\rightarrow\infty}
		\left(\frac {2} 1 \frac {3} 2 \frac {4} 3 ... \frac {n} {n - 1}\frac {n + 1} n \right) \E^{-S(n)}.
\end{align*}
Notice, we can pair up and cancel out each term in the product in the numerator and in the denominator except for a $1$ in the denominator and the $n + 1$ in the numerator
\begin{align*}
\lim_{n\rightarrow\infty}
	\left(\frac {\cancel{2}} 1 \frac {\cancel{3}} {\cancel{2}} \frac {\cancel{4}} {\cancel{3}} ... \frac {\cancel n} {\cancel{n - 1}}
	\frac {n + 1} {\cancel n} \right) \E^{-S(n)}
&= \lim_{n\rightarrow\infty} (n + 1) \E^{-S(n)}.
\end{align*} \qed \\

\noindent
Consequently, obtain the limit
$$
\gamma = \lim_{n\rightarrow\infty} \Bigg(  \sum_{k=1}^n \frac 1 k - \log(n + 1)  \Bigg).
$$
\textit{Solution:} \\
This comes from taking the log of both sides of the following:
\begin{align*}
\E^{-\gamma} &= \lim_{n\rightarrow\infty} (n + 1) \E^{-S(n)} \\
\log \E^{-\gamma} &= \log\big( \lim_{n\rightarrow\infty} (n + 1) \E^{-S(n)} \big) \\
-\gamma &= \lim_{n\rightarrow\infty} \log\big( (n + 1) \E^{-S(n)} \big) \\
\gamma &= - \bigg(\lim_{n\rightarrow\infty} \log (n + 1) + \log \big(\E^{-S(n)}\big) \bigg) \\
\gamma &= - \bigg(\lim_{n\rightarrow\infty} \log (n + 1) - S(n) \bigg) \\
\gamma &= \lim_{n\rightarrow\infty} \Bigg (\sum_{\ell = 0}^n \frac 1 \ell - \log (n + 1) \Bigg ).
\end{align*} \qed \\

\noindent
This is the same Gamma function you may have seen defined as
$$
\Gamma(z)=\int_0^{\infty} t^{z-1} e^{-t} d t
$$
This better known representation is only valid for
$\operatorname{Re}(z)>0$. The representation given here is valid in
all of $\mathbb{C}$. It takes a bit of work to show that our
representation is an analytic continuation of the integral
representation (this requires the Dominated Convergence Theorem), but
it is quite doable. Not now though. \\
\newpage

\item Consider a sequence of numbers $(a_n)_{n \geq 0}$ such that $|a_n| < 1$ and
$$ \sum_{n = 0}^\infty (1 - |a_n|) < \infty. $$
Define a Blaschke factor
\begin{align*}
B(a,z) =
	\begin{cases}
		\frac{|a|}{a} \frac{ a - z}{ 1 - \bar a z} & a \neq 0,\\
  		z & a  =0.
  	\end{cases}
\end{align*}
\begin{itemize}
\item Show that
\begin{align*}
H(z) = \prod_{n=0}^\infty B(a_n,z),
\end{align*}
defines an analytic function in the open unit disk $|z| < 1$. \\
\textit{Solution:} \\
Let's start with the case where $a_n \neq 0$ and use the substitution $a_n = (1 - r_n)\E^{\I \theta_n}$, then
\begin{align*}
H(z) = \prod_{n=0}^\infty B(a_n,z)
	&= \prod_{n=0}^\infty \frac{|a_n|}{a_n} \frac{ a_n - z}{ 1 - \bar a_n z} \\
	&= \prod_{n=0}^\infty \frac{|(1 - r_n)\E^{\I \theta_n}|}{(1 - r_n)\E^{\I \theta_n}} \frac{ (1 - r_n)\E^{\I \theta_n} - z}{ 1 - (1 - r_n)\E^{-\I \theta_n} z} \\
	&= \prod_{n=0}^\infty \frac{\cancel{(1 - r_n)}}{\cancel{(1 - r_n)}\E^{\I \theta_n}} \frac{ (1 - r_n)\E^{\I \theta_n} - z}{ 1 -(1 - r_n)\E^{-\I \theta_n} z} \\
	&= \prod_{n=0}^\infty \frac{ (1 - r_n)\E^{\I \theta_n} - z}{ \E^{\I \theta_n} - (1 - r_n)\E^{\I \theta_n}\E^{-\I \theta_n} z} \\
	&= \prod_{n=0}^\infty \frac{ (1 - r_n)\E^{\I \theta_n} - z}{ \E^{\I \theta_n} - (1 - r_n) z} \\
	&= \prod_{n=0}^\infty 1 + \frac{ (1 - r_n)\E^{\I \theta_n} - z}{ \E^{\I \theta_n} - (1 - r_n) z} - 1.
\end{align*}
Let $b_n(z) = \frac{ (1 - r_n)\E^{\I \theta_n} - z}{ \E^{\I \theta_n} - (1 - r_n) z} - 1$.
Then we have 
$$
H(z) = \prod_{n=0}^\infty \big(1 + b_n(z)\big).
$$
Now that our infinite product is in this form, we can use the Weierstrass $M$ test for infinite products which states (restating for my comprehension and reference but slightly applied to our context). \\

\noindent
Let $b_n(z)$ be analytic in a domain $D$ for all $n$.
Suppose for all $z \in D$ and $n \geq N$ either
\begin{enumerate}
\item $\big|\log \big(1 + b_n(z)\big)\big| \leq M_n$, or
\item $\big|b_n(z)\big| \leq M_n$
\end{enumerate}
where $\sum_{n = 1}^\infty M_n < \infty$ and $M_n$ are constants.
Then the product
$$P (z) = \prod_{n=1}^\infty \big(1 + b_n(z)\big)$$
is uniformly convergent to an analytic function $P(z)$ in $D$.
Furthermore $P(z)$ is zero only when a finite number of it's factors $1 + b_n(z)$ are zero in $D$. \\

\noindent
Let's begin by using the second option, which is bounding the $|b_n(z)|$.
\begin{align*}
\left|\frac{ (1 - r_n)\E^{\I \theta_n} - z}{ \E^{\I \theta_n} - (1 - r_n) z} - 1 \right| 
	&= \left|\frac{ (1 - r_n)\E^{\I \theta_n} - z}{ \E^{\I \theta_n} - (1 - r_n) z} - \frac{\E^{\I \theta_n} - (1 - r_n) z}{\E^{\I \theta_n} - (1 - r_n) z}\right| \\
	&= \left|\frac{ (1 - r_n)\E^{\I \theta_n} - z - (\E^{\I \theta_n} - (1 - r_n) z)}{ \E^{\I \theta_n} - (1 - r_n) z} \right| \\
	&= \left|\frac{ \cancel{\E^{\I \theta_n}} - r_n\E^{\I \theta_n} - \cancel{z} - \cancel{\E^{\I \theta_n}} + \cancel{z} + r_n z}{ \E^{\I \theta_n} - (1 - r_n) z} \right| \\
	&= \left|\frac{ - r_n\E^{\I \theta_n} + r_n z}{ \E^{\I \theta_n} - (1 - r_n) z} \right|.
\end{align*}
Applying the triangle inequality in the numerator followed by the inverse triangle inequality in the denominator gives us,
\begin{align*}
\left|\frac{ - r_n\E^{\I \theta_n} + r_n z}{ \E^{\I \theta_n} - (1 - r_n) z} \right|
	&\leq \frac{ |-r_n\E^{\I \theta_n}| + |r_n z|}{ \left|\E^{\I \theta_n} - (1 - r_n) z \right|} \\
	&\leq \frac{ r_n + r_n |z|}{ \left|\E^{\I \theta_n} - (1 - r_n) z \right|} \\
	&\leq \frac{ r_n + r_n |z|}{ \big| |\E^{\I \theta_n}| - |(1 - r_n) z| \big|} \\
	&= \frac{ r_n + r_n |z|}{ \big| 1 - |1 - r_n| |z| \big|} \\
	&\leq \frac{ r_n}{ \big| 1 - |1 - r_n|\big|}.
\end{align*}
In the final step,  we set $|z|$ to be it's upper bound 1 in the denominator and it's lower bound 0 in the numerator giving us another $\leq$ inequality.
Now we have
$$ M_n = \frac{ r_n}{ \big| 1 - |1 - r_n|\big|}. $$
We want to use this to show the sum $ \sum_{n = 1}^\infty M_n $ is finite.
Using the ratio test we have
\begin{align*}
\lim_{n\rightarrow\infty} \Bigg| \frac {M_{n + 1}}{M_n} \Bigg|
	&= \left| \frac {\frac{ r_{n + 1}}{ \big| 1 - |1 - r_{n + 1}|\big|}}{\frac{ r_n}{ \big| 1 - |1 - r_n|\big|}} \right| \\
	&= \left|  \frac{ r_{n + 1}}{ \big| 1 - |1 - r_{n + 1}|\big|} \frac{ \big| 1 - |1 - r_n|\big|} { r_n}\right| \\
	&= \left|  \frac{ r_{n + 1}}{ r_n} \right| \frac{ \big| 1 - |1 - r_n|\big|}{ \big| 1 - |1 - r_{n + 1}|\big|} \\
	&\leq \left|  \frac{ r_{n + 1}}{ r_n} \right| \frac{|1| + \big|- |1 - r_n|\big|}{ \big| 1 - |1 - r_{n + 1}|\big|} \\
	&\leq \left|  \frac{ r_{n + 1}}{ r_n} \right| \frac{1 + |1 - r_n|}{ \big| 1 - |1 - r_{n + 1}|\big|} \\
	&\leq \left|  \frac{ r_{n + 1}}{ r_n} \right| \frac{1 + |1 - r_n|}{1} \\
	&\leq \left|  \frac{ r_{n + 1}}{ r_n} \right| \frac{1 + |1 - r_n|}{1} ....\text{ big issues }
\end{align*}
Since $|a_n| < 1$ and $a_n = (1 - r_n)\E^{\I \theta_n}$, implies $0 < r_n < 2$.

\noindent
Let's begin by looking at the modulus of $\log$ of an arbitrary factor
\begin{align*}
& \Bigg|\log \bigg( 1 + \frac{|a_n| a_n - |a_n|z - a_n + |a_n|^2 z}{a_n( 1 - \bar a_n z)}\bigg) \Bigg| \\
	&\quad = \Bigg|\log \bigg(\frac{a_n( 1 - \bar a_n z) + |a_n| a_n - |a_n|z - a_n + |a_n|^2 z}{a_n( 1 - \bar a_n z)}\bigg) \Bigg| \\
	&\quad = \Bigg|\log \bigg( \frac{|a_n| (a_n - z)}{a_n( 1 - \bar a_n z)}\bigg) \Bigg| \\
	&\quad = \big|\log ( |a_n| (a_n - z)) - \log(a_n( 1 - \bar a_n z)) \big| \\
	&\quad= \big|\log |a_n| + \log (a_n - z) - \log a_n - \log( 1 - \bar a_n z) \big| \\
	&\quad \leq \big|\log |a_n| - \log a_n \big| + \big| \log (a_n - z) - \log( 1 - \bar a_n z) \big| \\
	&\quad \leq \big| - \log a_n \big| + \big| \log (a_n - z) - \log( 1 - \bar a_n z) \big|.
\end{align*}
Letting $a_n = r_n\E^{\I\theta_n}$ and $z = \rho \E^{\I\phi}$ we have
\begin{align*}
& \big| - \log a_n \big| + \big| \log (a_n - z) - \log( 1 - \bar a_n z) \big| \\
	& \quad = \big| - \log \left(r_n\E^{\I\theta_n}\right) \big| + \big| \log (r_n\E^{\I\theta_n} - \rho \E^{\I\phi}) - \log( 1 - \frac 1 {r_n} \E^{-\I\theta_n} \rho \E^{\I\phi}) \big| \\
	& \quad = \big| - \log r_n - \I\theta_n \big| + \big| \log (r_n\E^{\I\theta_n} - \rho \E^{\I\phi}) - \log( 1 - \frac 1 {r_n} \rho \E^{-\I\theta_n} \E^{\I\phi}) \big| \\
	& \quad = \log r_n + \I\theta_n + \big| \log (r_n\E^{\I\theta_n} - \rho \E^{\I\phi}) - \log( 1 - \frac \rho {r_n} \E^{\I (\phi - \theta_n)}) \big| \\
	& \quad \leq \log r_n + \I\theta_n.
\end{align*}
So we let $M_n = \log r_n + \I\theta_n$.
Now we wish to show that the following sum is finite
$$\sum_{n = 0}^\infty \log r_n + \I\theta_n.$$
Let's look at the ratio test
\begin{align*}
\lim_{n\rightarrow\infty} \Bigg| \frac {\log r_{n + 1} + \I\theta_{n + 1}}{\log r_n + \I\theta_n} \Bigg|.
\end{align*}
\textbf{TODO: I have no idea where I need to go from here.} \\
use the M test for infinite products, uniform convergence of analytic functions is analytic function \\
  
\item Show that $H(z)$ has zeros at $z = a_n$ for every $n$.  It
  might seem that this construction of an analytic function with an
  infinite number of zeros in a bounded region implies that $H(z) =
  0$ for all $z$.  Why is this not the case? \\
\textit{Solution:} \\
\textbf{TODO} \\
Something about the oddity of the zeros on the edge of the unit disc \\
\end{itemize}
\newpage

\item We define the Weierstrass $\wp$-function as
$$
\wp(z)=\frac{1}{z^2}+\sum_{j, k=-\infty}^{\infty}\left(\frac{1}{\left(z-j \omega_1-k \omega_2\right)^2}-\frac{1}{\left(j \omega_1+k \omega_2\right)^2}\right),
$$
where $(j, k)=(0,0)$ is excluded from the double sum. Also, you may
assume that $\omega_1$ is a positive real number, and that $\omega_2$
is on the positive imaginary axis. All considerations below are meant
for the entire complex plane, except the poles of $\wp(z)$.
\begin{enumerate}
\item Show that $\wp\left(z+M \omega_1+N \omega_2\right)=\wp(z)$, for any two integers $M, N$. In other words, $\wp(z)$ is a doubly-periodic function: it has two independent periods in the complex plane. Doubly periodic functions are called elliptic functions. \\
\textit{Solution:} \\
\textbf{TODO} \\

\item Establish that $\wp(z)$ is an even function: $\wp(-z)=\wp(z)$. \\
\textit{Solution:} \\
\textbf{TODO} \\

\item Find Laurent expansions for $\wp(z)$ and $\wp^{\prime}(z)$ in a neighborhood of the origin in the form
$$ \wp(z)=\frac{1}{z^2}+\alpha_0+\alpha_2 z^2+\alpha_4 z^4+\ldots $$
and
$$ \wp^{\prime}(z)=-\frac{2}{z^3}+\beta_1 z+\beta_3 z^3+\ldots $$
Give expressions for the coefficients introduced above. \\
\textit{Solution:} \\
\textbf{TODO} \\

\item Show that $\wp(z)$ satisfies the differential equation
$$ \left(\wp^{\prime}\right)^2=a \wp^3+b \wp^2+c \wp+d, $$

for suitable choices of $a, b, c, d$. Find these constants. You may need to invoke Liouville's theorem to obtain this final result. It turns out that the function $\varphi(z)$ is determined by the coefficients $c$ and $d$, implying that it is possible to recover $\omega_1$ and $\omega_2$ from the knowledge of $c$ and $d$. \\
\textit{Solution:} \\
\textbf{TODO} \\


{\bf OH Notes: }
For (c) take the derivative of the terms in the original sum of the $\wp(z)$ function, apply Taylor's theorem to get something with a $\mathcal O(z^3)$.
Do we need to show any uniform convergence in (c) to go from the representation of $\wp(z)$ to the representation of $\wp^\prime(z)$.
First do everything assuming no issues with convergence or uniformity.
Get what you are looking for, then go back and justify things after.
Consider using Liouville theorem for part (d). \\

\noindent
doubly periodic and bounded is just constant \\

\noindent
Consider doing a Taylor expansion for the function in the summand and showing that the thing

\end{enumerate}
\end{enumerate}
\end{document}

%%% Local Variables:
%%% mode: latex
%%% TeX-master: t
%%% End:
