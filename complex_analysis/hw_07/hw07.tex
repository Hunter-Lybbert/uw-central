\documentclass[10pt]{amsart}
\usepackage[margin=1.4in]{geometry}
\usepackage[usenames,dvipsnames,cmyk]{xcolor} %load first
\usepackage{cancel}
\usepackage{graphicx,subfig}
\usepackage{mathtools}

\graphicspath{ {./images/} }

\usepackage{amssymb,amsmath,enumitem,url}

\newcommand{\D}{\mathrm{d}}
\newcommand{\I}{\mathrm{i}}
\DeclareMathOperator{\E}{e}
\DeclareMathOperator{\OO}{O}
\DeclareMathOperator{\oo}{o}
\DeclareMathOperator{\erfc}{erfc}
\DeclareMathOperator{\real}{Re}
\DeclareMathOperator{\imag}{Im}
\usepackage{tikz}
\usepackage[framemethod=tikz]{mdframed}
\theoremstyle{nonumberplain}

\mdtheorem[innertopmargin=5pt]{lemma}{Lemma}
\mdtheorem[innertopmargin=-5pt]{sol}{Solution}
%\newmdtheoremenv[innertopmargin=-5pt]{sol}{Solution}
\definecolor{MichiganBlue}{HTML}{00274C}
\definecolor{MichiganYellow}{HTML}{FFCB05}  
\definecolor{NicePurple}{RGB}{75,56,76} %PrincePurple
\definecolor{NiceRed}{RGB}{230,37,52}
\definecolor{MidnightBlue}{rgb}{0.1, 0.1, 0.44}
\usepackage[colorlinks=true, linkcolor=MidnightBlue, citecolor=MidnightBlue, urlcolor=MidnightBlue]{hyperref}

\begin{document}
\pagestyle{empty}

\newcommand{\mline}{\vspace{.2in}\hrule\vspace{.2in}}


\noindent
\text{Hunter Lybbert} \\
\text{Student ID: 2426454} \\
\text{11-11-24} \\
\text{AMATH 567} \\

\title{\bf { Homework 7} }


\maketitle
\noindent
Collaborators*: TBD\\
\\
\tiny
\text{*Listed in no particular order. And anyone I discussed at least part of one problem with is considered a collaborator.}
\normalsize


\mline
\begin{enumerate}[label={\bf {\arabic*}:}]
\item  From A\&F: 3.5.1 b, c, d (Only consider singularities in the finite
  complex plane)\\
\item From A\&F: 3.5.3 a, c, d\\
\item Introducing the Gamma function: Do A\&F: 3.6.6. This is the same Gamma function you may have seen defined as
$$
\Gamma(z)=\int_0^{\infty} t^{z-1} e^{-t} d t
$$
This better known representation is only valid for
$\operatorname{Re}(z)>0$. The representation given here is valid in
all of $\mathbb{C}$. It takes a bit of work to show that our
representation is an analytic continuation of the integral
representation (this requires the Dominated Convergence Theorem), but
it is quite doable. Not now though.\\
\item Consider a sequence of numbers $(a_n)_{n \geq 0}$ such that
  $|a_n| < 1$ and
  \begin{align*}
    \sum_{n = 0}^\infty (1 - |a_n|) < \infty.
  \end{align*}
  Define a Blaschke  factor
  \begin{align*}
    B(a,z) = \begin{cases} \frac{|a|}{a} \frac{ a - z}{ 1 - \bar a z}
      & a \neq 0,\\
      z & a  =0.\end{cases}
  \end{align*}
  \begin{itemize}
  \item Show that
  \begin{align*}
    H(z) = \prod_{n=0}^\infty B(a_n,z),
  \end{align*}
  defines an analytic function in the open unit disk $|z| < 1$.
  \item Show that $H(z)$ has zeros at $z = a_n$ for every $n$.  It
    might seem that this construction of an analytic function with an
    infinite number of zeros in a bounded region implies that $H(z) =
    0$ for all $z$.  Why is this not the case?\\
  \end{itemize}

  \item We define the Weierstrass $\wp$-function as

$$
\wp(z)=\frac{1}{z^2}+\sum_{j, k=-\infty}^{\infty}\left(\frac{1}{\left(z-j \omega_1-k \omega_2\right)^2}-\frac{1}{\left(j \omega_1+k \omega_2\right)^2}\right),
$$

where $(j, k)=(0,0)$ is excluded from the double sum. Also, you may
assume that $\omega_1$ is a positive real number, and that $\omega_2$
is on the positive imaginary axis. All considerations below are meant
for the entire complex plane, except the poles of $\wp(z)$.
\begin{enumerate}
\item Show that $\wp\left(z+M \omega_1+N \omega_2\right)=\wp(z)$, for any two integers $M, N$. In other words, $\wp(z)$ is a doubly-periodic function: it has two independent periods in the complex plane. Doubly periodic functions are called elliptic functions.
\item Establish that $\wp(z)$ is an even function: $\wp(-z)=\wp(z)$.
\item Find Laurent expansions for $\wp(z)$ and $\wp^{\prime}(z)$ in a neighborhood of the origin in the form

$$
\wp(z)=\frac{1}{z^2}+\alpha_0+\alpha_2 z^2+\alpha_4 z^4+\ldots
$$

and

$$
\wp^{\prime}(z)=-\frac{2}{z^3}+\beta_1 z+\beta_3 z^3+\ldots
$$


Give expressions for the coefficients introduced above.
\item Show that $\varphi(z)$ satisfies the differential equation

$$
\left(\wp^{\prime}\right)^2=a \wp^3+b \wp^2+c \wp+d,
$$

for suitable choices of $a, b, c, d$. Find these constants. You may need to invoke Liouville's theorem to obtain this final result. It turns out that the function $\varphi(z)$ is determined by the coefficients $c$ and $d$, implying that it is possible to recover $\omega_1$ and $\omega_2$ from the knowledge of $c$ and $d$.
\end{enumerate}
\end{enumerate}
\end{document}

%%% Local Variables:
%%% mode: latex
%%% TeX-master: t
%%% End:
