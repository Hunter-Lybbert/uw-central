\documentclass[10pt]{amsart}
\usepackage[margin=1.4in]{geometry}
\usepackage[usenames,dvipsnames,cmyk]{xcolor} %load first

\usepackage{amssymb,amsmath,enumitem,url}

\newcommand{\D}{\mathrm{d}}
\newcommand{\I}{\mathrm{i}}
\DeclareMathOperator{\E}{e}
\DeclareMathOperator{\OO}{O}
\DeclareMathOperator{\oo}{o}
\DeclareMathOperator{\erfc}{erfc}
\DeclareMathOperator{\real}{Re}
\DeclareMathOperator{\imag}{Im}
\usepackage{tikz}
\usepackage[framemethod=tikz]{mdframed}
\theoremstyle{nonumberplain}

\mdtheorem[innertopmargin=-5pt]{sol}{Solution}
%\newmdtheoremenv[innertopmargin=-5pt]{sol}{Solution}
\definecolor{MichiganBlue}{HTML}{00274C}
\definecolor{MichiganYellow}{HTML}{FFCB05}  
\definecolor{NicePurple}{RGB}{75,56,76} %PrincePurple
\definecolor{NiceRed}{RGB}{230,37,52}
\definecolor{MidnightBlue}{rgb}{0.1, 0.1, 0.44}
\usepackage[colorlinks=true, linkcolor=MidnightBlue, citecolor=MidnightBlue, urlcolor=MidnightBlue]{hyperref}

\begin{document}
\pagestyle{empty}

\newcommand{\mline}{\vspace{.2in}\hrule\vspace{.2in}}

\noindent
\text{Hunter Lybbert} \\
\text{Student ID: 2426454} \\
\text{10-07-24} \\
\text{AMATH 567} \\

\title{\bf { Homework 2 } }


\maketitle
\noindent
Collaborators*: TBD \\
\\
\tiny
\text{*Listed in no particular order. And anyone I discussed at least part of one problem with is considered a colaborator.}
\normalsize

\mline
\begin{enumerate}[label={\bf {\arabic*}:}]
\item From A\&F: 1.2.12. \\
Show that a circle in the $z$ plane corresponds to a circle on the sphere.
(Note the remark following the reference to Figure 1.2.7 in Section 1.2.2)\\
Hints from Office Hours: \\
\begin{itemize}
\item Use 1.2.25 equations
\item perhaps start with a circle centered at the origin for the intuition
\item show the intersection of the plane with a sphere is a circle \\
\end{itemize}
\textit{Solution:} \\
Let's first define an darbitrary complex number $z = x + iy$ and the center of our circle in the complex plane $c = a + ib$.
The following is the equation for a circle on the complex plane
\begin{eqnarray*}
|z - c|^2 &=& r^2 \\
|x + iy - a - ib|^2 &=& r^2 \\
|(x - a) + i(y -b)|^2 &=& r^2 \\
\sqrt{(x - a)^2 + (y - b)^2}^2 &=& r^2 \\
(x - a)^2 + (y - b)^2 &=& r^2
\end{eqnarray*}
From the discussion in A\&F pg16-18, we know that we have the following formulae for converting between the complex plane and the 3 space $(X, Y, Z)$
$$X = \frac{4x}{|z|^2 + 4} \quad Y = \frac{4y}{|z|^2 + 4} \quad Z = \frac{2|z|^2}{|z|^2 + 4}.$$
Let's cmobine these
\begin{eqnarray*}
(x - a)^2 + (y - b)^2 &=& r^2 \\
x^2 -2ax + a^2 + y^2 -2by + b^2 &=& r^2 \\
x^2 + y^2 -2ax + a^2 -2by + b^2 &=& r^2 \\
x^2 + y^2 &=& r^2 +2ax - a^2 +2by - b^2 \\
|z|^2 &=& r^2 +2ax - a^2 +2by - b^2 \\
\end{eqnarray*}
We want to find $A$, $B$, $C$, and $D$ that are real s.t. $$AX + BY + CZ = D$$
\begin{eqnarray*}
X &=& \frac{4x}{r^2 +2ax - a^2 +2by - b^2 + 4} \\
Y &=& \frac{4y}{r^2 +2ax - a^2 +2by - b^2 + 4} \\
Z &=& \frac{2\left(r^2 +2ax - a^2 +2by - b^2\right)}{r^2 +2ax - a^2 +2by - b^2 + 4}
\end{eqnarray*}
Then our equation is
\begin{eqnarray*}
&& A\frac{4x}{r^2 +2ax - a^2 +2by - b^2 + 4}  \\
&+& B\frac{4y}{r^2 +2ax - a^2 +2by - b^2 + 4} \\
&+& C\frac{2\left(r^2 +2ax - a^2 +2by - b^2\right)}{r^2 +2ax - a^2 +2by - b^2 + 4} \\
&=& D
\end{eqnarray*}
We have projected the circle from the complex plane onto the sphere. Notice we've constructed this equation showing they all lie in the plane. The intersection of the sphere and this plane is a subset of a circle on the plane. Since we were working with equalities throughout it is actually the whole circle on the sphere.
\textbf{Almost Complete} \\

\item From A\&F: 1.3.5.\\
Show that the functions $\Re(z)$ and $\Im(z)$ are nowhere differentiable. \\
\textit{Solution:} \\
\textbf{Part 1:} $f(z) = \Re(z)$ \\
Let's begin with the definition of the derivative
\begin{eqnarray*}
\lim_{h \rightarrow 0} \frac{f(z + h) - f(z)}{h} &=& \lim_{h \rightarrow 0} \frac{f(x_z + x_h + i (y_z + y_h) ) - x_z}{h} \\
								 &=& \lim_{h \rightarrow 0} \frac{x_z + x_h - x_z}{x_h + iy_h} \\
								 &=& \lim_{h \rightarrow 0} \frac{x_h}{x_h + iy_h} = \frac{0}{0}.						
\end{eqnarray*}
Now we want to use L'Hôpital's rule so we create two cases fixing $y_h$ and $x_h$ each in turn.
Starting from where we left off
$$ \lim_{h \rightarrow 0} \frac{x_h}{x_h + iy_h} = \lim_{h \rightarrow 0} \frac{ \frac{d}{dx_h}x_h}{\frac{d}{dx_h}(x_h + iy_h)} = \lim_{h \rightarrow 0} \frac{1}{1} = 1 \: \text{when $y_h$ is fixed} $$
$$ \lim_{h \rightarrow 0} \frac{x_h}{x_h + iy_h} = \lim_{h \rightarrow 0} \frac{ \frac{d}{dy_h}x_h}{\frac{d}{dy_h}(x_h + iy_h)} = \lim_{h \rightarrow 0} \frac{0}{i} = 0 \: \text{when $x_h$ is fixed}. $$
Since we get a different result at an arbitrary $z$ depending on which $x_h$ and $y_h$ is currently fixed the limit does not exist and therefore $f(z) = \Re(z)$ is nowhere differentiable. \\
\qed

\noindent
\textbf{Part 2:} $f(z) = \Im(z)$ \\
For this one we will use a different method. Recall that the Cauchy-Riemann equations are a necessary condition that must hold if $f(z)$ is differentiable (A\&F pg. 33). Therefore we can show that the Cauchy-Riemann equations do not hold and therefore the function is non differentiable. We use the fact that $f(z) = u(x, y) + i v(x, y)$ and 
$$f(z) = \Im(z) = y \quad\text{since $z = x +iy$}$$
to get 
\begin{eqnarray*}
u(x, y) &=& y \\
v(x, y) &=& 0.
\end{eqnarray*}
Now we want to check if both of the following hold
$$u_x = v_y$$
$$v_x = - u_y.$$
Let's calculate them
$$u_x = 0, v_y = 0, v_x = 0, u_y = 1$$
Therefore the first condition holds
\begin{eqnarray*}
u_x &=& v_y \\
0 &=& 0 \\
\end{eqnarray*}
however, the second does not
\begin{eqnarray*}
v_x = - u_y \\
0 \neq - 1.
\end{eqnarray*}
Therefore $f(z) = \Im(z)$ is nowhere differentiable. \\
\qed

\item Consider the function
    \begin{align*}
      \varphi(z) = z + \sqrt{z^2 - 1}, \quad z > 1.
    \end{align*}
    Show that
    \begin{align*}
      \log \varphi(z) = \int_1^z \frac{\D x}{\sqrt{x^2 - 1}}.
    \end{align*}
\textit{Solution:} \\
Let's begin from evaluating the integral on the right and showing that the result is what we have on the left. First we substitute $x = \sec\theta$, $\D x = \sec\theta\tan\theta \D \theta$
\begin{eqnarray*}
\int_1^z \frac{\D x}{\sqrt{x^2 - 1}} &=& \int_0^{arc\sec z} \frac{\sec\theta\tan\theta \D \theta}{\sqrt{{\sec^2\theta} - 1}} \\ \\
&=& \int_0^{arc\sec z} \frac{ \sec \theta \tan \theta \D \theta}{\sqrt{\tan^2 \theta}} \\ \\
&=& \int_0^{arc\sec z} \frac{ \sec \theta \tan \theta \D \theta}{\tan \theta} \\ \\
&=& \int_0^{arc\sec z} \sec \theta \D \theta \\
&=& \log\left| \sec \theta + \tan\theta \right| |_{0}^{arc\sec z} \\
&=& \log\left| \sec (arc\sec z) + \tan(arc\sec z)\right| - \log\left| \sec 0 + \tan 0 \right| \\
&=& \log\left| z + \sqrt{\tan^2(arc\sec z)}\right| - \log\left| 1 + 0 \right| \\
&=& \log\left| z + \sqrt{\sec^2(arc\sec z) - 1}\right| \\
&=& \log\left| z + \sqrt{z^2 - 1} \right| \\
&=& \log\left| z + \sqrt{z^2 - 1} \right| \\
&=& \log\left|\varphi(z)\right| \\
&=& \log(\varphi(z)) \quad \text{since} \: z > 1 \: \text{then} \: \varphi(z) > 0 
\end{eqnarray*}
\qed
\item Find all zeroes of $\tan (z), z \in \mathbb{C}$. What can you
  conclude about the zeroes of $\tanh (z)=\sinh (z) / \cosh (z), z \in
  \mathbb C$?\\
\textit{Solution:} \\
Let's have some fun with trig identities 
$$\tan(z) = \frac{\sin(x + iy)}{\cos(x + iy)} = \frac{\sin x\cos iy + \cos x \sin iy }{\cos x \cos iy  - \sin x \sin iy}$$
Now it's important to note the following before we proceed:
\begin{eqnarray*}
\cos iy &=& \frac{e^{i(iy)} + e^{-i(iy)}}{2}  = \frac{e^{-y} + e^{y}}{2} = \frac{e^{y} + e^{-y}}{2} =\cosh y \\ \\
\sin iy &=& \frac{e^{i(iy)} - e^{-i(iy)}}{2i} = \frac{-i}{-i} \frac{e^{-y} - e^{y}}{2i} = \frac{-i(e^{-y} - e^{y})}{2} = i \frac{e^{y} - e^{-y}}{2} = i \sinh y
\end{eqnarray*}
\begin{eqnarray*}
\frac{\sin x\cos iy + \cos x \sin iy }{\cos x \cos iy  - \sin x \sin iy} &=& \frac{\sin x\cosh y + i \cos x \sinh y }{\cos x \cosh y  - i\sin x \sinh y } \\ \\
&=& \frac{\sin x\cosh y + i \cos x \sinh y }{\cos x \cosh y  - i\sin x \sinh y } \frac{(\cos x \cosh y  + i\sin x \sinh y )}{(\cos x \cosh y  + i\sin x \sinh y )} \\ \\
&=& \frac{\sin x\cosh y + i \cos x \sinh y }{\cos^2 x \cosh^2 y  + \sin^2 x \sinh^2 y } (\cos x \cosh y  + i\sin x \sinh y ) \\ \\
&=& \frac{\sin x \cos x\cosh^2 y + i \cos^2 x \sinh y \cosh y + i\sin^2 x\cosh y \sinh y - \cos x \sin x  \sinh^2} {\cos^2 x \cosh^2 y  + \sin^2 x \sinh^2 y } \\ \\
&=& \frac{(\sin x \cos x\cosh^2 y - \cos x \sin x  \sinh^2 y) + i (\cos^2 x \sinh y \cosh y + \sin^2 x\cosh y \sinh y)} {\cos^2 x \cosh^2 y  + \sin^2 x \sinh^2 y } \\ \\
&=& \frac{\sin x \cos x(\cosh^2 y - \sinh^2 y) + i \sinh y \cosh y (\cos^2 x + \sin^2 x)} {\cos^2 x \cosh^2 y  + \sin^2 x \sinh^2 y } \\ \\
&=& \frac{\sin x \cos x + i \sinh y \cosh y} {\cos^2 x \cosh^2 y  + \sin^2 x \sinh^2 y } \\ \\
\end{eqnarray*}
Since $\sin x \cos x$ is always real, nothing will cancel out the $i$ in $i \sinh y \cosh y$ unless $\sinh y \cosh y = 0$.
Therefore we need both of the following to hold
$$ \sin x \cos x = 0 $$
$$ \sinh y \cosh y = 0. $$
Looking first at $ \sin x \cos x = 0 $ we know this is 0 if $\sin x = 0$ or $\cos x = 0$ therefore $x = \frac{k\pi}{2}, \: k \in \mathbb{Z}.$
Now we need $\sinh y \cosh y = 0$ to hold, which implies $y = 0$ since $\cosh y$ is never 0 and $\sinh y = 0$ only when $y = 0$ therefore $y = 0$.
Putting in this definite value of $y$ to the above we end up with \\
$$\frac{\sin x \cos x + i \sinh y \cosh y} {\cos^2 x \cosh^2 y  + \sin^2 x \sinh^2 y } = \frac{\sin x \cos x} {\cos^2 x}$$ \\
Now we don't want $\cos^2x = 0$ in our denominator so now we limit $x = k\pi$, $k \in \mathbb{Z}$
Therefore, the zeros of $\tan (z), z \in \mathbb{C}$ are $z = k\pi$ for $k \in \mathbb{Z}$ since $z = x + iy$, $x = k\pi$, and $y = 0$ as established above.\\

\noindent
From the above argument and conclusion about the zeroes of $\tan (z), z \in \mathbb{C}$ we can also draw some conclusions about the zeroes of $\tanh (z), z \in \mathbb{C}$.
What we learn is
\begin{eqnarray*}
\tanh z = \frac{\sinh z}{\cosh z} &=& \frac{\frac{e^z - e^{-z}}{2}}{\frac{e^z + e^{-z}}{2}} \\ \\
 					       &=& \frac{e^z - e^{-z}}{e^z + e^{-z}} \\ \\
					       &=& \frac{e^z - e^{-z}}{e^z + e^{-z}} \frac{e^z}{e^z} \\ \\
					       &=& \frac{e^{2z} -  1}{e^{2z} + 1}. \\
\end{eqnarray*}
In order for $e^{2z} -  1 = 0$ this implies $e^{2z} = 1$.
\begin{eqnarray*}
1 &=& e^{2z} \\
   &=& e^{2(x + iy)} \\
   &=& e^{2x}e^{2iy} \\
   &=& e^{2x}(\cos(2y) + i \sin(2y)) \\
\end{eqnarray*}
For the expression on the right to be equal to $1$ the imaginary part of the right side needs to satisfy $i \sin(2y) = 0$.
This only holds at $y = \ell \pi$, $\ell \in \mathbb{Z}$.
Now we have 
\begin{eqnarray*}
1 &=& e^{2x}(\cos(2y) + i \sin(2y)) \\
   &=& e^{2x}(\cos(\ell 2\pi) + i \sin(\ell 2 \pi)) \\
   &=& e^{2x}(1 + 0) \\
   &=& e^{2x} \\
\end{eqnarray*}
Which implies $x = 0$ in order to satisfy $ e^{2x} = 1$.
In conclusion the zeroes of $\tanh(z), \: z \in \mathbb{C}$ are at $ z = i \ell \pi, \: \ell \in \mathbb{Z}$. Interestingly enough the zeroes of $\tan(z)$ are at real multiples of $\pi$ and the zeroes of $\tanh(z)$ are at the imaginary multiples of $\pi$.
\qed
\\

\item Consider $f_\epsilon(z)=\epsilon /\left(\epsilon^2+z^2\right)$, where
  $\epsilon$ is a small positive number, and $z \in \mathbb{C} /\{i
  \epsilon,-i \epsilon\}$. Plot $\left|f_\epsilon(z)\right|$, for various
  values of $\epsilon$. Discuss the influence the singularities of a
  function in the complex plane have on its behavior on the real
  line. Compute
  \begin{align*}
    \int_{-\infty}^\infty f_\epsilon(x) \D x.
  \end{align*}\\
\textit{Solution:} \\
\textbf{Incomplete} \\
\item Visualizing complex functions is not as easy as visualizing
  real-valued functions, since we need 4 dimensions: two for the input,
  two for the output. Different visualizations are commonly used, such
  as showing 3-dimensional plots of the real and imaginary
  parts. Plotting the modulus is informational, but it eliminates a
  lot of information.
  \begin{itemize}
\item To see this, plot the real and imaginary part of the exponential function $\exp (z)=\exp (x+\I y)$, for $x \in[-1,1], y \in[-2 \pi, 2 \pi]$. Now plot the modulus over the same region, and compare.
\item A "new" popular way to do this is to plot the modulus of the
function with the color defined by the phase. The \href{https://dlmf.nist.gov/}{Digital Library of
Mathematical Functions} has lots of examples. Create a plot of the
$|\exp (z)|=|\exp (x+\I y)|$, for $x \in[-1,1]$, $y \in[-2 \pi, 2
\pi]$. colored by the argument. Experimenting with other functions is
highly encouraged! The book visual Complex Functions: An Introduction
with Phase Portraits by Elias Wegert (Birkhäuser, 2012) is a good
companion to our textbook, if you think geometrically.
\end{itemize}
\textit{Solution:} \\
\textbf{Incomplete} \\
\item From A\&F: 2.1.1 \\
Which of the following satisfy the Cauchy-Riemann (C-R) equations? If they satisfy the C-R equations, give the analytic function of z. \\
a) $f(x, y) = x - iy + 1$ \\
\textit{Solution:} \\
Identify $u$ and $v$ and their partial derivatives
$$ u(x,y) = x + 1 \implies u_x = 1, \: u_y = 0$$
$$v(x,y) = - y \implies v_x = 0, \: v_y = -1$$
Therefore, $v_x = 0 = - 0 = - u_y$ holds, however $u_x = 1 \neq - 1 = v_y$ does not hold.
In conclusion $f(x, y) = x - iy + 1$ does not satisfy the Cauchy-Riemann (C-R) equations.\\
\qed

\noindent
b) $f(x, y) = y^3 - 3x^2y + i(x^3 - 3xy^2 + 2)$ \\
\textit{Solution:} \\
Identify $u$ and $v$ and their partial derivatives
$$ u(x,y) = y^3 - 3x^2y \implies u_x = -6xy, \: u_y = 3y^2 -3x^2$$
$$v(x,y) = x^3 - 3xy^2 + 2 \implies v_x = 3x^2 - 3y^2, \: v_y = -6xy$$
Therefore, $v_x = 3x^2 - 3y^2 = - (3y^2 -3x^2) = - u_y$ and $u_x = -6xy = -6xy = v_y$ both hold.
In conclusion $f(x, y) = y^3 - 3x^2y + i(x^3 - 3xy^2 + 2)$ satisfies the Cauchy-Riemann (C-R) equations.
To determine the analytic function notice
\begin{eqnarray*}
f(x, y) &=& y^3 - 3x^2y + i(x^3 - 3xy^2 + 2) \\
	 &=& y^3 - 3x^2y + ix^3 - i3xy^2 + i2
\end{eqnarray*}
\begin{eqnarray*}
z^3 &=& (x + iy)^3 \\
      &=& (x + iy)(x + iy)(x + iy) \\
      &=& (x^2 + i2xy - y^2)(x + iy) \\
      &=& x^3 + i3x^2y - 3xy^2 - iy^3 \\
      &=& -i(ix^3 - 3x^2y - i3xy^2 + y^3) \\
      &=& -i(y^3 - 3x^2y + ix^3 - i3xy^2) \\
      &=& -i(f(x,y) - 2i).
\end{eqnarray*}
Continuing
\begin{eqnarray*}
z^3 &=& -i(f(x,y) - 2i) \\
\frac{z^3}{-i} &=& f(x,y) - 2i \\
\frac{z^3}{-i}\frac{i}{i} &=& f(x,y) - 2i \\
iz^3 &=& f(x,y) - 2i \\
iz^3 + 2i &=& f(x,y).
\end{eqnarray*}
Therefore, the analytic function of $z$ is $f(z) = iz^3 + 2i$. \\
\qed

\noindent
c) $f(x, y) = e^y(\cos x + i\sin y)$ \\
\textit{Solution:} \\
Identify $u$ and $v$ and their partial derivatives
$$ f(x, y) = e^y(\cos x + i\sin y) = e^y\cos x + ie^y\sin y $$
$$ u(x,y) = e^y\cos x \implies u_x = -e^y \sin x, \: u_y = e^y\cos x $$
$$ v(x,y) = e^y\sin y \implies v_x = 0, \: v_y = e^y\sin y + e^y \cos y $$
Therefore neither
$$v_x = 0 \neq - e^y\cos x = - u_y$$ nor $$u_x = -e^y \sin x \neq e^y\sin y + e^y \cos y = v_y$$ hold.
In conclusion $f(x, y) = e^y(\cos x + i\sin y)$ does not satisfy the Cauchy-Riemann (C-R) equations. \\
\qed
\item From A\&F: 2.1.7 \\
Consider the complex analytic function, $\Omega(z) = \phi(x, y) + i\psi(x, y)$, in a domain $D$.
Let us transform from $z$ to $w$ using $w = f(z)$, $w = u + iv$ where $f(z)$ is analytic in $D$, with the corresponding domain in the $w$ plane, $D'$.
Establish the following: \\
\textbf{Part 1:}
$$
\frac{\partial\phi}{\partial x} = \frac{\partial u}{\partial x} \frac{\partial\phi}{\partial u} + \frac{\partial v}{\partial x} \frac{\partial\phi}{\partial v}
$$
\textit{Solution:}\\
Notation is very important for this problem. 
Notice the problem sets up that $f(z) = w$ where $w = u + iv$.
We typically say $f(z)$ can be written in a form s.t. $f(z) = u(x,y) + iv(x, y)$ so we really have $f(z) = w = u(x,y) + iv(x,y)$ where $x$ and $y$ are the real and imaginary parts of $z$, respectively.
This gives us \\
$$
\Omega(f(z)) = \Omega(w) = \phi(u(x, y), v(x, y)) + i\psi(u(x, y), v(x, y)).
$$ \\
Therefore we arrive at a function $\phi(u(x, y), v(x, y))$.
Taking the derivative of this with respect to $x$ requires the chain rule.
Applying the chain rule once, we immediately get the following \\
$$\frac{\partial\phi}{\partial x} = \frac{\partial\phi}{\partial u} \frac{\partial u}{\partial x} + \frac{\partial\phi}{\partial v} \frac{\partial v}{\partial x}.$$
\qed \\
\textbf{Part 2:}
\begin{eqnarray*}
\frac{\partial^2\phi}{\partial x^2} &=& \frac{\partial^2 u}{\partial x^2} \frac{\partial\phi}{\partial u}
							- \frac{\partial^2 u}{\partial x \partial y} \frac{\partial\phi}{\partial v}
							+ \left(\frac{\partial v}{\partial x}\right)^2 \frac{\partial^2 \phi}{\partial u^2}
							- 2 \frac{\partial u}{\partial x} \frac{\partial u}{\partial y} \frac{\partial^2 \phi}{\partial u \partial v}
							+ \left(\frac{\partial u}{\partial y}\right)^2 \frac{\partial^2 \phi}{\partial v^2} \\
\end{eqnarray*}
\textit{Solution:} \\
Starting from our previous result
\begin{eqnarray*}
\frac{\partial\phi}{\partial x} &=& \frac{\partial\phi}{\partial u} \frac{\partial u}{\partial x} + \frac{\partial\phi}{\partial v} \frac{\partial v}{\partial x} \\ \\
\frac{\partial}{\partial x} \left(\frac{\partial\phi}{\partial x}\right) &=& \frac{\partial}{\partial x} \left(\frac{\partial\phi}{\partial u} \frac{\partial u}{\partial x} + \frac{\partial\phi}{\partial v} \frac{\partial v}{\partial x}\right) \\ \\
\frac{\partial^2\phi}{\partial x^2} &=& \frac{\partial}{\partial x} \left(\frac{\partial\phi}{\partial u} \frac{\partial u}{\partial x} + \frac{\partial\phi}{\partial v} \frac{\partial v}{\partial x}\right) \\ \\
						&=& \frac{\partial}{\partial x} \left(\frac{\partial\phi}{\partial u} \frac{\partial u}{\partial x}\right)
							+ \frac{\partial}{\partial x}  \left(\frac{\partial\phi}{\partial v} \frac{\partial v}{\partial x}\right) \\ \\
						&=& \left[ 
							\frac{\partial}{\partial x} \left( \frac{\partial\phi}{\partial u} \right) \cdot \frac{\partial u}{\partial x}
							+ \frac{\partial\phi}{\partial u} \cdot \frac{\partial}{\partial x} \left(\frac{\partial u}{\partial x} \right)
						\right] \\ \\
						&& + \left[ 
							\frac{\partial}{\partial x} \left( \frac{\partial\phi}{\partial v} \right) \cdot \frac{\partial v}{\partial x}
							+ \frac{\partial\phi}{\partial v} \cdot \frac{\partial}{\partial x} \left(\frac{\partial v}{\partial x} \right)
						\right] \\ \\
						&=& \left[ 
							\left( \frac{\partial^2 \phi}{\partial u^2} \frac{\partial u}{\partial x} + \frac{\partial^2 \phi}{\partial u \partial v} \frac{\partial v}{\partial x} \right) \frac{\partial u}{\partial x}
							+ \frac{\partial\phi}{\partial u} \frac{\partial^2 u}{\partial x^2}
						\right] \\ \\
						&& + \left[ 
							\left( \frac{\partial^2 \phi}{\partial v \partial u} \frac{\partial u}{\partial x} + \frac{\partial^2 \phi}{\partial v^2} \frac{\partial v}{\partial x} \right) \frac{\partial v}{\partial x}
							+ \frac{\partial\phi}{\partial v} \frac{\partial^2 v}{\partial x^2}
						\right] \\ \\
						&=& \frac{\partial^2 \phi}{\partial u^2} \left(\frac{\partial u}{\partial x}\right)^2 + \frac{\partial^2 \phi}{\partial u \partial v} \frac{\partial v}{\partial x} \frac{\partial u}{\partial x}
							+ \frac{\partial\phi}{\partial u} \frac{\partial^2 u}{\partial x^2} \\ \\
						&& + \frac{\partial^2 \phi}{\partial v \partial u} \frac{\partial u}{\partial x}\frac{\partial v}{\partial x} + \frac{\partial^2 \phi}{\partial v^2} \left(\frac{\partial v}{\partial x}\right)^2
							+ \frac{\partial\phi}{\partial v} \frac{\partial^2 v}{\partial x^2} \\ \\
\end{eqnarray*}
Now let's utilize the fact that $f(z)$ is analytic therefore it satisfies the Cauchy-Riemann equations implying
$$\frac{\partial u}{\partial x} = \frac{\partial v}{\partial y} \quad \text{and} \quad \frac{\partial v}{\partial x} = - \frac{\partial u}{\partial y}.$$
Making use of the second equality we proceed in our calculation of $\frac{\partial^2\phi}{\partial x^2}$ by making few substitutions
\begin{eqnarray*}
\frac{\partial^2\phi}{\partial x^2} &=& \frac{\partial^2 \phi}{\partial u^2} \left(\frac{\partial u}{\partial x}\right)^2 + \frac{\partial^2 \phi}{\partial u \partial v} \left(- \frac{\partial u}{\partial y}\right) \frac{\partial u}{\partial x}
							+ \frac{\partial\phi}{\partial u} \frac{\partial^2 u}{\partial x^2} \\
						&& + \frac{\partial^2 \phi}{\partial v \partial u} \frac{\partial u}{\partial x}\left(- \frac{\partial u}{\partial y}\right) + \frac{\partial^2 \phi}{\partial v^2} \left(- \frac{\partial u}{\partial y}\right)^2
							+ \frac{\partial\phi}{\partial v} \frac{\partial^2 v}{\partial x^2} \\ \\
						&=& \frac{\partial^2 \phi}{\partial u^2} \left(\frac{\partial u}{\partial x}\right)^2
							- 2 \frac{\partial u}{\partial x} \frac{\partial u}{\partial y} \frac{\partial^2 \phi}{\partial u \partial v}
							+ \frac{\partial\phi}{\partial u} \frac{\partial^2 u}{\partial x^2}
							+ \frac{\partial^2 \phi}{\partial v^2} \left(\frac{\partial u}{\partial y}\right)^2
							+ \frac{\partial\phi}{\partial v} \frac{\partial^2 v}{\partial x^2} \\ \\
						&=& \frac{\partial\phi}{\partial u} \frac{\partial^2 u}{\partial x^2}
							+ \frac{\partial\phi}{\partial v} \frac{\partial^2 v}{\partial x^2}
							+  \left(\frac{\partial u}{\partial x}\right)^2 \frac{\partial^2 \phi}{\partial u^2}
							- 2 \frac{\partial u}{\partial x} \frac{\partial u}{\partial y} \frac{\partial^2 \phi}{\partial u \partial v}
							+  \left(\frac{\partial u}{\partial y}\right)^2\frac{\partial^2 \phi}{\partial v^2} \\
						&=& \frac{\partial\phi}{\partial u} \frac{\partial^2 u}{\partial x^2}
							+ \frac{\partial\phi}{\partial v}  \frac{\partial}{\partial x} \left(\frac{\partial v}{\partial x}\right)
							+  \left(\frac{\partial u}{\partial x}\right)^2 \frac{\partial^2 \phi}{\partial u^2}
							- 2 \frac{\partial u}{\partial x} \frac{\partial u}{\partial y} \frac{\partial^2 \phi}{\partial u \partial v}
							+  \left(\frac{\partial u}{\partial y}\right)^2\frac{\partial^2 \phi}{\partial v^2} \\
						&=& \frac{\partial\phi}{\partial u} \frac{\partial^2 u}{\partial x^2}
							+ \frac{\partial\phi}{\partial v}  \frac{\partial}{\partial x} \left(- \frac{\partial u}{\partial y}\right)
							+  \left(\frac{\partial u}{\partial x}\right)^2 \frac{\partial^2 \phi}{\partial u^2}
							- 2 \frac{\partial u}{\partial x} \frac{\partial u}{\partial y} \frac{\partial^2 \phi}{\partial u \partial v}
							+  \left(\frac{\partial u}{\partial y}\right)^2\frac{\partial^2 \phi}{\partial v^2} \\
						&=& \frac{\partial\phi}{\partial u} \frac{\partial^2 u}{\partial x^2}
							- \frac{\partial\phi}{\partial v} \frac{\partial^2 u}{\partial x \partial y}
							+  \left(\frac{\partial u}{\partial x}\right)^2 \frac{\partial^2 \phi}{\partial u^2}
							- 2 \frac{\partial u}{\partial x} \frac{\partial u}{\partial y} \frac{\partial^2 \phi}{\partial u \partial v}
							+  \left(\frac{\partial u}{\partial y}\right)^2\frac{\partial^2 \phi}{\partial v^2}.
\end{eqnarray*}
\qed \\
\noindent
\textbf{Part 3:} 
Also find the corresponding formulae for $\frac{\partial \phi}{\partial y}$ and $\frac{\partial^2 \phi }{\partial y^2}$. \\
\textit{Solution:} \\
Recall that we are looking at $\phi(u(x, y), v(x, y))$\\
Taking the derivative of this with respect to $y$ requires the chain rule.
Applying the chain rule once, we immediately get the following \\
$$\frac{\partial\phi}{\partial y} = \frac{\partial\phi}{\partial u} \frac{\partial u}{\partial y} + \frac{\partial\phi}{\partial v} \frac{\partial v}{\partial y}.$$
And now continuing on with the second derivative we get:
\begin{eqnarray*}
\frac{\partial\phi}{\partial y} &=& \frac{\partial\phi}{\partial u} \frac{\partial u}{\partial y} + \frac{\partial\phi}{\partial v} \frac{\partial v}{\partial y} \\ \\
\frac{\partial}{\partial y} \left(\frac{\partial\phi}{\partial y}\right) &=& \frac{\partial}{\partial y} \left(\frac{\partial\phi}{\partial u} \frac{\partial u}{\partial y} + \frac{\partial\phi}{\partial v} \frac{\partial v}{\partial y}\right) \\ \\
\frac{\partial^2\phi}{\partial y^2} &=& \frac{\partial}{\partial y} \left(\frac{\partial\phi}{\partial u} \frac{\partial u}{\partial y} + \frac{\partial\phi}{\partial v} \frac{\partial v}{\partial y}\right) \\ \\
						&=& \frac{\partial}{\partial y} \left(\frac{\partial\phi}{\partial u} \frac{\partial u}{\partial y}\right)
							+ \frac{\partial}{\partial y}  \left(\frac{\partial\phi}{\partial v} \frac{\partial v}{\partial y}\right) \\ \\
						&=& \left[ 
							\frac{\partial}{\partial y} \left( \frac{\partial\phi}{\partial u} \right) \cdot \frac{\partial u}{\partial y}
							+ \frac{\partial\phi}{\partial u} \cdot \frac{\partial}{\partial y} \left(\frac{\partial u}{\partial y} \right)
						\right] \\ \\
						&& + \left[ 
							\frac{\partial}{\partial y} \left( \frac{\partial\phi}{\partial v} \right) \cdot \frac{\partial v}{\partial y}
							+ \frac{\partial\phi}{\partial v} \cdot \frac{\partial}{\partial y} \left(\frac{\partial v}{\partial y} \right)
						\right] \\ \\
						&=& \left[ 
							\left( \frac{\partial^2 \phi}{\partial u^2} \frac{\partial u}{\partial y} + \frac{\partial^2 \phi}{\partial u \partial v} \frac{\partial v}{\partial y} \right) \frac{\partial u}{\partial y}
							+ \frac{\partial\phi}{\partial u} \frac{\partial^2 u}{\partial y^2}
						\right] \\ \\
						&& + \left[ 
							\left( \frac{\partial^2 \phi}{\partial v \partial u} \frac{\partial u}{\partial y} + \frac{\partial^2 \phi}{\partial v^2} \frac{\partial v}{\partial y} \right) \frac{\partial v}{\partial y}
							+ \frac{\partial\phi}{\partial v} \frac{\partial^2 v}{\partial y^2}
						\right] \\ \\
						&=& \frac{\partial^2 \phi}{\partial u^2} \left(\frac{\partial u}{\partial y}\right)^2 + \frac{\partial^2 \phi}{\partial u \partial v} \frac{\partial v}{\partial y} \frac{\partial u}{\partial y}
							+ \frac{\partial\phi}{\partial u} \frac{\partial^2 u}{\partial y^2} \\ \\
						&& + \frac{\partial^2 \phi}{\partial v \partial u} \frac{\partial u}{\partial y}\frac{\partial v}{\partial y} + \frac{\partial^2 \phi}{\partial v^2} \left(\frac{\partial v}{\partial y}\right)^2
							+ \frac{\partial\phi}{\partial v} \frac{\partial^2 v}{\partial y^2} \\ \\
\end{eqnarray*}
We need to decide where our substitutions are going to take place
\begin{eqnarray*}
\frac{\partial^2\phi}{\partial y^2} &=& \frac{\partial^2 \phi}{\partial u^2} \left(\frac{\partial u}{\partial y}\right)^2 + \frac{\partial^2 \phi}{\partial u \partial v} \frac{\partial v}{\partial y} \left(- \frac{\partial v}{\partial x}\right)
							+ \frac{\partial\phi}{\partial u} \frac{\partial}{\partial y} \left(- \frac{\partial v}{\partial x} \right)\\ \\
						&& + \frac{\partial^2 \phi}{\partial v \partial u} \left(- \frac{\partial v}{\partial x}\right)\frac{\partial v}{\partial y} + \frac{\partial^2 \phi}{\partial v^2} \left(\frac{\partial v}{\partial y}\right)^2
							+ \frac{\partial\phi}{\partial v} \frac{\partial^2 v}{\partial y^2} \\ \\
						&=& \frac{\partial^2 \phi}{\partial u^2} \left(\frac{\partial u}{\partial y}\right)^2
							-2 \frac{\partial^2 \phi}{\partial u \partial v} \frac{\partial v}{\partial y} \frac{\partial v}{\partial x}
							- \frac{\partial\phi}{\partial u} \frac{\partial^2 v}{\partial x \partial y}
							+ \frac{\partial^2 \phi}{\partial v^2} \left(\frac{\partial v}{\partial y}\right)^2
							+ \frac{\partial\phi}{\partial v} \frac{\partial^2 v}{\partial y^2} \\ \\
						&=& \frac{\partial\phi}{\partial v} \frac{\partial^2 v}{\partial y^2} 
							- \frac{\partial\phi}{\partial u} \frac{\partial^2 v}{\partial x \partial y}
							+ \left(\frac{\partial u}{\partial y}\right)^2 \frac{\partial^2 \phi}{\partial u^2}
							-2 \frac{\partial^2 \phi}{\partial u \partial v} \frac{\partial v}{\partial y} \frac{\partial v}{\partial x}
							+ \left(\frac{\partial v}{\partial y}\right)^2 \frac{\partial^2 \phi}{\partial v^2}. \\ \\
\end{eqnarray*}
Therefore we have derived the similar formula for $\frac{\partial^2 \phi }{\partial y^2}$ as we had for $\frac{\partial^2 \phi }{\partial x^2}$.
\qed \\

\noindent
\textbf{Part 3:}  Recall that $f'(z) = \frac{\partial u }{\partial x} + i\frac{\partial u}{\partial y}$, and $u(x, y)$ satisfies Laplace's equation the domain D.
Show that
$$
\nabla_{x, y}^2 \phi = 
\frac{\partial^2\phi}{\partial x^2} + \frac{\partial^2 \phi }{\partial y^2} = 
\left(u_x^2 + y_y^2\right) \left(\frac{\partial^2\phi}{\partial u^2} + \frac{\partial^2 \phi }{\partial v^2}\right) =
\left| f'(z)\right|^2 \nabla_{u, v}^2 \phi
$$
\textit{Solution:}
The first equality comes by definition of $\nabla_{x, y}^2 \phi$.
Similarly the third equality comes from the definition of $f'(z)$, the modulus and $\nabla_{u, v}^2 \phi$.
So what we really need to show is that
$$
\frac{\partial^2\phi}{\partial x^2} + \frac{\partial^2 \phi }{\partial y^2} = 
\left(u_x^2 + u_y^2\right) \left(\frac{\partial^2\phi}{\partial u^2} + \frac{\partial^2 \phi }{\partial v^2}\right).
$$
Lets get to it
\begin{eqnarray*}
\frac{\partial^2\phi}{\partial x^2} + \frac{\partial^2 \phi }{\partial y^2} &=& \left[\frac{\partial\phi}{\partial u} \frac{\partial^2 u}{\partial x^2}
													- \frac{\partial\phi}{\partial v} \frac{\partial^2 u}{\partial x \partial y}
													+  \left(\frac{\partial u}{\partial x}\right)^2 \frac{\partial^2 \phi}{\partial u^2}
													-2 \frac{\partial u}{\partial x} \frac{\partial u}{\partial y} \frac{\partial^2 \phi}{\partial u \partial v}
													+  \left(\frac{\partial u}{\partial y}\right)^2\frac{\partial^2 \phi}{\partial v^2} \right] \\ \\
												  && + \left[\frac{\partial\phi}{\partial v} \frac{\partial^2 v}{\partial y^2} 
													- \frac{\partial\phi}{\partial u} \frac{\partial^2 v}{\partial x \partial y}
													+ \left(\frac{\partial u}{\partial y}\right)^2 \frac{\partial^2 \phi}{\partial u^2}
													-2 \frac{\partial^2 \phi}{\partial u \partial v} \frac{\partial v}{\partial y} \frac{\partial v}{\partial x}
													+ \left(\frac{\partial v}{\partial y}\right)^2 \frac{\partial^2 \phi}{\partial v^2} \right]\\ \\
												   &=& \frac{\partial\phi}{\partial u} \frac{\partial^2 u}{\partial x^2}
													+ \frac{\partial\phi}{\partial v} \frac{\partial^2 v}{\partial y^2} 
													- \frac{\partial\phi}{\partial v} \frac{\partial^2 u}{\partial x \partial y}
													- \frac{\partial\phi}{\partial u} \frac{\partial^2 v}{\partial x \partial y}
													-2 \frac{\partial u}{\partial x} \frac{\partial u}{\partial y} \frac{\partial^2 \phi}{\partial u \partial v}
													-2 \frac{\partial^2 \phi}{\partial u \partial v} \frac{\partial v}{\partial y} \frac{\partial v}{\partial x} \\ \\
												   && +  \left(\frac{\partial u}{\partial x}\right)^2 \frac{\partial^2 \phi}{\partial u^2}
													+ \left(\frac{\partial u}{\partial y}\right)^2 \frac{\partial^2 \phi}{\partial u^2}
													+ \left(\frac{\partial v}{\partial y}\right)^2 \frac{\partial^2 \phi}{\partial v^2} 
													+  \left(\frac{\partial u}{\partial y}\right)^2\frac{\partial^2 \phi}{\partial v^2} \\ \\
												   &=& \frac{\partial\phi}{\partial u} \frac{\partial^2 u}{\partial x^2}
													+ \frac{\partial\phi}{\partial v} \frac{\partial^2 v}{\partial y^2} 
													- \frac{\partial\phi}{\partial v} \frac{\partial^2 u}{\partial x \partial y}
													- \frac{\partial\phi}{\partial u} \frac{\partial^2 v}{\partial x \partial y}
													-2 \frac{\partial u}{\partial x} \frac{\partial u}{\partial y} \frac{\partial^2 \phi}{\partial u \partial v}
													-2 \frac{\partial^2 \phi}{\partial u \partial v} \frac{\partial v}{\partial y} \frac{\partial v}{\partial x} \\ \\
												    && +  \left(\frac{\partial u}{\partial x}\right)^2 \frac{\partial^2 \phi}{\partial u^2}
												        	 + \left(\frac{\partial u}{\partial y}\right)^2 \frac{\partial^2 \phi}{\partial u^2}
													 + \left(\frac{\partial v}{\partial y}\right)^2 \frac{\partial^2 \phi}{\partial v^2} 
													 + \left(\frac{\partial u}{\partial y}\right)^2\frac{\partial^2 \phi}{\partial v^2} \\ \\
												&=& \frac{\partial\phi}{\partial u} \frac{\partial^2 u}{\partial x^2}
													+ \frac{\partial\phi}{\partial v} \frac{\partial^2 v}{\partial y^2} 
													- \frac{\partial\phi}{\partial v} \frac{\partial^2 u}{\partial x \partial y}
													- \frac{\partial\phi}{\partial u} \frac{\partial^2 v}{\partial x \partial y}
													-2 \frac{\partial u}{\partial x} \frac{\partial u}{\partial y} \frac{\partial^2 \phi}{\partial u \partial v}
													+2 \frac{\partial^2 \phi}{\partial u \partial v} \frac{\partial u}{\partial x} \frac{\partial u}{\partial y} \\ \\
												    && +  \left(\frac{\partial u}{\partial x}\right)^2 \frac{\partial^2 \phi}{\partial u^2}
												        	 + \left(\frac{\partial u}{\partial y}\right)^2 \frac{\partial^2 \phi}{\partial u^2}
													 + \left(\frac{\partial v}{\partial y}\right)^2 \frac{\partial^2 \phi}{\partial v^2} 
													 + \left(\frac{\partial u}{\partial y}\right)^2\frac{\partial^2 \phi}{\partial v^2} \\ \\
												&=& \frac{\partial\phi}{\partial u} \frac{\partial^2 u}{\partial x^2}
													- \frac{\partial\phi}{\partial u} \frac{\partial^2 v}{\partial x \partial y} 
													+ \frac{\partial\phi}{\partial v} \frac{\partial^2 v}{\partial y^2} 
													- \frac{\partial\phi}{\partial v} \frac{\partial^2 u}{\partial x \partial y}\\ \\
												    && +  \left(\frac{\partial u}{\partial x}\right)^2 \frac{\partial^2 \phi}{\partial u^2}
												        	 + \left(\frac{\partial u}{\partial y}\right)^2 \frac{\partial^2 \phi}{\partial u^2}
													 + \left(\frac{\partial v}{\partial y}\right)^2 \frac{\partial^2 \phi}{\partial v^2} 
													 + \left(\frac{\partial u}{\partial y}\right)^2\frac{\partial^2 \phi}{\partial v^2} \\ \\
												&=& \frac{\partial\phi}{\partial u} \left(\frac{\partial^2 u}{\partial x^2} - \frac{\partial^2 v}{\partial x \partial y} \right)
													+ \frac{\partial\phi}{\partial v} \left(\frac{\partial^2 v}{\partial y^2} - \frac{\partial^2 u}{\partial x \partial y} \right)\\ \\
												    && +  \left(\frac{\partial u}{\partial x}\right)^2 \frac{\partial^2 \phi}{\partial u^2}
												        	 + \left(\frac{\partial u}{\partial y}\right)^2 \frac{\partial^2 \phi}{\partial u^2}
													 + \left(\frac{\partial v}{\partial y}\right)^2 \frac{\partial^2 \phi}{\partial v^2} 
													 + \left(\frac{\partial u}{\partial y}\right)^2\frac{\partial^2 \phi}{\partial v^2} \\ \\
												&=& \frac{\partial\phi}{\partial u} \left(\frac{\partial^2 u}{\partial x^2} - \frac{\partial}{\partial x} \left(\frac{\partial v}{\partial y} \right)\right)
													+ \frac{\partial\phi}{\partial v} \left(\frac{\partial^2 v}{\partial y^2} - \frac{\partial}{\partial y} \left(\frac{\partial u}{\partial x} \right) \right) \\ \\
												    && +  \left(\frac{\partial u}{\partial x}\right)^2 \frac{\partial^2 \phi}{\partial u^2}
												        	 + \left(\frac{\partial u}{\partial y}\right)^2 \frac{\partial^2 \phi}{\partial u^2}
													 + \left(\frac{\partial v}{\partial y}\right)^2 \frac{\partial^2 \phi}{\partial v^2} 
													 + \left(\frac{\partial u}{\partial y}\right)^2\frac{\partial^2 \phi}{\partial v^2} \\ \\
												&=& \frac{\partial\phi}{\partial u} \left(\frac{\partial^2 u}{\partial x^2} - \frac{\partial}{\partial x} \left(\frac{\partial u}{\partial x} \right)\right)
													+ \frac{\partial\phi}{\partial v} \left(\frac{\partial^2 v}{\partial y^2} - \frac{\partial}{\partial y} \left(\frac{\partial v}{\partial y} \right) \right) \\ \\
												    && +  \left(\frac{\partial u}{\partial x}\right)^2 \frac{\partial^2 \phi}{\partial u^2}
												        	 + \left(\frac{\partial u}{\partial y}\right)^2 \frac{\partial^2 \phi}{\partial u^2}
													 + \left(\frac{\partial v}{\partial y}\right)^2 \frac{\partial^2 \phi}{\partial v^2} 
													 + \left(\frac{\partial u}{\partial y}\right)^2\frac{\partial^2 \phi}{\partial v^2} \\ \\
												&=& \frac{\partial\phi}{\partial u} \left(\frac{\partial^2 u}{\partial x^2} - \frac{\partial^2 u}{\partial x^2} \right)
													+ \frac{\partial\phi}{\partial v} \left(\frac{\partial^2 v}{\partial y^2} - \frac{\partial^2 v}{\partial y^2} \right) \\ \\
												    && +  \left(\frac{\partial u}{\partial x}\right)^2 \frac{\partial^2 \phi}{\partial u^2}
												        	 + \left(\frac{\partial u}{\partial y}\right)^2 \frac{\partial^2 \phi}{\partial u^2}
													 + \left(\frac{\partial v}{\partial y}\right)^2 \frac{\partial^2 \phi}{\partial v^2} 
													 + \left(\frac{\partial u}{\partial y}\right)^2\frac{\partial^2 \phi}{\partial v^2} \\ \\
												&=& \left(\frac{\partial u}{\partial x}\right)^2 \frac{\partial^2 \phi}{\partial u^2}
												        	 + \left(\frac{\partial u}{\partial y}\right)^2 \frac{\partial^2 \phi}{\partial u^2}
													 + \left(\frac{\partial v}{\partial y}\right)^2 \frac{\partial^2 \phi}{\partial v^2} 
													 + \left(\frac{\partial u}{\partial y}\right)^2\frac{\partial^2 \phi}{\partial v^2} \\ \\
\end{eqnarray*}
\begin{eqnarray*}
\left(u_x^2 + u_y^2\right) \left(\frac{\partial^2\phi}{\partial u^2} + \frac{\partial^2 \phi }{\partial v^2}\right) &=& u_x^2\frac{\partial^2\phi}{\partial u^2}
	+ u_x^2\frac{\partial^2 \phi }{\partial v^2}
	+ u_y^2\frac{\partial^2\phi}{\partial u^2}
	+ u_y^2\frac{\partial^2 \phi }{\partial v^2} \\ \\
&=& \left(\frac{\partial u}{\partial x}\right)^2 \frac{\partial^2\phi}{\partial u^2}
	+ \left(\frac{\partial u}{\partial x}\right)^2 \frac{\partial^2 \phi }{\partial v^2}
	+ \left(\frac{\partial u}{\partial y}\right)^2 \frac{\partial^2\phi}{\partial u^2}
	+ \left(\frac{\partial u}{\partial y}\right)^2 \frac{\partial^2 \phi }{\partial v^2} \\ \\
\end{eqnarray*}
\qed \\
\noindent
Consequently, we find that if $\phi$ satisfies Laplace's equation $\nabla_{x, y}^2 \phi = 0$ in the domain $D$,
then so long as $f'(z) \neq 0$ in $D$ it also satisfies Laplace's equation $\nabla_{u, v}^2 \phi = 0$ in domain $D'$. \\
\item Show that the derivative of $f(z)=|z|^2$ is defined at $z=0$, but nowhere else.\\
\textit{Solution:} \\
Once again we are going to use the C-R equations and the fact that satisfying them is a necessary condition for differentiability.
\begin{eqnarray*}
f(z) &=&|z|^2 \\
      &=&\sqrt{x^2 + y^2}^2 \\
      &=& x^2 + y^2 \\
      &=& x^2 + y^2 + i \cdot 0
\end{eqnarray*}
Therefore $u(x, y) = x^2 + y^2$ and $v(x, y) = 0$. Now let's calculate the necessary partials
$$u_x = 2x$$ $$v_y = 0$$ $$v_x = 0$$ $$u_y = 2y.$$
Now we need the following to hold for $f(z)$ to be differentiable
\begin{eqnarray*}
u_x &=& v_y \\
2x &=& 0
\end{eqnarray*}
\text{and}
\begin{eqnarray*}
v_x &=& - u_y \\
0 &=& - 2y.
\end{eqnarray*}
Both of these only hold if $x=0$ and $y=0$ or in other words if $z=0$.
Which means the C-R equations only hold at $z=0$ and the derivative of $f(z) = |z|^2$ is defined at $z=0$ but nowhere else. \\
\qed

\item Derive the polar-coordinates form of the Cauchy-Riemann equations
$$
u_r=\frac{1}{r} v_\theta, \quad v_r=-\frac{1}{r} u_\theta.
$$
where $x=r \cos \theta$ and $y=r \sin \theta$ \\
\textit{Solution:} \\
We know we can represent $f(z) = u(x, y) + i v(x, y)$.
Let's make a substitution to polar coordinates
\begin{eqnarray*}
f(z) &=& u(x, y) + i v(x, y) \\
      &=& u(r \cos \theta, r \sin \theta) + i v(r \cos \theta, r \sin \theta) \\
      &=& U(r, \theta) + iV(r, \theta)
\end{eqnarray*}
We need to calculate the derivatives using the chain rule
$$U_r = u_xx_r + u_yy_r$$
$$U_\theta = u_xx_\theta + u_yy_\theta$$
$$V_r = v_xx_r + v_yy_r$$
$$V_\theta = v_xx_\theta + v_yy_\theta.$$

Now we compute the necessary partials and plug them back in
\begin{eqnarray*}
x_r &=& \cos \theta \\
x_\theta &=& - r \sin \theta \\
y_r &=& \sin \theta \\
y_\theta &=& r \cos \theta
\end{eqnarray*}
$$\downarrow$$
\begin{eqnarray*}
U_r &=& u_x \cos \theta + u_y \sin \theta \\
U_\theta &=& - u_x r \sin \theta + u_y r \cos \theta \\
V_r &=& v_x \cos \theta  + v_y  \sin \theta \\
V_\theta &=& - v_x r \sin \theta + v_y r \cos \theta.
\end{eqnarray*}
Now try to make them look like each other with the C-R equations
\begin{eqnarray*}
U_r &=& u_x \cos \theta + u_y \sin \theta \\
       &=& v_y \cos \theta + u_y \sin \theta
\end{eqnarray*}
then
\begin{eqnarray*}
V_\theta &=& - v_x r \sin \theta + v_y r \cos \theta \\
	      &=&  u_y r \sin \theta + v_y r \cos \theta \\
	      &=&  r (u_y \sin \theta + v_y \cos \theta ) \\
	      &=&  r (v_y \cos \theta + u_y \sin \theta) \\
	      &=&  r U_r
\end{eqnarray*}
Therefore $U_r = \frac{1}{r}V_\theta$.
\begin{eqnarray*}
V_r &=& v_x \cos \theta  + v_y  \sin \theta \\
       &=& -u_y \cos \theta  + v_y  \sin \theta
\end{eqnarray*}
then
\begin{eqnarray*}
U_\theta &=& - u_x r \sin \theta + u_y r \cos \theta \\
	      &=& - v_y r \sin \theta + u_y r \cos \theta \\
	      &=& - r (v_y \sin \theta - u_y \cos \theta) \\
	      &=& - r (- u_y \cos \theta + v_y \sin \theta) \\
	      &=& - r V_r
\end{eqnarray*}
Therefore $V_r = - \frac{1}{r}U_\theta$.
And with this we have derived the polar-coordinates form of the Cauchy-Riemann equations. \\
\qed
\end{enumerate}

\end{document}

%%% Local Variables:
%%% mode: latex
%%% TeX-master: t
%%% End:
