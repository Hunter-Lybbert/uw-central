\documentclass[10pt]{amsart}
\usepackage[margin=1.4in]{geometry}
\usepackage[usenames,dvipsnames,cmyk]{xcolor} %load first
\usepackage{cancel}
\usepackage{graphicx,subfig}
\usepackage{mathtools}

\graphicspath{ {./images/} }

\usepackage{amssymb,amsmath,enumitem,url}

\newcommand{\D}{\mathrm{d}}
\newcommand{\I}{\mathrm{i}}
\DeclareMathOperator{\E}{e}
\DeclareMathOperator{\OO}{O}
\DeclareMathOperator{\oo}{o}
\DeclareMathOperator{\erfc}{erfc}
\DeclareMathOperator{\real}{Re}
\DeclareMathOperator{\imag}{Im}
\usepackage{tikz}
\usepackage[framemethod=tikz]{mdframed}
\theoremstyle{nonumberplain}

\mdtheorem[innertopmargin=5pt]{lemma}{Lemma}
\mdtheorem[innertopmargin=-5pt]{sol}{Solution}
%\newmdtheoremenv[innertopmargin=-5pt]{sol}{Solution}
\definecolor{MichiganBlue}{HTML}{00274C}
\definecolor{MichiganYellow}{HTML}{FFCB05}  
\definecolor{NicePurple}{RGB}{75,56,76} %PrincePurple
\definecolor{NiceRed}{RGB}{230,37,52}
\definecolor{MidnightBlue}{rgb}{0.1, 0.1, 0.44}
\usepackage[colorlinks=true, linkcolor=MidnightBlue, citecolor=MidnightBlue, urlcolor=MidnightBlue]{hyperref}

\begin{document}
\pagestyle{empty}

\newcommand{\mline}{\vspace{.2in}\hrule\vspace{.2in}}

\noindent
\text{Hunter Lybbert} \\
\text{Student ID: 2426454} \\
\text{11-04-24} \\
\text{AMATH 567} \\

\title{\bf { Homework 6} }


\maketitle
\noindent
Collaborators*: Nate Ward, Sophia Kamien, Peter Xu, Matthew Watkins, Anja Vogt, Hailey Sparks, Laura Thomas\\
\\
\tiny
\text{*Listed in no particular order. And anyone I discussed at least part of one problem with is considered a collaborator.}
\normalsize


\mline
\begin{enumerate}[label={\bf {\arabic*}:}]
\item  From A\&F: 3.3.2\\
Given the function $$f(z) = \frac z {a^2 - z^2}, \: \: a > 0,$$
expand $f(z)$ in a Laurent series in powers of $z$ in the regions \\

\noindent
(a) $|z| < a$ \\
\textit{Solution:} \\
We begin with a little algebra
$$ f(z) = \frac z {a^2 - z^2} = \frac z {a^2} \frac 1 {1 - \frac{z^2}{a^2}}. $$
In this case, since $|z| < a$, then $\frac {z^2}{a^2} < 1$.
Therefore we can make use of the common geometric series
$$
f(z) = \frac z {a^2} \frac 1 {1 - \frac{z^2}{a^2}}
	= \frac z {a^2} \sum_{n=0}^\infty \left(\frac{z^2}{a^2}\right)^n
	= \frac z {a^2} \sum_{n=0}^\infty \frac{z^{2n}}{a^{2n}}
	= \sum_{n=0}^\infty \frac{z^{2n + 1}}{a^{2n + 2}}
	= \sum_{n=0}^\infty \frac{1}{a^{2n + 2}} z^{2n + 1}.
$$
\qed \\

\noindent
(b) $|z| > a$ \\
\textit{Solution:} \\
We begin with a little algebra
$$ f(z) = \frac z {a^2 - z^2} = -\frac z {z^2 - a^2} = -\frac z {z^2} \frac 1 {1 - \frac{a^2}{z^2}}. $$
In this case, since $|z| > a$, then $\frac {a^2}{z^2} < 1$.
Therefore we can make use of the common geometric series
\begin{align*}
f(z) &= -\frac z {z^2} \frac 1 {1 - \frac{a^2}{z^2}} \\
	&= -\frac 1 {z} \sum_{n=0}^\infty \left(\frac{a^2}{z^2}\right)^n \\
	&= -\frac 1 {z} \sum_{n=0}^\infty \frac{a^{2n}}{z^{2n}} \\
	&= - \sum_{n=0}^\infty \frac{a^{2n}}{z^{2n + 1}} \\
	&= - \sum_{n=0}^\infty a^{2n}\frac{1}{z^{2n + 1}} \\
	&= - \sum_{n=0}^\infty a^{2n}z^{-(2n + 1)} \\
	&= - \sum_{n=0}^\infty a^{2n}z^{-2n -1} \\
	&= - \sum_{n=-\infty}^0 a^{2n}z^{2n -1}.
\end{align*}
\qed \\
\newpage

\item From A\&F: 3.3.5\\
Let
$$\exp\left(\frac t 2 \left( z - \frac1 z \right)\right) = \sum_{n = -\infty}^\infty J_n(t)z^n.$$
Show from the definition of Laurent series and using properties of integration that
\begin{align*}
J_n(t) &= \frac 1 {2 \pi} \int_{-\pi}^\pi \E^{-\I(n \theta - t \sin \theta)}\D \theta \\
	&= \frac 1 {\pi} \int_{0}^\pi \cos(n \theta - t \sin \theta)\D \theta.
\end{align*}
The functions $J_n(t)$ are called the Bessel function, which are well known special functions in mathematics and physics. \\

\noindent
\textit{Solution:} \\
Let $f(z) = \exp\left(\frac t 2 \left(\frac {z - 1} {z}\right)\right)$.
We begin by looking at the general Laurent series centered at $z=0$, since our function is undefined at this point it is the only singularity we are concerned with. Therefore we have
$$f(z) = \sum_{n=-\infty}^{\infty} C_n (z - 0)^n = \sum_{n=-\infty}^{\infty} C_n z^n.$$
Where the $C_n$ is given by
$$
C_n = \frac 1 {2 \pi \I}\oint_C \frac {f(\xi)}{\xi^{n + 1}}\D \xi
	= \frac 1 {2 \pi \I}\oint_C \frac {\exp\left(\frac t 2 \left(\xi - \frac1 \xi\right)\right)}{\xi^{n + 1}}\D \xi.
$$
This is really incomplete notationally since our $C_n$'s depend on $t$ so reverting back to the provided notation we have
$$
J_n(t) = \frac 1 {2 \pi \I}\oint_C \frac {\exp\left(\frac t 2 \left(\xi - \frac 1 \xi \right)\right)}{\xi^{n + 1}}\D \xi.
$$
Additionally, I have yet to specify my contour $C$, but it needs to be within the annulus for which our Laurent series converges.
Since, the original function $f(z)$ only has a singularity at $z=0$ the Laurent series really converges uniformly throughout the complex plane except at the origin.
Therefore we make the convenient choice for our contour $C$ to be a counterclockwise traversal of the unit circle.
Using the parameterization $\xi = \E^{\I \theta} \text{ with }\theta \in [-\pi, \pi)$, we have
\begin{align*}
J_n(t)
&= \frac 1 {2 \pi \I}\int_{-\pi}^{\pi} \frac
	{\exp\left(\frac t 2 \left(\E^{\I \theta} - \frac{1} {\E^{\I \theta}}\right)\right)}
	{\big(\E^{\I \theta}\big)^{n + 1}}
\I\E^{\I \theta} \D \theta \\
&= \frac 1 {2 \pi }\int_{-\pi}^{\pi} \frac
	{\exp\left(\frac t 2 \left(\E^{\I \theta} - \frac{1} {\E^{\I \theta}}\right)\right)}
	{\E^{\I n\theta}}
\D \theta \\
&= \frac 1 {2 \pi }\int_{-\pi}^{\pi} \exp\left(\frac t 2 \left( \E^{\I \theta} - \E^{-\I \theta}\right) - \I n\theta\right) \D \theta \\
&= \frac 1 {2 \pi }\int_{-\pi}^{\pi} \exp\left(
	\frac t 2 \left( \cancel{\cos\theta} + \I \sin \theta - \cancel{\cos\theta} +\I \sin\theta\right) - \I n\theta
\right) \D \theta \\
&= \frac 1 {2 \pi }\int_{-\pi}^{\pi} \exp\left(
	\frac t 2 \left( 2\I \sin \theta\right) - \I n\theta
\right) \D \theta \\
&= \frac 1 {2 \pi }\int_{-\pi}^{\pi} \exp\left( t \I \sin \theta - \I n\theta \right) \D \theta \\
&= \frac 1 {2 \pi }\int_{-\pi}^{\pi} \E^{-\I(n\theta - t \sin \theta)} \D \theta.
\end{align*}
Therefore
$$
J_n(t) = \frac 1 {2 \pi }\int_{-\pi}^{\pi} \E^{-\I(n\theta - t \sin \theta)} \D \theta,
$$
as desired.
Furthermore,
\begin{align*}
J_n(t) &= \frac 1 {2 \pi }\int_{-\pi}^{\pi} \E^{-\I(n\theta - t \sin \theta)} \D \theta \\
	&= \frac 1 {2 \pi }\int_{-\pi}^{\pi} \cos \big(n\theta - t \sin \theta\big) - \I\sin\big( n\theta - t \sin \theta\big) \D \theta \\
	&= \frac 1 {2 \pi }\int_{-\pi}^{0} \cos \big(n\theta - t \sin \theta\big) - \I\sin\big( n\theta - t \sin \theta\big) \D \theta \\
	& \quad\quad + \frac 1 {2 \pi }\int_{0}^{\pi} \cos \big(n\theta - t \sin \theta\big) - \I\sin\big( n\theta - t \sin \theta\big) \D \theta \\
	&= -\frac 1 {2 \pi }\int_{0}^{-\pi} \cos \big(n\theta - t \sin \theta\big) - \I\sin\big( n\theta - t \sin \theta\big) \D \theta \\
	& \quad\quad + \frac 1 {2 \pi }\int_{0}^{\pi} \cos \big(n\theta - t \sin \theta\big) - \I\sin\big( n\theta - t \sin \theta\big) \D \theta.
\end{align*}
Now we need to do a substitution for $n\theta - t \sin \theta$ in each of these integrals.
For the integral from $0$ to $-\pi$ let $\theta = -\theta^\prime$ and for the integral from $0$ to $\pi$ let $\theta = \theta^\prime$.
Continuing where we left off we then have
\begin{align*}
	&= -\frac 1 {2 \pi }\int_{0}^{\pi} \cos \big(-n\theta^\prime - t \sin (-\theta^\prime)\big) - \I\sin\big(-n\theta^\prime - t \sin (-\theta^\prime)\big) (-\D \theta^\prime) \\
	&\quad\quad+ \frac 1 {2 \pi }\int_{0}^{\pi} \cos \big(n\theta^\prime - t \sin \theta^\prime\big) - \I\sin\big( n\theta^\prime - t \sin \theta^\prime\big) \D \theta^\prime \\
	&= \frac 1 {2 \pi }\int_{0}^{\pi} \cos \bigg(-\big(n\theta^\prime - t \sin (\theta^\prime)\big)\bigg) - \I\sin \bigg(-\big(n\theta^\prime - t \sin (\theta^\prime)\big)\bigg) \D \theta^\prime \\
	&\quad\quad+ \frac 1 {2 \pi }\int_{0}^{\pi} \cos \big(n\theta^\prime - t \sin \theta^\prime\big) - \I\sin\big( n\theta^\prime - t \sin \theta^\prime\big) \D \theta^\prime \\
	&= \frac 1 {2 \pi }\int_{0}^{\pi} \cos \big(n\theta^\prime - t \sin (\theta^\prime)\big) + \I\sin \big(n\theta^\prime - t \sin (\theta^\prime)\big) \D \theta^\prime 
	+ \frac 1 {2 \pi }\int_{0}^{\pi} \cos \big(n\theta^\prime - t \sin \theta^\prime\big) - \I\sin\big( n\theta^\prime - t \sin \theta^\prime\big) \D \theta^\prime \\
	&= \frac 1 {2 \pi } \int_{0}^{\pi} \cos \big(n\theta^\prime - t \sin (\theta^\prime)\big) + \cancel{\I\sin \big(n\theta^\prime - t \sin (\theta^\prime)\big)}
	+ \cos \big(n\theta^\prime - t \sin \theta^\prime\big) - \cancel{\I\sin\big( n\theta^\prime - t \sin \theta^\prime\big)} \D \theta^\prime \\
	&= \frac 1 {2 \pi } \int_{0}^{\pi} \cos \big(n\theta^\prime - t \sin (\theta^\prime)\big) + \cos \big(n\theta^\prime - t \sin \theta^\prime\big) \D \theta^\prime \\
	&= \frac 2 {2 \pi } \int_{0}^{\pi} \cos \big(n\theta^\prime - t \sin (\theta^\prime)\big) \D \theta^\prime \\
	&= \frac 1 \pi \int_{0}^{\pi} \cos \big(n\theta^\prime - t \sin (\theta^\prime)\big) \D \theta^\prime.
\end{align*}
Though we finished in terms of another variable $\theta^\prime$ this could easily be changed out with another substitution $\theta^\prime = \theta$.
And thus we see
$$
J_n(t) = \frac 1 {2 \pi }\int_{-\pi}^{\pi} \E^{-\I(n\theta - t \sin \theta)} \D \theta = \frac 1 \pi \int_{0}^{\pi} \cos \big(n\theta - t \sin (\theta)\big) \D \theta.
$$
\qed \\
\newpage

\item Bernoulli numbers: Consider the function
$$
f(z)=\frac{z}{e^z-1} .
$$
\begin{enumerate}
\item Show that $f(z)$ has a removable singularity at $z=0$. Assume from now on that the definition of $f(z)$ has been extended to remove the singularity. \\

 \noindent
\textit{Solution:} \\
If we can show the limit exists at the potential singularity then we can say it is removable.
We can calculate the limit of $f(z)$ as $z\rightarrow 0$ explicitly:
\begin{align*}
\lim_{z \rightarrow 0} f(z) &= \lim_{z \rightarrow 0} \frac z {\E^z - 1} = \frac 0 0 \quad \quad \text{applying L'Hôpitals rule} \\
	&= \lim_{z \rightarrow 0} \frac z {\E^z - 1} = \lim_{z\rightarrow0} \frac 1 {\E^z} = \frac 1 1 = 1
\end{align*}
Therefore, we could choose $f(0) = 1$ in order to extend $f(z)$ to be analytic in the region and therefore remove the singularity.
Furthermore, we can also show this is a removable singularity by looking at the reciprocal of $f(z)$.
If it does not have any zeros, then $f(z)$ will not have any actual singularities or it won't blow up anywhere.
We use a Taylor series centered at $z = 0$ for $\E^z$ and see the following
\begin{align*}
\frac 1 {f(z)} = \frac {\E^z - 1}{z} = \frac 1 z (\E^z - 1) &= \frac 1 z \bigg( \sum_{j=0}^\infty \frac {z^j}{j!} - 1\bigg) \\
	&= \frac 1 z \bigg( \sum_{j=1}^\infty \frac {z^j}{j!}\bigg) \\
	&= \sum_{j=1}^\infty \frac {z^{j - 1}}{j!} \\
	&= \sum_{j=0}^\infty \frac {z^{j}}{(j+1)!}
\end{align*}
which has no zeros.
As an aside I want to highlight that our original function
$$
f(z) = \frac z {\E^z - 1}
$$
blows up at $2\pi \I k$ for $k \in \mathbb Z$.
Since our Taylor series representation of $1/f(z)$ centered at $z = 0$ has no zeros, then we can say the singularity at $z=1$ is removable.
Finally, we can conclude from these two pieces of evidence that this particular singularity is removable.
We will assume from now on that $f(z)$ has been extended to remove the singularity at $z=0$.
\qed \\

\item Suppose you were to find a Taylor series for $f(z)$, centered at $z=0$. What would be its radius of convergence? \\

\noindent
\textit{Solution:} \\
We determined there is no singularity for $f(z)$ at $z=0$, so a Taylor series representation can be constructed centered at $z=0$.
And the radius of convergence will be the distance from $z=0$ to its nearest singularity.
This will be the singularities at $z = -2 \pi \I$ or $z = 2 \pi \I$, therefore the radius of convergence will be $2\pi$.
\qed \\

\item Find the Taylor series in the form
$$
f(z)=\sum_{n=0}^{\infty} \frac{B_n}{n!} z^n .
$$
The numbers $B_n$ are known as the Bernoulli numbers. \\

\noindent
\textit{Solution:} \\
\begin{align*}
f(z) = \frac {z}{\E^z - 1} &= \sum_{n=0}^{\infty} \frac{B_n}{n!} z^n \\
	z &= (\E^z - 1)\sum_{n=0}^{\infty} \frac{B_n}{n!} z^n \\
	z &= \bigg( \sum_{m=0}^\infty \frac {z^m}{m!} - 1\bigg) \bigg( \sum_{n=0}^{\infty} \frac{B_n}{n!} z^n \bigg) \\
	z &= \bigg( \sum_{m=1}^\infty \frac {z^m}{m!}\bigg) \bigg(\sum_{n=0}^{\infty} \frac{B_n}{n!} z^n \bigg) \\
	z &= \bigg( \sum_{m=0}^\infty \frac {z^{m+1}}{(m+1)!}\bigg) \bigg(\sum_{n=0}^{\infty} \frac{B_n}{n!} z^n. \bigg)
\end{align*}
Using the Cauchy Product formula
$$
\bigg( \sum_{i=0}^{\infty} a_i \bigg) \bigg(\sum_{j=0}^{\infty} b_j \bigg) = \sum_{k=0}^\infty \sum_{\ell=0}^k a_\ell b_{k - \ell}
$$
we can continue from where we left off and get
\begin{align*}
z &= \bigg( \sum_{m=0}^\infty \frac {z^{m+1}}{(m+1)!}\bigg) \bigg(\sum_{n=0}^{\infty} \frac{B_n}{n!} z^n\bigg) \\
	&= \sum_{k=0}^\infty \sum_{\ell=0}^k \frac {z^{k - \ell+1}}{(k - \ell+1)!} \frac{B_{\ell} z^\ell}{\ell!} \\
	&= \sum_{k=0}^\infty \sum_{\ell=0}^k \frac 1{(k - \ell+1)!} \frac{B_{\ell}}{\ell!} z^{k+1} \\
	&= \sum_{k=0}^\infty \sum_{\ell=0}^k \frac {(k + 1)!}{(k - \ell+1)!} \frac{B_{\ell}}{\ell!} \frac{z^{k+1}}{(k + 1)!} \\
	&= \sum_{k=0}^\infty \sum_{\ell=0}^k \frac {(k + 1)!}{(k + 1 - \ell)!\ell!} B_{\ell} \frac{z^{k+1}}{(k + 1)!} \\
	&= \sum_{k=0}^\infty \sum_{\ell=0}^k {k + 1 \choose \ell} B_{\ell} \frac{z^{k+1}}{(k + 1)!} \\
	&= \sum_{k=1}^\infty \sum_{\ell=0}^{k-1} {k \choose \ell} B_{\ell} \frac{z^{k}}{k!}.
\end{align*}
Now that we have
\begin{align*}
z &= \sum_{k=1}^\infty \sum_{\ell=0}^{k-1} {k \choose \ell} B_{\ell} \frac{z^{k}}{k!} \\
	&= \sum_{k=1}^\infty \frac{z^{k}}{k!} \sum_{\ell=0}^{k-1} {k \choose \ell} B_{\ell}
\end{align*}
we can see
\begin{align*}
z &= \frac z {1!} {1 \choose 0}B_0 + \sum_{k=2}^\infty \frac{z^{k}}{k!} \sum_{\ell=0}^{k - 1} {k \choose \ell} B_{\ell} \\
z &= zB_0 + \sum_{k=2}^\infty \frac{z^{k}}{k!} \sum_{\ell=0}^{k - 1} {k \choose \ell} B_{\ell}.
\end{align*}
Therefore we need the following to hold
\begin{align*}
B_0 &= 1\\
\sum_{\ell=0}^{k-1} {k \choose \ell} B_{\ell} &= 0, \text{ for $k > 1$}.
\end{align*}
Notice,
\begin{align*}
\sum_{\ell=0}^{k-1} {k \choose \ell} B_{\ell} &= 0 \\
{k \choose {k-1}}B_{k-1} + \sum_{\ell=0}^{k-2} {k \choose \ell} B_{\ell} &= 0 \\
kB_{k-1} &= - \sum_{\ell=0}^{k-2} {k \choose \ell} B_{\ell} \\
B_{k-1} &= - \frac 1 k \sum_{\ell=0}^{k-2} {k \choose \ell} B_{\ell}, \text{ for $k > 1$}.
\end{align*}
To clean this up let's reindex a little
\begin{align*}
B_{k} &= - \frac 1 {k+1} \sum_{\ell=0}^{k-1} {{k+1} \choose \ell} B_{\ell}, \text{ for $k > 0$}.
\end{align*}
Notice this is a recurrence relation for the Bernoulli numbers.
We can combine this with the original series representation of $f(z)$ to get
\begin{align*}
f(z) = \frac {z}{\E^z - 1} &= \sum_{n=0}^{\infty} \frac{B_n}{n!} z^n \\
	&= \sum_{n=0}^{\infty} \frac{z^n}{n!} B_n \\
	&= - \sum_{n=0}^{\infty} \frac{z^n}{n!} \left(\frac 1 {n+1} \sum_{\ell=0}^{n-1} {{n+1} \choose \ell} B_{\ell} \right) \\
	&= - \sum_{n=0}^{\infty} \frac{z^n}{n!}\frac 1 {n+1} \sum_{\ell=0}^{n-1} {{n+1} \choose \ell} B_{\ell}.
\end{align*}
\qed \\
\item Find a recursion formula for the Bernoulli numbers, and use it to find $B_0, \ldots, B_{12}$.
\textit{Solution:}\\
We will use the recurrence relation we found in part (c)
$$
B_{k} = - \frac 1 {k+1} \sum_{\ell=0}^{k-1} {{k+1} \choose \ell} B_{\ell}, \text{ for $k > 0$}
$$
to calculate the first 12 Bernoulli numbers.
We already know that $B_0 = 1$.
Then we have
\begin{align*}
B_1 &= - \frac 1 2 \left( {2 \choose 0}B_0 \right) = - \frac 1 2 \\
B_2 &= - \frac 1 3 \left( {3 \choose 0}B_0 +  {3 \choose 1}B_1 \right)
	= - \frac 1 3 \left( 1 - \frac 3 2 \right)
	= \frac 1 6 \\
B_3 &= - \frac 1 4 \left( {4 \choose 0}B_0 +  {4 \choose 1}B_1 + {4 \choose 2}B_2 \right)
	= - \frac 1 4 \left( 1 - 2 + 1 \right)
	= 0 \\
B_4 &= - \frac 1 5 \left( {5 \choose 0}B_0 +  {5 \choose 1}B_1 + {5 \choose 2}B_2 + {5 \choose 3}B_3 \right) \\
	&= - \frac 1 5 \left( 1 - \frac 5 2 + \frac 5 3 \right)
	= - \frac 1 {30} \\
B_5 &= - \frac 1 6 \left( {6 \choose 0}B_0 +  {6 \choose 1}B_1 + {6 \choose 2}B_2 + {6 \choose 3}B_3 + {6 \choose 4}B_4 \right) \\
	&= - \frac 1 6 \left( 1 - \frac 6 2 + \frac {15} 6 + 0 - \frac {15}{30} \right)
	= 0 \\
B_6 &= - \frac 1 7 \left( {7 \choose 0}B_0 +  {7 \choose 1}B_1 + {7 \choose 2}B_2 + {7 \choose 3}B_3 + {7 \choose 4}B_4 + {7 \choose 5}B_5 \right) \\
	&= - \frac 1 7 \left( 1 - \frac 7 2 + \frac {21} 6 + 0 - \frac {35}{30} + 0 \right)
	= \frac 1 {42}.
\end{align*}
We're half way there!
Continuing on, we have
\begin{align*}
B_7 &= - \frac 1 8 \left(
		{8 \choose 0}B_0
		+  {8 \choose 1}B_1
		+ {8 \choose 2}B_2
		+ {8 \choose 3}B_3
		+ {8 \choose 4}B_4
		+ {8 \choose 5}B_5
		+ {8 \choose 6}B_6
	\right) \\
	&= - \frac 1 8 \left( 1 - \frac 8 2 + \frac {28} 6 + 0 - \frac {70}{30} + 0 + \frac {28} {42} \right)
	= 0 \\
B_8 &= - \frac 1 9 \left(
		{9 \choose 0}B_0
		+  {9 \choose 1}B_1
		+ {9 \choose 2}B_2
		+ {9 \choose 3}B_3
		+ {9 \choose 4}B_4
		+ {9 \choose 5}B_5
		+ {9 \choose 6}B_6
		+ {9 \choose 7}B_7
	\right) \\
	&= - \frac 1 9 \left( 1 - \frac 9 2 + \frac {36} 6 + 0 - \frac {126} {30} + 0 + \frac {84}{42} + 0\right)
	= - \frac 1 {30} \\
B_9 &= - \frac 1 {10} \left(
		{10 \choose 0}B_0
		+  {10 \choose 1}B_1
		+ {10 \choose 2}B_2
		+ {10 \choose 3}B_3
		+ {10 \choose 4}B_4
		+ {10 \choose 5}B_5
		+ {10 \choose 6}B_6
		+ {10 \choose 7}B_7
		+ {10 \choose 8}B_8
	\right) \\
	&= - \frac 1 {10} \left( 1 - \frac {10} 2 + \frac {45} 6 + 0 - \frac {210} {30} + 0 + \frac {210}{42} + 0 - \frac {45}{30} \right)
	= 0 \\
B_{10} &= - \frac 1 {11} \left(
		{11 \choose 0}B_0
		+  {11 \choose 1}B_1
		+ {11 \choose 2}B_2
		+ {11 \choose 3}B_3
		+ {11 \choose 4}B_4 \right. \\
		& \quad\quad\quad\quad\left. + {11 \choose 5}B_5
		+ {11 \choose 6}B_6
		+ {11 \choose 7}B_7
		+ {11 \choose 8}B_8
		+ {11 \choose 9}B_9
	\right) \\
	&= - \frac 1 {11} \left( 1 - \frac {11} 2 + \frac {55} 6 + 0 - \frac {330} {30} + 0 + \frac {462}{42} + 0 - \frac {165}{30} + 0\right)
	= \frac 5 {66} \\
B_{11} &= - \frac 1 {12} \left(
		{12 \choose 0}B_0
		+  {12 \choose 1}B_1
		+ {12 \choose 2}B_2
		+ {12 \choose 3}B_3
		+ {12 \choose 4}B_4 \right. \\
		& \quad\quad\quad\quad\left. + {12 \choose 5}B_5
		+ {12 \choose 6}B_6
		+ {12 \choose 7}B_7
		+ {12 \choose 8}B_8
		+ {12 \choose 9}B_9
		+ {12 \choose 10}B_{10}
	\right) \\
	&= - \frac 1 {12} \left( 1 - \frac {12} 2 + \frac {66} 6 + 0 - \frac {495} {30} + 0 + \frac {924}{42} + 0 - \frac {495}{30} + 0 + \frac {66\cdot5}{66} \right)
	= 0 \\
B_{12} &= - \frac 1 {13} \left(
		{13 \choose 0}B_0
		+  {13 \choose 1}B_1
		+ {13 \choose 2}B_2
		+ {13 \choose 3}B_3
		+ {13 \choose 4}B_4 
		+ {13 \choose 5}B_5\right. \\
		& \quad\quad\quad\quad\left. 
		+ {13 \choose 6}B_6
		+ {13 \choose 7}B_7
		+ {13 \choose 8}B_8
		+ {13 \choose 9}B_9
		+ {13 \choose 10}B_{10}
		+ {13 \choose 11}B_{11}
	\right) \\
	&= - \frac 1 {13} \left( 1 - \frac {13} 2 + \frac {78} 6 + 0 - \frac {715} {30} + 0 + \frac {1716}{42} + 0 - \frac {1287}{30} + 0 + \frac {186\cdot5}{66} + 0\right)
	= -\frac {691}{2730}
\end{align*}
\qed
\\


\item Show that $B_{2 n+1}=0$ for $n \geq 1$. \\
\textit{Solution:} \\
Let's consider our function
$$f(z) = \frac z {\E^z - 1} = \frac {B_0}{0!} + \frac {B_1}{1!}z + \frac {B_2}{2!}z^2 + \frac {B_3}{3!}z^3 + ...$$
Now define
\begin{align*}
g(z) &= f(z) - B_1z \\
	&= \frac z {\E^z - 1} - B_1z.
\end{align*}
If we can show that $g(z)$ is an even function then we know that all of the coefficients of the odd powers of $z$ in the power series expansion of $f(z)$ are 0, therefore implying $B_{2 n+1}=0$ for $n \geq 1$.
Let's check if this criteria is met
\begin{align*}
g(-z) &= f(-z) - B_1(-z) \\
	&= \frac {-z} {\E^{-z} - 1} + B_1z \\
	&= \frac {-z} {\E^{-z} - 1} -\frac z 2 \\
	&= \frac {-2z} {(\E^{-z} - 1)2} -\frac {z (\E^{-z} - 1)} {(\E^{-z} - 1)2} \\
	&= \frac {-2z - z (\E^{-z} - 1)} {(\E^{-z} - 1)2} \\
	&= \frac {-2z - z\E^{-z} + z} {(\E^{-z} - 1)2} \\
	&= \frac {-z\E^{-z} -z} {2(\E^{-z} - 1)} \\
	&= \frac {-z(\E^{-z} +1)} {2(\E^{-z} - 1)} \\
	&= \frac {-z(\E^{-z} +1)} {2(\E^{-z} - 1)} \frac {\E^{z/2}}{\E^{z/2}} \\
	&= \frac {-z(\E^{-z/2} +\E^{z/2})} {2(\E^{-z/2} - \E^{z/2})} \\
	&= \frac {-z(\E^{z/2} +\E^{-z/2})} {-2(\E^{z/2} - \E^{-z/2})} \\
	&= \frac {z(\E^{z} +1)} {2(\E^{z} - 1)} \frac {\E^{-z/2}}{\E^{-z/2}} \\
	&= \frac {z\E^{z} + z} {2(\E^{z} - 1)} \\
	&= \frac {z2 + z\E^{z} - z} {2(\E^{z} - 1)} \\
	&= \frac {z2} {2(\E^{z} - 1)} + \frac {z(\E^{z} - 1)}{2(\E^{z} - 1)} \\
	&= \frac {z} {\E^{z} - 1} - (-\frac 1 2) z \\
	&= \frac {z} {\E^{z} - 1} - B_1 z \\
	&= g(z).
\end{align*}
Therefore, since $g(-z) = g(z)$ then $g(z)$ is an even function and all of the coefficients of the odd power terms in the power series representation of $f(z) = \frac z {\E^z - 1}$ are 0.
Hence, $B_{2 n+1}=0$ for $n \geq 1$.
\qed \\

\item Use your result to find a Taylor series for $z \operatorname{coth} z$, in terms of the Bernoulli numbers. Where is this series valid? Using this result, find a Laurent series for $\cot z$. Where is this series valid?  \\
\textit{Solution:} \\
Notice that the hyperbolic sine and cosine functions are given by
$$ 
\sinh z = \frac{\E^z - \E^{-z}}{2} \quad\text{and}\quad \cosh z =  \frac{\E^z + \E^{-z}}{2}.
$$
Then we can also observe that from part (d) $g(z)$ is 
$$
g(z) = \frac {z(\E^{z/2} +\E^{-z/2})} {2(\E^{z/2} - \E^{-z/2})}
	= \frac {z\frac{\E^{z/2} +\E^{-z/2}}{2}} {2\frac{\E^{z/2} - \E^{-z/2}}{2}}
	= \frac {z}{2} \frac{\cosh \frac z 2} {\sinh \frac z 2}
	= \frac {z}{2} \coth \frac z 2.
$$
Recall the Taylor series expansion of $g(z)$ has no odd powers of $z$ so we can now right it as
$$
\frac {z}{2} \coth \frac z 2 = g(z) = \sum_{n=0}^\infty \frac{B_{2n}}{(2n)!}z^{2n}.
$$
And thus
$$
z \coth z = 2\sum_{n=0}^\infty \frac{B_{2n}}{(2n)!}z^{2n},
$$
which is valid away from the origin.
Dividing both sides by $z$ and substituting in $z\I$ for $z$, we get the expansion of $\cot z$
$$
\cot z = 2\sum_{n=0}^\infty (-1)^n \frac{B_{2n}}{(2n)!}z^{2n - 1}
$$
which is valid for $|z| \leq \pi$.
\end{enumerate}
\newpage

\item Consider $g(z) = 1/f(z)$ where $f(z)$ is as in the previous
  problem.
  \begin{enumerate}
  \item Using the formula for $g(z)$, use software that uses double
    precision floating point arithmetic to compute the errors $e_n:=
    |g(2^{-n}) - g(0)|$ for $n= 1,2,\ldots, 52$.  Produce a plot of
    these errors.
\textit{Solution:} \\
\textbf{TODO: I don't have time} \\

  \item Derive an approximation $G(z)$ to $g(z)$, near $z = 0$, that does not suffer
    from the instability you notice.  Plot the new errors $E_n:=
    |G(2^{-n}) - g(0)|$ for $n= 1,2,\ldots, 52$.  Ensure that these
    errors are less than $10^{-10}$ for all $n$.\\
\textit{Solution:} \\
\textbf{TODO: I don't have time} \\
\end{enumerate}
\newpage

\item Analytic continuation: \\
(a) Consider
$$
F(z)=1+z+z^2+z^3+\ldots=\sum_{n=0}^{\infty} z^n .
$$
Where is this function analytic? \\
\textit{Solution:} \\
Looking at this the ratio test we have
$$
\lim_{n \rightarrow \infty} \left| \frac {z^{n + 1}}{z^n} \right| = \lim_{n \rightarrow \infty}\left| z\right|.
$$
Which implies the Taylor series only converges inside the unit disc or when $|z| < 1$.
Therefore, our series is analytic inside the unit disc.
\qed \\

\noindent
(b) Use the above representation to induce a Taylor representation of $F(z)$ centered at $z=-1 / 2$. Call this representation $G(z)$. Your final result should be of the form
$$
G(z)=\sum_{m=0}^{\infty} c_m\left(z+\frac{1}{2}\right)^m
$$
Where is this series valid? \\
If you can answer this question without
using that both $F(z)$ and $G(z)$ are representations of $1 /(1-z)$,
you will receive 2 bonus points.\\
\textit{Solution:} \\
Follow the path forged in class by doing the following
\begin{align*}
\sum_{n=0}^{\infty}z^n = \sum_{n=0}^{\infty}(z - 0)^n &= \sum_{n=0}^{\infty}\left(z + \frac 1 2 - \frac 1 2 - 0\right)^n \\
	&= \sum_{n=0}^{\infty}\left(\big(z + \frac 1 2\big) + \big(-\frac 1 2\big)\right)^n \\
	&= \sum_{n=0}^{\infty} \sum_{m=0}^n {n \choose m}\bigg(z + \frac 1 2\bigg)^m\bigg(-\frac 1 2\bigg)^{n - m} \\
	&= \sum_{m=0}^{\infty} \sum_{n=m}^\infty {n \choose m}\bigg(z + \frac 1 2\bigg)^m\bigg(-\frac 1 2\bigg)^{n - m} \\
	&= \sum_{m=0}^{\infty}\bigg(z + \frac 1 2\bigg)^m \sum_{n=m}^\infty {n \choose m} \bigg(-\frac 1 2\bigg)^{n - m} \\
	&= \sum_{m=0}^{\infty}\bigg(z + \frac 1 2\bigg)^m c_m.
\end{align*}
Where our coeffecients $c_m$ are
$$
c_m = \sum_{n=m}^\infty {n \choose m} \left(-\frac 1 2\right)^{n - m}.
$$
Now do the ratio test to determine the radius of convergence.
After setting up the basic limit using the typical formula, we will need to immediately perform the following substitutions
$k = n - m$ and $j = n - m - 1$. 
Here we go
\begin{align*}
\lim_{m \rightarrow \infty} \left| \frac {\sum_{n=m+1}^\infty {n \choose m+1} \left(-\frac 1 2\right)^{n - (m + 1)}}{\sum_{n=m}^\infty {n \choose m} \left(-\frac 1 2\right)^{n - m}} \right|
	&= \lim_{m \rightarrow \infty} \left| \frac {\sum_{j=0}^\infty {j + m + 1 \choose m+1} \left(-\frac 1 2\right)^j}{\sum_{k=0}^\infty {k + m \choose m} \left(-\frac 1 2\right)^k} \right| \\
	&= \lim_{m \rightarrow \infty} \left| \frac {\frac 1 {(1 - (-1/2))^{m+2}}}{\frac 1 {(1 - (-1/2))^{m+1}}} \right| \\
	&= \lim_{m \rightarrow \infty} \left| \frac {\frac 1 {(3/2)^{m+2}}}{\frac 1 {(3/2)^{m+1}}} \right| \\
	&= \lim_{m \rightarrow \infty} \left|  \frac 1 {(3/2)^{m+2}} \frac {(3/2)^{m+1}} 1  \right| \\
	&= \lim_{m \rightarrow \infty} \left|  \frac 1 {3/2}  \right| \\
	&= \frac 2 3
\end{align*}
We also utilize the negative binomial series from the second equality to the third.
Therefore our radius of convergence of our new series centered at $z = -\frac 1 2$ is $\frac 3 2$.
Hence our new series is valid inside the disc centered at $z=-\frac 1 2$ with radius $\frac 3 2$.
In other words, it is valid for all $z$ such that
$$
\left| z + \frac 1 2\right| < \frac 3 2.
$$
\newpage

\item This problem is from Whittaker and Watson's "A course of modern
  analysis": Shew\footnote{Aka ``Show''.} that

$$
\sum_{n=1}^{\infty} \frac{z^{n-1}}{\left(1-z^n\right)\left(1-z^{n+1}\right)}= \begin{cases}\frac{1}{(1-z)^2}, & |z|<1 \\ \frac{1}{z(1-z)^2}, & |z|>1 .\end{cases}
$$
This might appear to contradict the idea of analytic
continuation. Please comment.\\
\textit{Solution:} \\
We will begin by using partial fractions on the summand term
\begin{align*}
\frac{z^{n-1}}{\left(1-z^n\right)\left(1-z^{n+1}\right)} &= \frac A {1 - z^n} + \frac B {1 - z^{n + 1}} \\
z^{n-1} &= A(1 - z^{n + 1}) + B(1 - z^n) \\
z^{n-1} &= A - Az^{n + 1} + B -B z^n
\end{align*}
Notice if we plug in $z = 0$ we get $0 = A + B$ implying $A = -B$.
Using these facts we get
\begin{align*}
z^{n-1} &= A - Az^{n + 1} + B -B z^n \\
z^{n-1} &= - Az^{n + 1} - B z^n \\
z^{n-1} &= z^n(- Az - B) \\
\frac 1 z &= - Az - B \\
\frac 1 z &= - Az + A \\
\frac 1 z &= A(1 - z) \\
\frac 1 {z(1 - z)} &= A.
\end{align*}
Therefore, 
$$
A = \frac 1 {z(1 - z)} \quad \text{and} \quad B = -\frac 1 {z(1 - z)}.
$$
Thus we have
\begin{align*}
\sum_{n=1}^{\infty} \frac{z^{n-1}}{\left(1-z^n\right)\left(1-z^{n+1}\right)}
	&= \sum_{n=1}^{\infty} \frac {\frac 1 {z (1 - z)}}{1 - z^n} + \frac{-\frac 1 {z (1 - z)}}{1 - z^{n + 1}} \\
	&= \sum_{n=1}^{\infty} \frac {\frac 1 {z (1 - z)}}{1 - z^n} - \frac{\frac 1 {z (1 - z)}}{1 - z^{n + 1}}.
\end{align*}
Let's look at a finite sum of this form
\begin{align*}
& \sum_{n=1}^{N} \frac {\frac 1 {z (1 - z)}}{1 - z^n} - \frac{\frac 1 {z (1 - z)}}{1 - z^{n + 1}} \\
	& \quad = \frac {\frac 1 {z (1 - z)}}{1 - z} - \cancel{\frac{\frac 1 {z (1 - z)}}{1 - z^2}}
		+ \cancel{\frac {\frac 1 {z (1 - z)}}{1 - z^2}} - \cancel{\frac{\frac 1 {z (1 - z)}}{1 - z^3}}
		+ \cancel{\frac {\frac 1 {z (1 - z)}}{1 - z^3}} - \cancel{\frac{\frac 1 {z (1 - z)}}{1 - z^4}} + ...
		+ \cancel{\frac {\frac 1 {z (1 - z)}}{1 - z^N}} - \frac{\frac 1 {z (1 - z)}}{1 - z^{N + 1}} \\
	& \quad = \frac {\frac 1 {z (1 - z)}}{1 - z} - \frac{\frac 1 {z (1 - z)}}{1 - z^{N + 1}} \\
	& \quad = \frac {1}{z (1 - z)^2} - \frac1{z (1 - z)(1 - z^{N + 1})}
\end{align*}
Now we want to take the limit of this and see the results
\begin{align*}
\sum_{n=1}^{\infty} \frac{z^{n-1}}{\left(1-z^n\right)\left(1-z^{n+1}\right)}
	&= \sum_{n=1}^{\infty} \frac {\frac 1 {z (1 - z)}}{1 - z^n} - \frac{\frac 1 {z (1 - z)}}{1 - z^{n + 1}} \\
	&= \lim_{N\rightarrow \infty} \sum_{n=1}^{N} \frac {\frac 1 {z (1 - z)}}{1 - z^n} - \frac{\frac 1 {z (1 - z)}}{1 - z^{n + 1}} \\
	&= \lim_{N\rightarrow \infty} \frac {1}{z (1 - z)^2} - \frac1{z (1 - z)(1 - z^{N + 1})}
		= \begin{cases} \frac {1}{z (1 - z)^2} - \frac1{z (1 - z)}, & |z|<1 \\ \frac{1}{z(1-z)^2}, & |z|>1 .\end{cases}
\end{align*}
Looking more closely at the resulting expression in the case $|z| < 1$, we have
\begin{align*}
\frac {1}{z (1 - z)^2} - \frac1{z (1 - z)} &= \frac {1}{z (1 - z)^2} - \frac{1 - z}{z (1 - z)^2} \\
	&= \frac {1 - 1 + z}{z (1 - z)^2} \\
	&= \frac {z}{z (1 - z)^2} \\
	&= \frac 1 {(1 - z)^2}.
\end{align*}
Therefore,
$$
\sum_{n=1}^{\infty} \frac{z^{n-1}}{\left(1-z^n\right)\left(1-z^{n+1}\right)}= \begin{cases}\frac{1}{(1-z)^2}, & |z|<1 \\ \frac{1}{z(1-z)^2}, & |z|>1 .\end{cases}
$$
Now this might appear to contradict the idea of analytic continuation, however, the dense collection of singularities on the unit circle does not allow for us to apply analytic continuation here. \\

\noindent
The piecewise functions are analytic in their respective regions where they apply, so it appears like it should be a direct application of analytic continuation.
You may expect there to be some section of the unit circle away from $z = 1$ where each function would agree on a short contour which was included in the closer of each of the functions domains of analyticity.
However, it cannot since there is no continuous contour along the unit circle not broken up by at least one singularity.
This is a result of the following.
Notice, the $n{\rm th}$ term 
$$
\frac{z^{n-1}}{\left(1-z^n\right)\left(1-z^{n+1}\right)}
$$
blows up at the $n$ and $n + 1$ roots of $z$.
Therefore as $n \rightarrow \infty$ the singularities of this nature become dense and eventually cover the entire unit circle in the limit.
Therefore we cannot apply analytic continuation across the boundary between the two disjoint domains for which our two functions (each part of the piecewise function) are analytic. \\
\qed

\newpage

\item Suppose that $f$ is a function satisfying
  \begin{align*}
    |f(x)| \leq M, \quad x \in \mathbb R.
  \end{align*}
  Show that
  \begin{align*}
    \hat f(z) := \int_0^\infty \E^{\I z x} f(x) \D x,
  \end{align*}
  is an analytic function of $z$ for $\imag z > 0$.  You may assume
  that $f$ is continuous, but this is not a necessary assumption.\\
  
\noindent
\textit{Solution:} \\
Use a theorem, something about this being able to hold if the integral is finite, then take the limit as it becomes infinit \\
\textbf{TODO: I don't have time to get to this.}
\newpage

\item Use analytic continuation to show that
  \begin{align*}
    \sqrt{z -1} \sqrt{z + 1} = (z -1) \sqrt{ \frac{ z +1}{z-1}},
  \end{align*}
  where $\sqrt{\cdot}$ denotes the principal branch with $\arg z \in
  [-\pi, \pi)$. \\
  \textit{Solution:} \\
Consider the fact that both of these functions are analytic away from the branch cut along the real axis from $-1$ to $1$.
If we can identify a set of points $z$ for which the left hand side and right hand side agree and this set has an accumulation point in the open set for which they are analytic, then the two functions agree on the whole domain.
This is a form of analytic continuation which we covered in class.
Let's consider the set of points in the sequence $S$
$$
\left(2 + \frac 1 n \right)_{n = 0}^ \infty.
$$
Notice this sequence approximates the accumulation point $2$.
We now can see if the left hand side agrees with the right hand side on this sequence of points
\begin{align*}
\sqrt{z -1} \sqrt{z + 1} &= (z -1) \sqrt{ \frac{ z +1}{z-1}} \\
\lim_{n\rightarrow\infty} \sqrt{2 + \frac 1 n -1} \sqrt{2 + \frac 1 n + 1}
	&= \lim_{n\rightarrow\infty}\left(2 + \frac 1 n -1\right) \sqrt{ \frac{ 2 + \frac 1 n +1}{2 + \frac 1 n-1}} \\
\lim_{n\rightarrow\infty} \sqrt{1} \sqrt{3}
	&= \lim_{n\rightarrow\infty}(1)\sqrt{ \frac{ 3}{1}} \\
\sqrt{3} &= \sqrt{3} \\
\end{align*}
Since they agree in the limit the left hand side and the right hand side agree on this whole set of points in the sequence $S$.
Therefore, by the previously state theorem in class the functions agree in the whole domain for which they are analytic. \\
\qed \\
  
  \noindent
  Then show that
  \begin{align*}
    \sqrt{z -1} \sqrt{z + 1} = z + b_0 + b_1 z^{-1} + b_2  z^{-2} +
    O(z^{-3}), \quad z \to \infty,
  \end{align*}
  and find $b_0,b_1,b_2$. \\
  \textbf{TODO: I don't have time to get to this one.}
  
\end{enumerate}
\end{document}

%%% Local Variables:
%%% mode: latex
%%% TeX-master: t
%%% End:
