\documentclass[10pt]{amsart}
\usepackage[margin=1.4in]{geometry}
\usepackage[usenames,dvipsnames,cmyk]{xcolor} %load first
\usepackage{cancel}
\usepackage{graphicx,subfig}
\usepackage{mathtools}

\graphicspath{ {./images/} }

\usepackage{amssymb,amsmath,enumitem,url}

\newcommand{\D}{\mathrm{d}}
\newcommand{\I}{\mathrm{i}}
\DeclareMathOperator{\E}{e}
\DeclareMathOperator{\OO}{O}
\DeclareMathOperator{\oo}{o}
\DeclareMathOperator{\erfc}{erfc}
\DeclareMathOperator{\real}{Re}
\DeclareMathOperator{\imag}{Im}
\usepackage{tikz}
\usepackage[framemethod=tikz]{mdframed}
\theoremstyle{nonumberplain}

\mdtheorem[innertopmargin=5pt]{lemma}{Lemma}
\mdtheorem[innertopmargin=-5pt]{sol}{Solution}
%\newmdtheoremenv[innertopmargin=-5pt]{sol}{Solution}
\definecolor{MichiganBlue}{HTML}{00274C}
\definecolor{MichiganYellow}{HTML}{FFCB05}  
\definecolor{NicePurple}{RGB}{75,56,76} %PrincePurple
\definecolor{NiceRed}{RGB}{230,37,52}
\definecolor{MidnightBlue}{rgb}{0.1, 0.1, 0.44}
\usepackage[colorlinks=true, linkcolor=MidnightBlue, citecolor=MidnightBlue, urlcolor=MidnightBlue]{hyperref}

\begin{document}
\pagestyle{empty}

\newcommand{\mline}{\vspace{.2in}\hrule\vspace{.2in}}

\noindent
\text{Hunter Lybbert} \\
\text{Student ID: 2426454} \\
\text{11-05-24} \\
\text{AMATH 567} \\

\title{\bf { Homework 6} }


\maketitle
\noindent
Collaborators*: TBD\\
\\
\tiny
\text{*Listed in no particular order. And anyone I discussed at least part of one problem with is considered a collaborator.}
\normalsize


\mline
\begin{enumerate}[label={\bf {\arabic*}:}]
\item  From A\&F: 3.3.2\\
\item From A\&F: 3.3.5\\
\item Bernoulli numbers: Consider the function
$$
f(z)=\frac{z}{e^z-1} .
$$
\begin{enumerate}
\item Show that $f(z)$ has a removable singularity at $z=0$. Assume from now on that the definition of $f(z)$ has been extended to remove the singularity.
\item Suppose you were to find a Taylor series for $f(z)$, centered at $z=0$. What would be its radius of convergence?
\item Find the Taylor series in the form
$$
f(z)=\sum_{n=0}^{\infty} \frac{B_n}{n!} z^n .
$$
The numbers $B_n$ are known as the Bernoulli numbers.
\item Find a recursion formula for the Bernoulli numbers, and use it to find $B_0, \ldots, B_{12}$.
\item Show that $B_{2 n+1}=0$ for $n \geq 1$.
\item Use your result to find a Taylor series for $z \operatorname{coth} z$, in terms of the Bernoulli numbers. Where is this series valid? Using this result, find a Laurent series for $\cot z$. Where is this series valid?    \\
\end{enumerate}

\item Consider $g(z) = 1/f(z)$ where $f(z)$ is as in the previous
  problem.
  \begin{enumerate}
  \item Using the formula for $g(z)$, use software that uses double
    precision floating point arithmetic to compute the errors $e_n:=
    |g(2^{-n}) - g(0)|$ for $n= 1,2,\ldots, 52$.  Produce a plot of
    these errors.
  \item Derive an approximation $G(z)$ to $g(z)$, near $z = 0$, that does not suffer
    from the instability you notice.  Plot the new errors $E_n:=
    |G(2^{-n}) - g(0)|$ for $n= 1,2,\ldots, 52$.  Ensure that these
    errors are less than $10^{-10}$ for all $n$.\\
\end{enumerate}

\item Analytic continuation:
(a) Consider
$$
F(z)=1+z+z^2+z^3+\ldots=\sum_{n=0}^{\infty} z^n .
$$
Where is this function analytic?
(b) Use the above representation to induce a Taylor representation of $F(z)$ centered at $z=-1 / 2$. Call this representation $G(z)$. Your final result should be of the form
$$
G(z)=\sum_{m=0}^{\infty} c_m\left(z+\frac{1}{2}\right)^m
$$
Where is this series valid? If you can answer this question without
using that both $F(z)$ and $G(z)$ are representations of $1 /(1-z)$,
you will receive 2 bonus points.\\

\item This problem is from Whittaker and Watson's "A course of modern
  analysis": Shew\footnote{Aka ``Show''.} that

$$
\sum_{n=1}^{\infty} \frac{z^{n-1}}{\left(1-z^n\right)\left(1-z^{n+1}\right)}= \begin{cases}\frac{1}{(1-z)^2}, & |z|<1 \\ \frac{1}{z(1-z)^2}, & |z|>1 .\end{cases}
$$
This might appear to contradict the idea of analytic
continuation. Please comment.\\

\item Suppose that $f$ is a function satisfying
  \begin{align*}
    |f(x)| \leq M, \quad x \in \mathbb R.
  \end{align*}
  Show that
  \begin{align*}
    \hat f(z) := \int_0^\infty \E^{\I z x} f(x) \D x,
  \end{align*}
  is an analytic function of $z$ for $\imag z > 0$.  You may assume
  that $f$ is continuous, but this is not a necessary assumption.\\

\item Use analytic continuation to show that
  \begin{align*}
    \sqrt{z -1} \sqrt{z + 1} = (z -1) \sqrt{ \frac{ z +1}{z-1}},
  \end{align*}
  where $\sqrt{\cdot}$ denotes the principal branch with $\arg z \in
  [-\pi, \pi)$.  Then show that
  \begin{align*}
    \sqrt{z -1} \sqrt{z + 1} = z + b_0 + b_1 z^{-1} + b_2  z^{-2} +
    O(z^{-3}), \quad z \to \infty,
  \end{align*}
  and find $b_0,b_1,b_2$.
  
\end{enumerate}
\end{document}

%%% Local Variables:
%%% mode: latex
%%% TeX-master: t
%%% End:
