\documentclass[10pt]{amsart}
\usepackage[margin=1.4in]{geometry}
\usepackage[usenames,dvipsnames,cmyk]{xcolor} %load first
\usepackage{cancel}
\usepackage{graphicx,subfig}
\usepackage{mathtools}

\graphicspath{ {./images/} }

\usepackage{amssymb,amsmath,enumitem,url}

\newcommand{\D}{\mathrm{d}}
\newcommand{\I}{\mathrm{i}}
\DeclareMathOperator{\E}{e}
\DeclareMathOperator{\OO}{O}
\DeclareMathOperator{\oo}{o}
\DeclareMathOperator{\erfc}{erfc}
\DeclareMathOperator{\real}{Re}
\DeclareMathOperator{\imag}{Im}
\usepackage{tikz}
\usepackage[framemethod=tikz]{mdframed}
\theoremstyle{nonumberplain}

\mdtheorem[innertopmargin=5pt]{lemma}{Lemma}
\mdtheorem[innertopmargin=-5pt]{sol}{Solution}
%\newmdtheoremenv[innertopmargin=-5pt]{sol}{Solution}
\definecolor{MichiganBlue}{HTML}{00274C}
\definecolor{MichiganYellow}{HTML}{FFCB05}  
\definecolor{NicePurple}{RGB}{75,56,76} %PrincePurple
\definecolor{NiceRed}{RGB}{230,37,52}
\definecolor{MidnightBlue}{rgb}{0.1, 0.1, 0.44}
\usepackage[colorlinks=true, linkcolor=MidnightBlue, citecolor=MidnightBlue, urlcolor=MidnightBlue]{hyperref}

\begin{document}
\pagestyle{empty}

\newcommand{\mline}{\vspace{.2in}\hrule\vspace{.2in}}

\noindent
\text{Hunter Lybbert} \\
\text{Student ID: 2426454} \\
\text{11-04-24} \\
\text{AMATH 567} \\

\title{\bf { Homework 6} }


\maketitle
\noindent
Collaborators*: Nate Ward, Sophie, Peter\\
\\
\tiny
\text{*Listed in no particular order. And anyone I discussed at least part of one problem with is considered a collaborator.}
\normalsize


\mline
\begin{enumerate}[label={\bf {\arabic*}:}]
\item  From A\&F: 3.3.2\\
Given the function $$f(z) = \frac z {a^2 - z^2}, \: \: a > 0,$$
expand $f(z)$ in a Laurent series in powers of $z$ in the regions \\

\noindent
(a) $|z| < a$ \\
\textit{Solution:} \\
We begin with a little algebra
$$ f(z) = \frac z {a^2 - z^2} = \frac z {a^2} \frac 1 {1 - \frac{z^2}{a^2}}. $$
In this case, since $|z| < a$, then $\frac {z^2}{a^2} < 1$.
Therefore we can make use of the common geometric series
$$
f(z) = \frac z {a^2} \frac 1 {1 - \frac{z^2}{a^2}}
	= \frac z {a^2} \sum_{n=0}^\infty \left(\frac{z^2}{a^2}\right)^n
	= \frac z {a^2} \sum_{n=0}^\infty \frac{z^{2n}}{a^{2n}}
	= \sum_{n=0}^\infty \frac{z^{2n + 1}}{a^{2n + 2}}
	= \sum_{n=0}^\infty \frac{1}{a^{2n + 2}} z^{2n + 1}.
$$
\qed \\

\noindent
(b) $|z| > a$ \\
\textit{Solution:} \\
We begin with a little algebra
$$ f(z) = \frac z {a^2 - z^2} = -\frac z {z^2 - a^2} = -\frac z {z^2} \frac 1 {1 - \frac{a^2}{z^2}}. $$
In this case, since $|z| > a$, then $\frac {a^2}{z^2} < 1$.
Therefore we can make use of the common geometric series
\begin{align*}
f(z) &= -\frac z {z^2} \frac 1 {1 - \frac{a^2}{z^2}} \\
	&= -\frac 1 {z} \sum_{n=0}^\infty \left(\frac{a^2}{z^2}\right)^n \\
	&= -\frac 1 {z} \sum_{n=0}^\infty \frac{a^{2n}}{z^{2n}} \\
	&= - \sum_{n=0}^\infty \frac{a^{2n}}{z^{2n + 1}} \\
	&= - \sum_{n=0}^\infty a^{2n}\frac{1}{z^{2n + 1}} \\
	&= - \sum_{n=0}^\infty a^{2n}z^{-(2n + 1)} \\
	&= - \sum_{n=0}^\infty a^{2n}z^{-2n -1} \\
	&= - \sum_{n=-\infty}^0 a^{2n}z^{2n -1}.
\end{align*}
\qed \\
\newpage

\item From A\&F: 3.3.5\\
Let
$$\exp\left(\frac t 2 \left( z - \frac1 z \right)\right) = \sum_{n = -\infty}^\infty J_n(t)z^n.$$
Show from the definition of Laurent series and using properties of integration that
\begin{align*}
J_n(t) &= \frac 1 {2 \pi} \int_{-\pi}^\pi \E^{-\I(n \theta - t \sin \theta)}\D \theta \\
	&= \frac 1 {\pi} \int_{0}^\pi \cos(n \theta - t \sin \theta)\D \theta.
\end{align*}
The functions $J_n(t)$ are called the Bessel function, which are well known special functions in mathematics and physics. \\

\noindent
\textit{Solution:} \\
Let $f(z) = \exp\left(\frac t 2 \left(\frac {z - 1} {z}\right)\right)$.
We begin by looking at the general Laurent series centered at $z=0$, since our function is undefined at this point it is the only singularity we are concerned with. Therefore we have
$$f(z) = \sum_{n=-\infty}^{\infty} C_n (z - 0)^n = \sum_{n=-\infty}^{\infty} C_n z^n.$$
Where the $C_n$ is given by
$$
C_n = \frac 1 {2 \pi \I}\oint_C \frac {f(\xi)}{\xi^{n + 1}}\D \xi
	= \frac 1 {2 \pi \I}\oint_C \frac {\exp\left(\frac t 2 \left(\xi - \frac1 \xi\right)\right)}{\xi^{n + 1}}\D \xi.
$$
This is really incomplete notationally since our $C_n$'s depend on $t$ so reverting back to the provided notation we have
$$
J_n(t) = \frac 1 {2 \pi \I}\oint_C \frac {\exp\left(\frac t 2 \left(\xi - \frac 1 \xi \right)\right)}{\xi^{n + 1}}\D \xi.
$$
Additionally, I have yet to specify my contour $C$, but it needs to be within the annulus for which our Laurent series converges.
Since, the original function $f(z)$ only has a singularity at $z=0$ the Laurent series really converges uniformly throughout the complex plane except at the origin.
Therefore we make the convenient choice for our contour $C$ to be a counterclockwise traversal of the unit circle.
Using the parameterization $\xi = \E^{\I \theta} \text{ with }\theta \in [-\pi, \pi)$, we have
\begin{align*}
J_n(t)
&= \frac 1 {2 \pi \I}\int_{-\pi}^{\pi} \frac
	{\exp\left(\frac t 2 \left(\E^{\I \theta} - \frac{1} {\E^{\I \theta}}\right)\right)}
	{\big(\E^{\I \theta}\big)^{n + 1}}
\I\E^{\I \theta} \D \theta \\
&= \frac 1 {2 \pi }\int_{-\pi}^{\pi} \frac
	{\exp\left(\frac t 2 \left(\E^{\I \theta} - \frac{1} {\E^{\I \theta}}\right)\right)}
	{\E^{\I n\theta}}
\D \theta \\
&= \frac 1 {2 \pi }\int_{-\pi}^{\pi} \exp\left(\frac t 2 \left( \E^{\I \theta} - \E^{-\I \theta}\right) - \I n\theta\right) \D \theta \\
&= \frac 1 {2 \pi }\int_{-\pi}^{\pi} \exp\left(
	\frac t 2 \left( \cancel{\cos\theta} + \I \sin \theta - \cancel{\cos\theta} +\I \sin\theta\right) - \I n\theta
\right) \D \theta \\
&= \frac 1 {2 \pi }\int_{-\pi}^{\pi} \exp\left(
	\frac t 2 \left( 2\I \sin \theta\right) - \I n\theta
\right) \D \theta \\
&= \frac 1 {2 \pi }\int_{-\pi}^{\pi} \exp\left( t \I \sin \theta - \I n\theta \right) \D \theta \\
&= \frac 1 {2 \pi }\int_{-\pi}^{\pi} \E^{-\I(n\theta - t \sin \theta)} \D \theta.
\end{align*}
Therefore
$$
J_n(t) = \frac 1 {2 \pi }\int_{-\pi}^{\pi} \E^{-\I(n\theta - t \sin \theta)} \D \theta,
$$
as desired.
Furthermore,
\begin{align*}
J_n(t) &= \frac 1 {2 \pi }\int_{-\pi}^{\pi} \E^{-\I(n\theta - t \sin \theta)} \D \theta \\
	&= \frac 1 {2 \pi }\int_{-\pi}^{\pi} \cos \big(n\theta - t \sin \theta\big) - \I\sin\big( n\theta - t \sin \theta\big) \D \theta \\
	&= \frac 1 {2 \pi }\int_{-\pi}^{0} \cos \big(n\theta - t \sin \theta\big) - \I\sin\big( n\theta - t \sin \theta\big) \D \theta \\
	& \quad\quad + \frac 1 {2 \pi }\int_{0}^{\pi} \cos \big(n\theta - t \sin \theta\big) - \I\sin\big( n\theta - t \sin \theta\big) \D \theta \\
	&= -\frac 1 {2 \pi }\int_{0}^{-\pi} \cos \big(n\theta - t \sin \theta\big) - \I\sin\big( n\theta - t \sin \theta\big) \D \theta \\
	& \quad\quad + \frac 1 {2 \pi }\int_{0}^{\pi} \cos \big(n\theta - t \sin \theta\big) - \I\sin\big( n\theta - t \sin \theta\big) \D \theta.
\end{align*}
Now we need to do a substitution for $n\theta - t \sin \theta$ in each of these integrals.
For the integral from $0$ to $-\pi$ let $\theta = -\theta^\prime$ and for the integral from $0$ to $\pi$ let $\theta = \theta^\prime$.
Continuing where we left off we then have
\begin{align*}
	&= -\frac 1 {2 \pi }\int_{0}^{\pi} \cos \big(-n\theta^\prime - t \sin (-\theta^\prime)\big) - \I\sin\big(-n\theta^\prime - t \sin (-\theta^\prime)\big) (-\D \theta^\prime) \\
	&\quad\quad+ \frac 1 {2 \pi }\int_{0}^{\pi} \cos \big(n\theta^\prime - t \sin \theta^\prime\big) - \I\sin\big( n\theta^\prime - t \sin \theta^\prime\big) \D \theta^\prime \\
	&= \frac 1 {2 \pi }\int_{0}^{\pi} \cos \bigg(-\big(n\theta^\prime - t \sin (\theta^\prime)\big)\bigg) - \I\sin \bigg(-\big(n\theta^\prime - t \sin (\theta^\prime)\big)\bigg) \D \theta^\prime \\
	&\quad\quad+ \frac 1 {2 \pi }\int_{0}^{\pi} \cos \big(n\theta^\prime - t \sin \theta^\prime\big) - \I\sin\big( n\theta^\prime - t \sin \theta^\prime\big) \D \theta^\prime \\
	&= \frac 1 {2 \pi }\int_{0}^{\pi} \cos \big(n\theta^\prime - t \sin (\theta^\prime)\big) + \I\sin \big(n\theta^\prime - t \sin (\theta^\prime)\big) \D \theta^\prime 
	+ \frac 1 {2 \pi }\int_{0}^{\pi} \cos \big(n\theta^\prime - t \sin \theta^\prime\big) - \I\sin\big( n\theta^\prime - t \sin \theta^\prime\big) \D \theta^\prime \\
	&= \frac 1 {2 \pi } \int_{0}^{\pi} \cos \big(n\theta^\prime - t \sin (\theta^\prime)\big) + \cancel{\I\sin \big(n\theta^\prime - t \sin (\theta^\prime)\big)}
	+ \cos \big(n\theta^\prime - t \sin \theta^\prime\big) - \cancel{\I\sin\big( n\theta^\prime - t \sin \theta^\prime\big)} \D \theta^\prime \\
	&= \frac 1 {2 \pi } \int_{0}^{\pi} \cos \big(n\theta^\prime - t \sin (\theta^\prime)\big) + \cos \big(n\theta^\prime - t \sin \theta^\prime\big) \D \theta^\prime \\
	&= \frac 2 {2 \pi } \int_{0}^{\pi} \cos \big(n\theta^\prime - t \sin (\theta^\prime)\big) \D \theta^\prime \\
	&= \frac 1 \pi \int_{0}^{\pi} \cos \big(n\theta^\prime - t \sin (\theta^\prime)\big) \D \theta^\prime.
\end{align*}
Though we finished in terms of another variable $\theta^\prime$ this could easily be changed out with another substitution $\theta^\prime = \theta$.
And thus we see
$$
J_n(t) = \frac 1 {2 \pi }\int_{-\pi}^{\pi} \E^{-\I(n\theta - t \sin \theta)} \D \theta = \frac 1 \pi \int_{0}^{\pi} \cos \big(n\theta - t \sin (\theta)\big) \D \theta.
$$
\qed \\
\newpage

\item Bernoulli numbers: Consider the function
$$
f(z)=\frac{z}{e^z-1} .
$$
\begin{enumerate}
\item Show that $f(z)$ has a removable singularity at $z=0$. Assume from now on that the definition of $f(z)$ has been extended to remove the singularity. \\

 \noindent
 \textit{Solution:} \\
We are going to site a lot of the same logic as we are using from problem 3 last week where we were talking about a removable singularity.
In this case, however, notice we can calculate explicitly the limit of $f(z)$ as $z\rightarrow 0$ as follows:
\begin{align*}
\lim_{z \rightarrow 0} f(z) &= \lim_{z \rightarrow 0} \frac z {\E^z - 1} = \frac 0 0 \quad \quad \text{applying L'Hôpitals rule} \\
	&= \lim_{z \rightarrow 0} \frac z {\E^z - 1} = \lim_{z\rightarrow0} \frac 1 {\E^z} = \frac 1 1 = 1
\end{align*}
Therefore, we need to choose $f(0) = 1$ in order for $f(z)$ to be analytic in the region and therefore remove the singularity.
\textbf{TODO: connect this with problem 3 from last weeks hw.}
\qed \\
\item Suppose you were to find a Taylor series for $f(z)$, centered at $z=0$. What would be its radius of convergence?
\item Find the Taylor series in the form
$$
f(z)=\sum_{n=0}^{\infty} \frac{B_n}{n!} z^n .
$$
The numbers $B_n$ are known as the Bernoulli numbers.
\item Find a recursion formula for the Bernoulli numbers, and use it to find $B_0, \ldots, B_{12}$.
\textit{Solution:}\\
put things in terms of taylor series and move them over to the left side of the equation
\item Show that $B_{2 n+1}=0$ for $n \geq 1$.
\item Use your result to find a Taylor series for $z \operatorname{coth} z$, in terms of the Bernoulli numbers. Where is this series valid? Using this result, find a Laurent series for $\cot z$. Where is this series valid?    \\
\end{enumerate}
\newpage

\item Consider $g(z) = 1/f(z)$ where $f(z)$ is as in the previous
  problem.
  \begin{enumerate}
  \item Using the formula for $g(z)$, use software that uses double
    precision floating point arithmetic to compute the errors $e_n:=
    |g(2^{-n}) - g(0)|$ for $n= 1,2,\ldots, 52$.  Produce a plot of
    these errors.
  \item Derive an approximation $G(z)$ to $g(z)$, near $z = 0$, that does not suffer
    from the instability you notice.  Plot the new errors $E_n:=
    |G(2^{-n}) - g(0)|$ for $n= 1,2,\ldots, 52$.  Ensure that these
    errors are less than $10^{-10}$ for all $n$.\\
\end{enumerate}
\newpage

\item Analytic continuation: \\
(a) Consider
$$
F(z)=1+z+z^2+z^3+\ldots=\sum_{n=0}^{\infty} z^n .
$$
Where is this function analytic? \\
(b) Use the above representation to induce a Taylor representation of $F(z)$ centered at $z=-1 / 2$. Call this representation $G(z)$. Your final result should be of the form
$$
G(z)=\sum_{m=0}^{\infty} c_m\left(z+\frac{1}{2}\right)^m
$$
Where is this series valid? \\
If you can answer this question without
using that both $F(z)$ and $G(z)$ are representations of $1 /(1-z)$,
you will receive 2 bonus points.\\
\textit{Solution:}
expansion of the same function allows you justify things and compute the radius of convergence a certain way. \\

\noindent
Use the ratio test for a tedious 2 bonus points.

\newpage

\item This problem is from Whittaker and Watson's "A course of modern
  analysis": Shew\footnote{Aka ``Show''.} that

$$
\sum_{n=1}^{\infty} \frac{z^{n-1}}{\left(1-z^n\right)\left(1-z^{n+1}\right)}= \begin{cases}\frac{1}{(1-z)^2}, & |z|<1 \\ \frac{1}{z(1-z)^2}, & |z|>1 .\end{cases}
$$
This might appear to contradict the idea of analytic
continuation. Please comment.\\
\textit{Solution:} \\
Start from the right for the first case you can use the geometric series and multiply them with eachother.
\newpage

\item Suppose that $f$ is a function satisfying
  \begin{align*}
    |f(x)| \leq M, \quad x \in \mathbb R.
  \end{align*}
  Show that
  \begin{align*}
    \hat f(z) := \int_0^\infty \E^{\I z x} f(x) \D x,
  \end{align*}
  is an analytic function of $z$ for $\imag z > 0$.  You may assume
  that $f$ is continuous, but this is not a necessary assumption.\\
\newpage

\item Use analytic continuation to show that
  \begin{align*}
    \sqrt{z -1} \sqrt{z + 1} = (z -1) \sqrt{ \frac{ z +1}{z-1}},
  \end{align*}
  where $\sqrt{\cdot}$ denotes the principal branch with $\arg z \in
  [-\pi, \pi)$. \\
  \textit{Solution:} \\
  Consider that they are both analytic everywhere in the same domain (use the form of analytic continuation which depends on the accumulation point) \\
  Choose a contour for which the functions. agree on (positive real axis is a good choice). \\
  
  \noindent
  Then show that
  \begin{align*}
    \sqrt{z -1} \sqrt{z + 1} = z + b_0 + b_1 z^{-1} + b_2  z^{-2} +
    O(z^{-3}), \quad z \to \infty,
  \end{align*}
  and find $b_0,b_1,b_2$.
  
\end{enumerate}
\end{document}

%%% Local Variables:
%%% mode: latex
%%% TeX-master: t
%%% End:
