\documentclass[10pt]{amsart}
\usepackage[margin=1.4in]{geometry}
\usepackage[usenames,dvipsnames,cmyk]{xcolor} %load first
\usepackage{cancel}
\usepackage{graphicx,subfig}
\usepackage{mathtools}

\graphicspath{ {./images/} }

\usepackage{amssymb,amsmath,enumitem,url}

\newcommand{\D}{\mathrm{d}}
\newcommand{\I}{\mathrm{i}}
\DeclareMathOperator{\E}{e}
\DeclareMathOperator{\OO}{O}
\DeclareMathOperator{\oo}{o}
\DeclareMathOperator{\erfc}{erfc}
\DeclareMathOperator{\real}{Re}
\DeclareMathOperator{\imag}{Im}
\usepackage{tikz}
\usepackage[framemethod=tikz]{mdframed}
\theoremstyle{nonumberplain}

\mdtheorem[innertopmargin=5pt]{lemma}{Lemma}
\mdtheorem[innertopmargin=-5pt]{sol}{Solution}
%\newmdtheoremenv[innertopmargin=-5pt]{sol}{Solution}
\definecolor{MichiganBlue}{HTML}{00274C}
\definecolor{MichiganYellow}{HTML}{FFCB05}  
\definecolor{NicePurple}{RGB}{75,56,76} %PrincePurple
\definecolor{NiceRed}{RGB}{230,37,52}
\definecolor{MidnightBlue}{rgb}{0.1, 0.1, 0.44}
\usepackage[colorlinks=true, linkcolor=MidnightBlue, citecolor=MidnightBlue, urlcolor=MidnightBlue]{hyperref}

\begin{document}
\pagestyle{empty}

\newcommand{\mline}{\vspace{.2in}\hrule\vspace{.2in}}

\noindent
\text{Hunter Lybbert} \\
\text{Student ID: 2426454} \\
\text{12-02-24} \\
\text{AMATH 567} \\

\title{\bf { Homework 10} }


\maketitle
\noindent
Collaborators*: TBD \\
\\
\tiny
\text{*Listed in no particular order. And anyone I discussed at least part of one problem with is considered a collaborator.}
\normalsize


\mline
\begin{enumerate}[label={\bf {\arabic*}:}]
\item I sketched the following in class.  Complete the argument.  Show that for an integer $j \in (-N,N)$ and $h > 0$,
\begin{align*}
\lim_{h \to \infty} \int_{\I h}^{\I h + \pi} \frac{\E^{2 \I jz}}{\tan (N z)} \D z
	= \begin{cases} - \I \pi & j = 0,\\ 0 & \text{otherwise}.\end{cases}
\end{align*}
 
\noindent
\textit{Solution:} \\
\textbf{TODO:} \\
Following the sketch provided in class let's look at an import representation of $\tan (Nz)$
\begin{align*}
\tan(N z) &= \frac {\sin (N z)}{\cos (N z)} \\
	&= \frac{ \E^{\I N z} + \E^{- \I N z} }{2\I} \left( \frac {\E^{\I N z} - \E^{- \I N z}} 2 \right)^{-1} \\
	&= \frac{ \E^{\I N z} + \E^{- \I N z} }{2\I} \left( \frac 2 {\E^{\I N z} - \E^{- \I N z}} \right) \\
	&= \frac 1 \I \left( \frac { \E^{\I N z} + \E^{- \I N z} }{ \E^{\I N z} - \E^{- \I N z} } \right) \\
	&= \frac 1 \I \Bigg( \frac {\E^{\I N z}}{\E^{\I N z}} \left( \frac { 1 + \E^{- 2 \I N z} }{ 1 - \E^{- 2 \I N z} } \right) \Bigg) \\
	&= \frac 1 \I \Bigg( \frac { 1 + \E^{- 2 \I N z} }{ 1 - \E^{- 2 \I N z} } \Bigg) \\
	&= \frac 1 \I \Bigg( \frac { 1 + \left( \cos (N z) - \I \sin(N z) \right)^2 }{ 1 - \left( \cos (N z) - \I \sin(N z) \right)^2 } \Bigg) \\
	&= \frac 1 \I \Bigg( \frac { 1 +  \cos^2 (N z) - 2 \I \cos(N z) \sin(N z)- \sin^2(N z) }{ 1 - \cos^2 (N z) + 2 \I \cos(N z) \sin(N z) + \sin^2(N z) } \Bigg) \\
	&... \\
	&= \I + \mathcal O(\E^{-2Nh})
\end{align*}
\newpage

\item From A\&F: 4.2.1 (b) \\
 
\noindent
\textit{Solution:} \\
See solution to problem 2 from homework set 9. \\
\qed \\


\item From A\&F: 4.2.2 (a, h)\\
 
\noindent
\textit{Solution:} \\
\textbf{TODO:} \\
\begin{align*}
f(x) = mx + b
\end{align*}


\newpage

\item
  \begin{enumerate}
\item  Show that
$$
\operatorname{Res}_{z=k} f(z) \cot (\pi z)=\frac{1}{\pi} f(k),
$$
provided $f(z)$ is analytic at $z=k, ~~k \in \mathbb{Z}$. \\
 
\noindent
\textit{Solution:} \\
Recall from homework 6 problem 3 we derived a series representation for $\cot z$
\begin{align*}
\cot(\pi z) &= \sum_{n = 0}^\infty (-1)^n \frac {2^{2n}B_{2n}}{(2n)!} \pi^{2n - 1} z^{2n - 1} \\
	&= (-1)^0 \frac {2^{0}B_{0}}{0!} \pi^{0 - 1} z^{0 - 1} + \sum_{n = 1}^\infty (-1)^n \frac {2^{2n}B_{2n}}{(2n)!} \pi^{2n - 1} z^{2n - 1} \\
	&= \frac {B_0} {z\pi} + \sum_{n = 1}^\infty (-1)^n \frac {2^{2n}B_{2n}}{(2n)!} \pi^{2n - 1} z^{2n - 1}.
\end{align*}
Therefore, we can compute the residue by multiplying $f(z)$ through this Taylor series and evaluating the expression in the numerator of the simple pole at $z = 0$
\begin{align*}
\operatorname{Res}_{z=k} f(z) \cot (\pi z)
	&= \operatorname{Res}_{z=k} \left[ f(z) \left( \frac {B_0} {z\pi} + \sum_{n = 1}^\infty (-1)^n \frac {2^{2n}B_{2n}}{(2n)!} \pi^{2n - 1} z^{2n - 1}\right) \right] \\
	&= \operatorname{Res}_{z=k} \left[  \frac {B_0 f(z)} {z\pi} + f(z)\sum_{n = 1}^\infty (-1)^n \frac {2^{2n}B_{2n}}{(2n)!} \pi^{2n - 1} z^{2n - 1} \right] \\
	&= \frac{B_0}{\pi} f(k) \\
	&= \frac 1 {\pi} f(k).
\end{align*}
We can conclude this because there is no irregularities contributed by the analytic function $f(z)$. \\
\qed \\

\item Let $\Gamma_N$ be a square contour, with corners at $(N+1 / 2)( \pm 1 \pm i), N \in \mathbb{Z}^{+}$. Show that
$$
|\cot (\pi z)| \leq 2,
$$
for $z$ on $\Gamma_N$.\\
 
\noindent
\textit{Solution:} \\
\textbf{TODO:}
\begin{align*}
f(x) = mx + b
\end{align*}


\item Suppose $f(z)=p(z) / q(z)$, where $p(z)$ and $q(z)$ are polynomials, so that the degree of $q(z)$ is at least two more than the degree of $p(z)$. Show that
$$
\lim _{N \rightarrow \infty}\left|\oint_{\Gamma_N} \frac{p(z)}{q(z)} \cot (\pi z) d z\right|=0
$$
 
\noindent
\textit{Solution:} \\
\textbf{TODO:}
\begin{align*}
f(x) = mx + b
\end{align*}


\item Suppose, in addition, that $q(z)$ has no roots at the integers. Show that
$$
\sum_{k=-\infty}^{\infty} \frac{p(k)}{q(k)}=-\pi \sum_j \operatorname{Res}_{z=z_j} f(z) \cot (\pi z)
$$
where the $z_j$'s are the roots of $q(z)$. Notice that the sum on the
right-hand side has a finite number of terms. \\
 
\noindent
\textit{Solution:} \\
\textbf{TODO:}
\begin{align*}
f(x) = mx + b
\end{align*}

\newpage

\item Use the result of the previous problem to evaluate the following sums:
\begin{enumerate}
\item  $\displaystyle \sum_{k=-\infty}^{\infty} \frac{1}{k^2+1}$ \\
 
\noindent
\textit{Solution:} \\
\textbf{TODO:}
\begin{align*}
f(x) = mx + b
\end{align*}

\item  $\displaystyle \sum_{k=-\infty}^{\infty} \frac{1}{k^4+1}$ \\
 
\noindent
\textit{Solution:} \\
\textbf{TODO:}
\begin{align*}
f(x) = mx + b
\end{align*}


\item  $\displaystyle \sum_{k=-\infty}^{\infty} \frac{1}{k^2-1/4}$ \\
 
\noindent
\textit{Solution:} \\
\textbf{TODO:}
\begin{align*}
f(x) = mx + b
\end{align*}


\item  $\displaystyle \sum_{k=-\infty}^{\infty} \frac{1}{16k^4 -1}$ \\
 
\noindent
\textit{Solution:} \\
\textbf{TODO:}
\begin{align*}
f(x) = mx + b
\end{align*}


\end{enumerate}
\end{enumerate}

\end{enumerate}
  
\end{document}

%%% Local Variables:
%%% mode: latex
%%% TeX-master: t
%%% End:
