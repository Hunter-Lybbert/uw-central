\documentclass[10pt]{amsart}
\usepackage[margin=1.4in]{geometry}
\usepackage{amssymb,amsmath,enumitem,url}

\newcommand{\D}{\mathrm{d}}
\newcommand{\I}{\mathrm{i}}
\DeclareMathOperator{\E}{e}
\DeclareMathOperator{\OO}{O}
\DeclareMathOperator{\oo}{o}
\DeclareMathOperator{\erfc}{erfc}
\DeclareMathOperator{\real}{Re}
\DeclareMathOperator{\imag}{Im}
\usepackage{tikz}
\usepackage[framemethod=tikz]{mdframed}
\theoremstyle{nonumberplain}

\mdtheorem[innertopmargin=-5pt]{sol}{Solution}
%\newmdtheoremenv[innertopmargin=-5pt]{sol}{Solution}

\begin{document}
\pagestyle{empty}

\newcommand{\mline}{\vspace{.2in}\hrule\vspace{.2in}}


\title{\bf { AMATH 567 Fall 2024 \\ Homework 1 ---
    Due Sept. 30 on Gradescope by 1:30pm\\
  The 48 hour late penalty is waived for this assignment} }


\maketitle

\begin{center}
  All solutions must include significant justification to receive full credit.  If you handwrite your assignment you must either do so digitally or if it is written on paper you must \emph{scan} your work.  A standard photo is not sufficient.  \vskip 4pt

  If you work with others on the homework, you must name your collaborators.
\end{center}


\mline
\begin{enumerate}[label={\bf {\arabic*}:}]
\item  From A\&F: 1.1.1: (b, e)
Express each of the following complex numbers in exponential form: \\
b) $-i$\\
\textit{Solution:}
$ -i = 0 -i = 0 + i(-1) = \cos(\frac{3\pi}{2}) + i \sin(\frac{3\pi}{2}) = e^{i\frac{3\pi}{2}}$ \\
e) $\frac{1}{2}-\frac{\sqrt{3}}{2}i$ \\
\textit{Solution:}
$\frac{1}{2}-\frac{\sqrt{3}}{2}i = \cos(\frac{5\pi}{3}) + i \sin(\frac{5\pi}{3}) = e^{i\frac{5\pi}{3}}$ \\

\item From A\&F: 1.1.2: b, c, d \\


\item From A\&F: 1.1.3: d \\


\item From A\&F: 1.1.4: d,f \\


\item For $a, b \in \mathbb C$, define
  \begin{align*}
    a^b = e^{b \log a},
  \end{align*}
  where $a = r e^{i \theta}$, $- \pi < \theta \leq \pi$ and
  \begin{align*}
    \log a = \log r + i \theta,
  \end{align*}
  is the principal branch of the logarithm.  Find the real and
  imaginary parts of
  \begin{align*}
    i^i \quad \text{and} \quad (1 + i)^i.
  \end{align*}
 
 
\item Consider the function $e(z):=\sum_{n=0}^{\infty}
  \frac{z^n}{n!}$, which is defined for all $z \in \mathbb{C}$ (you
  need not show this). Using only the power series, show that
  $e\left(z_1+z_2\right)=e\left(z_1\right) e\left(z_2\right)$. Can you
  find other power series with the same property? \\

  
\item Consider the complex-valued expression
$$
f(z)=z^{1 / 2}
$$
where $z=x+i y$, with $x, y \in \mathbb{R}$. Derive explicit
expressions for the real and imaginary part(s) of $f(z)$ in terms of
$x$ and $y$. If you make any choices (e.g. for branch cuts), show how
they impact your answer. Your answer should not contain any trig
functions.\\



\item (Solution of the cubic) Consider the cubic equation

$$
x^3+a x^2+b x+c=0,
$$

where $a, b$ and $c$ are given numbers.
\begin{itemize}
\item Use the change of variables $x=y-a / 3$ to reduce the equation to the form

$$
y^3+p y+q=0
$$


Find expressions for $p$ and $q$.
\item Let $y=u+v$. We're replacing one unknown with two, so we get to impose another constraint later. Check that

$$
u^3+v^3+(3 u v+p)(u+v)+q=0 \text {. }
$$

\item Now we impose $3 u v+p=0$, so that

$$
u^3 v^3=-p^3 / 27
$$

Also, from above, we have

$$
u^3+v^3=-q .
$$

Find a quadratic equation satisfied by both $u^3$ and $v^3$.
\item Solve this quadratic equation, finding expressions for $u$ and $v$.
\item Finally, obtain an expression for $x$. How many different solutions does your expression give rise to?
\item Use your result to solve the cubic $x^3+3 x^2+6 x+8=0$.
\item (Bombelli's equation) Use your result to solve the cubic $x^3-15
  x-4=0$, writing your result explicitly in terms of real and
  imaginary parts.
  \end{itemize}



\end{enumerate}

\end{document}

%%% Local Variables:
%%% mode: latex
%%% TeX-master: t
%%% End:
