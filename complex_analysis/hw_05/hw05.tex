\documentclass[10pt]{amsart}
\usepackage[margin=1.4in]{geometry}
\usepackage[usenames,dvipsnames,cmyk]{xcolor} %load first
\usepackage{cancel}
\usepackage{graphicx,subfig}
\graphicspath{ {./images/} }

\usepackage{amssymb,amsmath,enumitem,url}

\newcommand{\D}{\mathrm{d}}
\newcommand{\I}{\mathrm{i}}
\DeclareMathOperator{\E}{e}
\DeclareMathOperator{\OO}{O}
\DeclareMathOperator{\oo}{o}
\DeclareMathOperator{\erfc}{erfc}
\DeclareMathOperator{\real}{Re}
\DeclareMathOperator{\imag}{Im}
\usepackage{tikz}
\usepackage[framemethod=tikz]{mdframed}
\theoremstyle{nonumberplain}

\mdtheorem[innertopmargin=5pt]{lemma}{Lemma}
\mdtheorem[innertopmargin=-5pt]{sol}{Solution}
%\newmdtheoremenv[innertopmargin=-5pt]{sol}{Solution}
\definecolor{MichiganBlue}{HTML}{00274C}
\definecolor{MichiganYellow}{HTML}{FFCB05}  
\definecolor{NicePurple}{RGB}{75,56,76} %PrincePurple
\definecolor{NiceRed}{RGB}{230,37,52}
\definecolor{MidnightBlue}{rgb}{0.1, 0.1, 0.44}
\usepackage[colorlinks=true, linkcolor=MidnightBlue, citecolor=MidnightBlue, urlcolor=MidnightBlue]{hyperref}

\begin{document}
\pagestyle{empty}

\newcommand{\mline}{\vspace{.2in}\hrule\vspace{.2in}}

\noindent
\text{Hunter Lybbert} \\
\text{Student ID: 2426454} \\
\text{10-28-24} \\
\text{AMATH 567} \\

\title{\bf { Homework 5} }


\maketitle
\noindent
Collaborators*: \\
\\
\tiny
\text{*Listed in no particular order. And anyone I discussed at least part of one problem with is considered a collaborator.}
\normalsize
\mline
\begin{enumerate}[label={\bf {\arabic*}:}]
\item  From A\&F: 2.6.5\\
Consider two entire functions with no zeroes and having a ratio equal to unity at infinity.
Use Liouville's Theorem to show that they are in fact the same function. \\
\textit{Solution:} \\
Let's define our two entire functions to be $f(z)$ and $g(z)$.
Recall that an entire function is analytic in all of the complex plane.
We can focus on the ratio between these two functions $\frac {f(z)}{g(z)}$ since we are also given that $f(z)$ and $g(z)$ have no zeros.
Let $h(z)$ be the ratio between $f$ and $g$
$$ h(z) = \frac {f(z)}{g(z)}. $$
If we can use Liouville's theorem to show that $h(z)$ is constant, then $f(z)$ and $g(z)$ are equal everywhere and are thus the same function. \\

\noindent
For reference, Liouville's Theorem states that if $f(z)$ is entire and bounded in the $z$ plane (including infinity), then $f(z)$ is a constant.
Hence we need to show that $h(z)$ is entire and bounded in the $z$ plane, then $h(z)$ is constant and we will have what we want.
We know that the functions $f(z)$ and $g(z)$ are entire. We also know that the function $\frac 1 {z}$ is analytic except when $z = 0$.
Since neither $f$ nor $g$ have zeros, then the potential of having $0$ in the denominator of $h(z)$ is no longer an issue.
Therefore $\frac 1 z,\:\: z\neq 0$ is entire.
Therefore $h(z)$ is entire since it is the composition of entire functions. \\

\noindent
Now we need to show that $h(z)$ is bounded in the $z$ plane.
Since $h(z)$ is entire, then it is analytic interior to and on a simple closed contour $C$ (which we will choose later), then by Theorem 2.6.2, we have
$$
h^{(n)}(z) = \frac {n!}{2 \pi \I} \oint_C \frac {f(\xi)}{(\xi - z)^{n + 1}} \D \xi.
$$
Now we can use the established inequality (2.6.13 in A \& F)
$$
\left| h^{(n)}(z) \right| \leq \frac {n!M}{R^n}.
$$
When $n = 1$ we have
$$
\left| h^\prime(z) \right| \leq \frac {M}{R}.
$$
We can take $R$ to be arbitrarily large to get $\left| h^\prime(z) \right| \leq 0$ implying $h^\prime(z) = 0$.
Using the fundamental theorem of calculus we can write
$$
h(\infty) - h(z) = \int_z^{\infty} h^\prime(z) \D z = \left. C \right|_z^{\infty} = C - C = 0.
$$
This gives $h(\infty) = h(z)$, therefore, by Liouville's Theorem $h(z)$ is constant.
From the problem's setup we know $h(\infty) = \frac {f(\infty)}{g(\infty)} = 1$.
Hence,
$$h(\infty) = h(z) = 1.$$
Therefore, $f(z)$ and $g(z)$ must be the same function, since their ratio is 1 for all $z$.\\
\qed
\\

\item From A\&F: 2.6.10
(The \textit{solution} is peppered throught, since there are many things to show and information given between many steps.)\\
In Cauchy's Integral Formula,
$$
f(z) = \frac 1 {2\pi \I} \oint_C \frac{f(\xi)}{\xi - z} \D \xi
$$
take the contour to be a circle of unit radius centered at the origin.
Let $\xi = \E^{\I\theta}$.
We now can plug the substitution in, along with the $\D \xi = \I \E^{\I \theta} \D \theta$ to get
\begin{align*}
f(z) &= \frac 1 {2\pi \I} \oint_C \frac{f(\xi)}{\xi - z} \D \xi \\
	&= \frac 1 {2\pi \cancel{\I}} \int_0^{2\pi} \frac{f(\xi) \cancel{\I} \E^{\I \theta}}{\xi - z} \D \theta \\
	&= \frac 1 {2\pi} \int_0^{2\pi} \frac{f(\xi)\xi}{\xi - z} \D \theta.
\end{align*}
\qed \\
Since $z$ is inside the unit circle and $z=r\E^{\I \theta}$, then $r < 1$.
Then we have
$$ \frac 1 {\bar z} =  \frac 1 {r \E^{-\I \theta}} = \frac 1 r { \E^{\I \theta}}.$$
Then $\frac 1 r  > 1$, hence $\frac 1 {\bar z}$ is outside the unit circle.
Therefore plugging in $\frac 1 {\bar z}$ to Cauchy's Formula from the beginning again we have
\begin{align*}
\frac 1 {2\pi \I} \oint_C \frac{f(\xi)}{\xi - \frac 1 {\bar z}} \D \xi &= 0 \\
\frac 1 {2\pi \cancel{\I}} \int_0^{2\pi} \frac{f(\xi) \cancel{\I} \E^{\I \theta}}{\xi - \frac 1 {\bar z}} \D \theta &= 0 \\
\frac 1 {2\pi} \int_0^{2\pi} \frac{f(\xi)\xi}{\xi - \frac 1 {\bar z}} \D \theta &= 0.
\end{align*}
\qed \\

\noindent
Notice,
\begin{align*}
f(z) &= \frac 1 {2\pi} \int_0^{2\pi} \frac{f(\xi)\xi}{\xi - z} \D \theta \mp 0 \\
	&= \frac 1 {2\pi} \int_0^{2\pi} \frac{f(\xi)\xi}{\xi - z} \D \theta \mp \frac 1 {2\pi} \int_0^{2\pi} \\
	&= \frac 1 {2\pi} \int_0^{2\pi} f(\xi) \left( \frac{\xi}{\xi - z} \mp \frac{\xi}{\xi - \frac 1 {\bar z}} \right)\D \theta.
\end{align*}
Now we can use $\xi = 1/{\bar \xi}$ to get
\begin{align*}
f(z) &= \frac 1 {2\pi} \int_0^{2\pi} f(\xi) \left( \frac{\xi}{\xi - z} \mp \frac{1/{\bar \xi}}{1/{\bar \xi} - \frac 1 {\bar z}} \right) \D \theta \\
	&= \frac 1 {2\pi} \int_0^{2\pi} f(\xi) \left( \frac{\xi}{\xi - z} \mp \frac {\bar \xi \bar z}{\bar \xi \bar z} \frac{1/{\bar \xi}}{1/{\bar \xi} - \frac 1 {\bar z}} \right) \D \theta \\
	&= \frac 1 {2\pi} \int_0^{2\pi} f(\xi) \left( \frac{\xi}{\xi - z} \mp \frac{\bar z}{\bar z - \bar \xi} \right) \D \theta \\
	&= \frac 1 {2\pi} \int_0^{2\pi} f(\xi) \left( \frac{\xi}{\xi - z} \pm \frac{\bar z}{\bar \xi - \bar z} \right) \D \theta.
\end{align*}
\qed \\
Using the plus sign we see
\begin{align*}
f(z) &= \frac 1 {2\pi} \int_0^{2\pi} f(\xi) \left( \frac{\xi}{\xi - z} + \frac{\bar z}{\bar \xi - \bar z} \right) \D \theta \\
	&= \frac 1 {2\pi} \int_0^{2\pi} f(\xi) \left( \frac{\xi (\bar \xi - \bar z)}{(\bar \xi - \bar z)(\xi - z)} + \frac{\bar z (\xi - z)}{(\bar \xi - \bar z)(\xi - z)} \right) \D \theta \\
	&= \frac 1 {2\pi} \int_0^{2\pi} f(\xi) \left( \frac{\xi (\bar \xi - \bar z)+ \bar z (\xi - z)}{(\bar \xi - \bar z)(\xi - z)}\right) \D \theta \\
	&= \frac 1 {2\pi} \int_0^{2\pi} f(\xi) \left( \frac{\xi\bar \xi \cancel{- \xi \bar z + \xi \bar z} - z \bar z)}{(\bar \xi - \bar z)(\xi - z)}\right) \D \theta \\
	&= \frac 1 {2\pi} \int_0^{2\pi} f(\xi) \left( \frac{|\xi|^2 - |z|^2}{(\bar \xi - \bar z)(\xi - z)}\right) \D \theta \\
	&= \frac 1 {2\pi} \int_0^{2\pi} f(\xi) \left( \frac{|\xi|^2 - |z|^2}{\overline{(\xi - z)}(\xi - z)}\right) \D \theta \\
	&= \frac 1 {2\pi} \int_0^{2\pi} f(\xi) \left( \frac{|\xi|^2 - |z|^2}{|\xi - z|^2}\right) \D \theta \\
	&= \frac 1 {2\pi} \int_0^{2\pi} f(\xi) \left( \frac{|\E^{\I\theta}|^2 - |z|^2}{|\xi - z|^2}\right) \D \theta \\
	&= \frac 1 {2\pi} \int_0^{2\pi} f(\xi) \left( \frac{1^2 - |z|^2}{|\xi - z|^2}\right) \D \theta \\
	&= \frac 1 {2\pi} \int_0^{2\pi} f(\xi) \left( \frac{1 - |z|^2}{|\xi - z|^2}\right) \D \theta.
\end{align*}
\qed \\

\noindent
(a) Deduce the ``Poisson formula" for the real part of $f(z): u(r, \phi) = \Re f, z = r\E^{i\phi}$.
\begin{align*}
f(z) &= \frac 1 {2\pi} \int_0^{2\pi} f(\xi) \left( \frac{1 - |z|^2}{|\xi - z|^2}\right) \D \theta \\
	&= \frac 1 {2\pi} \int_0^{2\pi} (u(\theta) + \I v(\theta))\left( \frac{1 - |z|^2}{|\xi - z|^2}\right) \D \theta \\
	&= \frac 1 {2\pi} \int_0^{2\pi} \left(u(\theta)\left( \frac{1 - |z|^2}{|\xi - z|^2}\right) + \I v(\theta)\left( \frac{1 - |z|^2}{|\xi - z|^2}\right)\right) \D \theta \\
	&= \frac 1 {2\pi} \int_0^{2\pi} u(\theta)\left( \frac{1 - |z|^2}{|\xi - z|^2}\right) \D \theta + \frac 1 {2\pi} \int_0^{2\pi} \I v(\theta)\left( \frac{1 - |z|^2}{|\xi - z|^2}\right) \D \theta \\
	&= \frac 1 {2\pi} \int_0^{2\pi} u(\theta)\left( \frac{1 - |z|^2}{|\xi - z|^2}\right) \D \theta + \I \frac 1 {2\pi} \int_0^{2\pi} v(\theta)\left( \frac{1 - |z|^2}{|\xi - z|^2}\right) \D \theta
\end{align*}
Thus,
\begin{align*}
u(r, \phi) = \frac 1 {2\pi} \int_0^{2\pi} u(\theta)\left( \frac{1 - |z|^2}{|\xi - z|^2}\right) \D \theta.
\end{align*}
We know $|z| = \left|r\E^{i\phi}\right| = r$.
Now let's look specifically at the denominator and plugin the substitutions for $z$ to get
\begin{align*}
|\xi - z|^2 &= |\E^{\I \theta} - r\E^{\I \phi}|^2 \\
	&= |\cos\theta + \I \sin\theta - r \cos \phi - \I r\sin \phi|^2 \\
	&= |\cos\theta - r \cos \phi + \I \left(\sin\theta - r\sin \phi \right)|^2 \\
	&= \left(\cos\theta - r \cos \phi\right)^2 + \left(\sin\theta - r\sin \phi \right)^2 \\
	&= \cos^2\theta - 2r \cos\theta\cos \phi + r^2 \cos^2 \phi + \sin^2\theta - 2\sin\theta r\sin \phi+ r^2\sin^2 \phi \\
	&= \cos^2\theta + \sin^2\theta - 2r \cos\theta\cos \phi - 2\sin\theta r\sin \phi + r^2 \cos^2 \phi +  r^2\sin^2 \phi \\
	&= 1 - 2r \cos\theta\cos \phi - 2\sin\theta r\sin \phi + r^2\left( \cos^2 \phi + \sin^2 \phi\right) \\
	&= 1 - 2r \cos\theta\cos \phi - 2\sin\theta r\sin \phi + r^2 \\
	&= 1 - 2r \left(\cos\theta \cos\phi + \sin\theta \sin\phi \right) + r^2 \\
	&= 1 - 2r \cos (\theta - \phi) + r^2.
\end{align*}
Hence,
$$
u(r, \phi) = \frac 1 {2\pi} \int_0^{2\pi} u(\theta) \frac{1 - r^2}{1 - 2r \cos (\theta - \phi) + r^2} \D \theta
$$
\qed \\

\noindent
(b) If we use the minus sign in the formula for $f(z)$ above, we get
\begin{align*}
f(z) &= \frac 1 {2\pi} \int_0^{2\pi} f(\xi) \left( \frac{\xi}{\xi - z} - \frac{\bar z}{\bar \xi - \bar z} \right) \D \theta \\
	&= \frac 1 {2\pi} \int_0^{2\pi} f(\xi) \left( \frac{\xi (\bar \xi - \bar z)}{(\bar \xi - \bar z)(\xi - z)} - \frac{\bar z (\xi - z)}{(\bar \xi - \bar z)(\xi - z)} \right) \D \theta \\
	&= \frac 1 {2\pi} \int_0^{2\pi} f(\xi) \left( \frac{\xi (\bar \xi - \bar z)- \bar z (\xi - z)}{(\bar \xi - \bar z)(\xi - z)}\right) \D \theta \\
	&= \frac 1 {2\pi} \int_0^{2\pi} f(\xi) \left( \frac{\xi\bar \xi - 2\xi \bar z + z \bar z)}{(\bar \xi - \bar z)(\xi - z)}\right) \D \theta \\
	&= \frac 1 {2\pi} \int_0^{2\pi} f(\xi) \left( \frac{|\xi|^2 - 2\xi \bar z + |z|^2)}{\overline{(\xi - z)}(\xi - z)}\right) \D \theta \\
	&= \frac 1 {2\pi} \int_0^{2\pi} f(\xi) \left( \frac{1 - 2\xi \bar z + r^2}{\left|\xi - z\right|^2}\right) \D \theta \\
	&= \frac 1 {2\pi} \int_0^{2\pi} f(\xi) \frac{1 - 2\xi \bar z + r^2}{1 - 2r \cos (\theta - \phi) + r^2} \D \theta \\
	&= \frac 1 {2\pi} \int_0^{2\pi} f(\xi) \frac{1 - 2\E^{\I \theta} r\E^{-\I \phi} + r^2}{1 - 2r \cos (\theta - \phi) + r^2} \D \theta \\
	&= \frac 1 {2\pi} \int_0^{2\pi} f(\xi) \frac{1 - 2r \E^{\I \theta - \I \phi} + r^2}{1 - 2r \cos (\theta - \phi) + r^2} \D \theta \\
	&= \frac 1 {2\pi} \int_0^{2\pi} f(\xi) \frac{1 - 2r \E^{\I (\theta - \phi)} + r^2}{1 - 2r \cos (\theta - \phi) + r^2} \D \theta \\
\end{align*}
\qed \\
% \newpage

\noindent
Additionally, if we take the imaginary part this time, we can see
\begin{align*}
f(z) &= \frac 1 {2\pi} \int_0^{2\pi} f(\xi) \frac{1 - 2r \E^{\I (\theta - \phi)} + r^2}{1 - 2r \cos (\theta - \phi) + r^2} \D \theta \nonumber \\ \\
	&= \frac 1 {2\pi} \int_0^{2\pi} (u(\theta) + \I v(\theta))\left(\frac{1 - 2r \cos(\theta - \phi) -2r\I \sin(\theta - \phi) + r^2}{1 - 2r \cos (\theta - \phi) + r^2} \right) \D \theta \nonumber \\ \\
	&= \frac 1 {2\pi} \int_0^{2\pi} (u(\theta) + \I v(\theta))\left(\frac{1 - 2r \cos(\theta - \phi)  + r^2 -2r\I \sin(\theta - \phi)}{1 - 2r \cos (\theta - \phi) + r^2} \right) \D \theta \nonumber \\ \\
	&= \frac 1 {2\pi} \int_0^{2\pi} (u(\theta) + \I v(\theta))\left(1 + \frac{-2r\I \sin(\theta - \phi)}{1 - 2r \cos (\theta - \phi) + r^2} \right) \D \theta \nonumber \\ \\
	&= \frac 1 {2\pi} \int_0^{2\pi} (u(\theta) + \I v(\theta))\left(1 - \frac{\I 2r\sin(\theta - \phi)}{1 - 2r \cos (\theta - \phi) + r^2} \right) \D \theta
\end{align*}
Let's expand out the terms in the integrand
\begin{align*}
	&= \frac 1 {2\pi} \int_0^{2\pi} \left[
		u(\theta)
		- u(\theta)\left(\frac{\I 2r\sin(\theta - \phi)}{1 - 2r \cos (\theta - \phi) + r^2} \right)
		+ \I v(\theta)
		- \I v(\theta)\left(\frac{\I 2r\sin(\theta - \phi)}{1 - 2r \cos (\theta - \phi) + r^2} \right)
	\right] \D \theta \\ \\
	&= \frac 1 {2\pi} \int_0^{2\pi} \left[
		u(\theta)
		- \I u(\theta)\left(\frac{2r\sin(\theta - \phi)}{1 - 2r \cos (\theta - \phi) + r^2} \right)
		+ \I v(\theta)
		- \I^2 v(\theta)\left(\frac{2r\sin(\theta - \phi)}{1 - 2r \cos (\theta - \phi) + r^2} \right)
	\right] \D \theta \\ \\
	&= \frac 1 {2\pi} \int_0^{2\pi} \left[
		u(\theta)
		- \I u(\theta)\left(\frac{2r\sin(\theta - \phi)}{1 - 2r \cos (\theta - \phi) + r^2} \right)
		+ \I v(\theta)
		+ v(\theta)\left(\frac{2r\sin(\theta - \phi)}{1 - 2r \cos (\theta - \phi) + r^2} \right)
	\right] \D \theta. \\
\end{align*}
Now the imaginary part of this is
\begin{align*}
\Im (f(z)) &= \frac 1 {2\pi} \int_0^{2\pi} \left(
	v(\theta) - u(\theta)\frac{2r\sin(\theta - \phi)}{1 - 2r \cos (\theta - \phi) + r^2}
\right) \D \theta \\
	&= \frac 1 {2\pi} \int_0^{2\pi} v(\theta) \D \theta - \frac 1 {2\pi} \int_0^{2\pi} u(\theta)\frac{2r\sin(\theta - \phi)}{1 - 2r \cos (\theta - \phi) + r^2} \D \theta \\
	&= C - \frac 1 {\pi} \int_0^{2\pi} u(\theta) \frac{r\sin(\theta - \phi)}{1 - 2r \cos (\theta - \phi) + r^2} \D \theta \\
	&= C - \frac 1 {\pi} \int_0^{2\pi} u(\theta) \frac{r\sin\big(-(-\theta + \phi)\big)}{1 - 2r \cos \big(-(-\theta + \phi)\big) + r^2} \D \theta \\
	&= C - \frac 1 {\pi} \int_0^{2\pi} u(\theta) \frac{- r\sin(\phi -\theta)}{1 - 2r \cos (\phi-\theta) + r^2} \D \theta \\
	&= C + \frac 1 {\pi} \int_0^{2\pi} u(\theta) \frac{r\sin(\phi -\theta)}{1 - 2r \cos (\phi-\theta) + r^2} \D \theta \\
\end{align*}
where $C = \frac 1 {2 \pi} \int_0^{2 \pi} v(1, \theta) \D \theta = v(r = 0)$.
This last relationship follows from the Cauchy Integral formula at z = 0 -- see the first equation in this exercise).
Hence,
$$
v(r, \phi) = C + \frac 1 {\pi} \int_0^{2\pi} u(\theta) \frac{r\sin(\phi -\theta))}{1 - 2r \cos (\phi-\theta) + r^2} \D \theta \\
$$
\qed \\

\noindent
(c) We wish to show
$$
\frac{r\sin(\phi -\theta)}{1 - 2r \cos (\phi-\theta) + r^2} = \Im \bigg(\frac {\xi + z}{\xi - z}\bigg)
$$
Let's try a bit
\begin{align*}
\frac {\xi + z}{\xi - z} &= \frac {\overline{(\xi - z)}(\xi + z)}{\overline{(\xi - z)}(\xi - z)} \\
	&= \frac {\overline{(\xi - z)}(\xi + z)}{|\xi - z|^2}. \\
\end{align*}
We have already computed this denominator once.
Using our previous result we continue
\begin{align*}
\frac {\overline{(\xi - z)}(\xi + z)}{|\xi - z|^2}
	&= \frac {(\bar \xi - \bar z)(\xi + z)}{1 - 2r \cos (\phi - \theta) + r^2} \\
	&= \frac {\bar \xi \xi + \bar \xi z - \bar z \xi - \bar z z}{1 - 2r \cos (\phi - \theta) + r^2} \\
	&= \frac {|\xi|^2 + \bar \xi z - \bar z \xi - |z|^2}{1 - 2r \cos (\phi - \theta) + r^2} \\
	&= \frac {1 + \bar \xi z - \bar z \xi - r^2}{1 - 2r \cos (\phi - \theta) + r^2}.
\end{align*}
Now, lets plugin our parameterizations of $\xi$ and $z$
\begin{align*}
	&= \frac {1 - r^2 + \E^{-\I \theta} r\E^{\I \phi} - r\E^{-\I \phi} \E^{\I \theta}}{1 - 2r \cos (\phi - \theta) + r^2} \\
	&= \frac {1 - r^2 + r\E^{\I (\phi - \theta)} - r\E^{\I(\theta - \phi)}}{1 - 2r \cos (\phi - \theta) + r^2} \\
	&= \frac {1 - r^2 + r(\cos(\phi - \theta) + \I \sin(\phi - \theta)) - r(\cos(\theta - \phi) + \I \sin(\theta - \phi))}{1 - 2r \cos (\phi - \theta) + r^2} \\
	&= \frac {1 - r^2 + r\cos(\phi - \theta) - r\cos(\theta - \phi) + \I r\sin(\phi - \theta) - \I r\sin(\theta - \phi)}{1 - 2r \cos (\phi - \theta) + r^2} \\
	&= \frac {1 - r^2 + \cancel{r\cos(\phi - \theta)} - \cancel{r\cos(\phi - \theta)} + \I r\sin(\phi - \theta) + \I r\sin(\phi - \theta)}{1 - 2r \cos (\phi - \theta) + r^2} \\
	&= \frac {1 - r^2 + \I 2 r\sin(\phi - \theta)}{1 - 2r \cos (\phi - \theta) + r^2}.
\end{align*}
We have arrived to 
\begin{align*}
\frac {\xi + z}{\xi - z} &= \frac {1 - r^2 + \I 2 r\sin(\phi - \theta)}{1 - 2r \cos (\phi - \theta) + r^2} \\
\Im \bigg[\frac {\xi + z}{\xi - z} \bigg] &= \Im \bigg[\frac {1 - r^2 + \I 2 r\sin(\phi - \theta)}{1 - 2r \cos (\phi - \theta) + r^2} \bigg] \\
	&= \frac {2 r\sin(\phi - \theta)}{1 - 2r \cos (\phi - \theta) + r^2}.
\end{align*}
Therefore, 
\begin{align*}
v(r, \phi) &= C + \frac 1 {\pi} \int_0^{2\pi} u(\theta) \frac{r\sin(\phi -\theta))}{1 - 2r \cos (\phi-\theta) + r^2} \D \theta \\
	&= C + \frac \Im {2\pi} \int_0^{2\pi} u(\theta) \frac {\xi + z}{\xi - z} \D \theta
\end{align*}
\qed \\
This example illustrates that prescribing the real part of $f(z)$ on $|z| = 1$ determines (a) the real part of $f(z)$ everywhere inside the circle and (b) the imaginary part of $f(z)$ inside the circle to within a constant.
We \textit{cannot} arbitrarily specify both the real and imaginary parts of an analytic function on $|z| = 1$.

\item Suppose $\Omega$ is an open simply connected region and $z_0 \in
  \Omega$.  Assume that $f(z)$ is analytic in $\Omega\setminus
  \{z_0\}$ and satisfies
  \begin{align*}
    |f(z)| \leq M |z - z_0|^{-\gamma}, \quad \gamma < 1.
  \end{align*}
  Show that if the a specific choice for $f(z_0)$ is made then $f$
  extends to an analytic function on $\Omega$. \\
(\textbf{1 part}, except maybe if there are multiple things to prove here) \\
\textit{Solution:} \\
Since 
$$|f(z)| \leq M |z - z_0|^{-\gamma}, \quad \gamma < 1$$
...\\
Since $f(z)$ is analytic we know
\begin{enumerate}
\item $f(z)$ satisfies the Cauchy Riemann (C-R) equations.
\item then on a contour $C \subset \Omega$ f(z) satisfies
$$
f^{(k)}(z) = \frac {k!}{2 \pi i} \oint_C \frac {f(\xi)}{(\xi - z)^{k + 1}}\D \xi
$$
\end{enumerate}
Do I need to use the Maximum Principles?? \\
Should I make a choice of $f(z_0)$ and prove that $f$ with that condition is analytic or should I just prove that one exists?

\newpage
\item Establish the following lemma:\\
    \begin{lemma}
      Suppose $\Omega$ is an open region and $f(z)$ is continuous on
      $\overline \Omega$.  Let $\Gamma$ be a contour in $\overline
      \Omega$.  Suppose a sequence of contours $\Gamma_n \subset
      \overline \Omega$ converge to
      $\Gamma$ in the sense that there exists parameterizations $z(t)$
      of $\Gamma$ and $z_n(t)$ of $\Gamma_n$ defined on $[a,b]$
      satisfying
      \begin{align*}
        z_n(t) \overset{n \to \infty}{\longrightarrow} z(t),  \quad \text{
        uniformly on } [a,b],\\
        z_n'(t) \overset{n \to \infty}{\longrightarrow} z'(t), \quad \text{
        uniformly on } [a,b].
      \end{align*}
      Then
      \begin{align*}
        \int_{\Gamma_n} f(z) \D z \overset{n \to
        \infty}{\longrightarrow}   \int_{\Gamma} f(z) \D z.
      \end{align*}
    \end{lemma}
    Hint: Use that $f$ is uniformly continuous on $\overline \Omega$.\\
(\textbf{1 part}, except maybe if there are multiple things to prove here) \\
\textit{Solution:} \\

\item for any $r, R > 0$, let $C = \partial \Sigma$, $\Sigma = \{z \in \mathbb C ~:~
      |\real z | \leq r \text{ and } 0 \leq -\imag z \leq R, ~ R >
    0\}$.   In this problem $\sqrt{z}$ denotes the principal branch
    with $\arg z  \in [-\pi, \pi)$.
    \begin{itemize}
    \item Show that if $f(z)$ is analytic in a region that contains $\Sigma$,
      \begin{align*}
        \oint_C f(z) \sqrt{z-1} \sqrt{z+1} \D z = 0.
      \end{align*}
(\textbf{1 part}) \\
\textit{Solution:} \\
Consider just siting Cauchy's Theorem...
\\
  \item Show that if $f(z)$ is analytic in a region that contains $\Sigma$
    \begin{align*}
        \oint_C \frac{f(z) \D z}{\sqrt{z-1} \sqrt{z+1}}= 0.
      \end{align*}
(\textbf{1 part}) \\
\textit{Solution:} \\
Deal with the singularities on the boundary.
\\
    \end{itemize}
\item From A\&F: 3.1.1 b,d \\
In the following we are given sequences.
Discuss their limits and whether the convergence is uniform, in the region $\alpha \leq |z| \leq \beta$, for finite $\alpha, \beta > 0$. \\
b) $$ \bigg\{ \frac 1 {z^n} \bigg\}_{n=1}^{\infty} $$
(\textbf{2 parts}) \\
\textit{Solution:}\\

\noindent
d) $$ \bigg\{ \frac 1 {1 + (nz)^2} \bigg\}_{n=1}^{\infty} $$
(\textbf{2 parts}) \\
\textit{Solution:}\\

\item From A\&F: 3.1.2 b,d \\
For each sequence in problem 1, what can be said if \\
(a) $\alpha = 0$ \\
(b) $\alpha > 0, \quad \beta = \infty$ \\
(\textbf{4 parts} 2x2)
\textit{Solution:}
\item From A\&F: 3.1.3
Compute the integrals
$$
\lim_{n \rightarrow \infty} \int_0^1 nz^{n -1}\D z \quad \text{and} \quad \int_0^1 \lim_{n \rightarrow \infty} \left(nz^{n -1} \right) \D z
$$
and show that they are not equal.
Explain why this is not a counter example to Theorem 3.1.1. (A \&F pg. 111) \\
(\textbf{3 parts})
\textit{Solution:} \\
For the integral on the right consider adding a limit outside the integral so the bound doesn't have any issues.
    
    
\textbf{There are approximately 25 things to do, 10 down!}
\end{enumerate}

\end{document}

%%% Local Variables:
%%% mode: latex
%%% TeX-master: t
%%% End:
