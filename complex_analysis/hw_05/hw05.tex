\documentclass[10pt]{amsart}
\usepackage[margin=1.4in]{geometry}
\usepackage[usenames,dvipsnames,cmyk]{xcolor} %load first
\usepackage{cancel}
\usepackage{graphicx,subfig}
\graphicspath{ {./images/} }

\usepackage{amssymb,amsmath,enumitem,url}

\newcommand{\D}{\mathrm{d}}
\newcommand{\I}{\mathrm{i}}
\DeclareMathOperator{\E}{e}
\DeclareMathOperator{\OO}{O}
\DeclareMathOperator{\oo}{o}
\DeclareMathOperator{\erfc}{erfc}
\DeclareMathOperator{\real}{Re}
\DeclareMathOperator{\imag}{Im}
\usepackage{tikz}
\usepackage[framemethod=tikz]{mdframed}
\theoremstyle{nonumberplain}

\mdtheorem[innertopmargin=5pt]{lemma}{Lemma}
\mdtheorem[innertopmargin=-5pt]{sol}{Solution}
%\newmdtheoremenv[innertopmargin=-5pt]{sol}{Solution}
\definecolor{MichiganBlue}{HTML}{00274C}
\definecolor{MichiganYellow}{HTML}{FFCB05}  
\definecolor{NicePurple}{RGB}{75,56,76} %PrincePurple
\definecolor{NiceRed}{RGB}{230,37,52}
\definecolor{MidnightBlue}{rgb}{0.1, 0.1, 0.44}
\usepackage[colorlinks=true, linkcolor=MidnightBlue, citecolor=MidnightBlue, urlcolor=MidnightBlue]{hyperref}

\begin{document}
\pagestyle{empty}

\newcommand{\mline}{\vspace{.2in}\hrule\vspace{.2in}}

\noindent
\text{Hunter Lybbert} \\
\text{Student ID: 2426454} \\
\text{10-28-24} \\
\text{AMATH 567} \\

\title{\bf { Homework 5} }


\maketitle
\noindent
Collaborators*: \\
\\
\tiny
\text{*Listed in no particular order. And anyone I discussed at least part of one problem with is considered a collaborator.}
\normalsize
\mline
\begin{enumerate}[label={\bf {\arabic*}:}]
\item  From A\&F: 2.6.5\\
Consider two entire functions with no zeroes and having a ratio equal to unity at infinity.
Use Liouville's Theorem to show that they are in fact the same function. \\
\textit{Solution:} \\
Let's define our two entire functions to be $f(z)$ and $g(z)$.
Recall that an entire function is analytic in all of the complex plane.
We can focus on the ratio between these two functions $\frac {f(z)}{g(z)}$ since we are also given that $f(z)$ and $g(z)$ have no zeros.
Let $h(z)$ be the ratio between $f$ and $g$
$$ h(z) = \frac {f(z)}{g(z)}. $$
If we can use Liouville's theorem to show that $h(z)$ is constant, then $f(z)$ and $g(z)$ are equal everywhere and are thus the same function. \\

\noindent
For reference, Liouville's Theorem states that if $f(z)$ is entire and bounded in the $z$ plane (including infinity), then $f(z)$ is a constant.
Hence we need to show that $h(z)$ is entire and bounded in the $z$ plane, then $h(z)$ is constant and we will have what we want.
We know that the functions $f(z)$ and $g(z)$ are entire. We also know that the function $\frac 1 {z}$ is analytic except when $z = 0$.
Since neither $f$ nor $g$ have zeros, then the potential of having $0$ in the denominator of $h(z)$ is no longer an issue.
Therefore $\frac 1 z,\:\: z\neq 0$ is entire.
Therefore $h(z)$ is entire since it is the composition of entire functions. \\

\noindent
Now we need to show that $h(z)$ is bounded in the $z$ plane.
Since $h(z)$ is entire, then it is analytic interior to and on a simple closed contour $C$ (which we will choose later), then by Theorem 2.6.2, we have
$$
h^{(n)}(z) = \frac {n!}{2 \pi \I} \oint_C \frac {f(\xi)}{(\xi - z)^{n + 1}} \D \xi.
$$
Now we can use the established inequality (2.6.13 in A \& F)
$$
\left| h^{(n)}(z) \right| \leq \frac {n!M}{R^n}.
$$
When $n = 1$ we have
$$
\left| h^\prime(z) \right| \leq \frac {M}{R}.
$$
We can take $R$ to be arbitrarily large to get $\left| h^\prime(z) \right| \leq 0$ implying $h^\prime(z) = 0$.
Using the fundamental theorem of calculus we can write
$$
h(\infty) - h(z) = \int_z^{\infty} h^\prime(z) \D z = \left. C \right|_z^{\infty} = C - C = 0.
$$
This gives $h(\infty) = h(z)$, therefore, by Liouville's Theorem $h(z)$ is constant.
From the problem's setup we know $h(\infty) = \frac {f(\infty)}{g(\infty)} = 1$.
Hence,
$$h(\infty) = h(z) = 1.$$
Therefore, $f(z)$ and $g(z)$ must be the same function, since their ratio is 1 for all $z$.\\
\qed
\\

\item From A\&F: 2.6.10\\
... deduce
$$
f(z) = \frac 1 {2\pi} \int_0^{2\pi} \frac{f(\xi)\xi}{\xi - z} \D \theta
$$
... explain why we have
$$
0 = \frac 1 {2\pi} \int_0^{2\pi} \frac{f(\xi)\xi}{\xi - 1/\bar z} \D \theta
$$
... use something to show
$$
f(z) = \frac 1 {2\pi} \int_0^{2\pi} f(\xi) \left( \frac{\xi}{\xi - z} \pm \frac{\bar z}{\xi - \bar z} \right) \D \theta
$$
then ... \\
\textit{Solution:} \\
This is a beast of a problem there are \textbf{approximately 9} things to show... \\

\item Suppose $\Omega$ is an open simply connected region and $z_0 \in
  \Omega$.  Assume that $f(z)$ is analytic in $\Omega\setminus
  \{z_0\}$ and satisfies
  \begin{align*}
    |f(z)| \leq M |z - z_0|^{-\gamma}, \quad \gamma < 1.
  \end{align*}
  Show that if the a specific choice for $f(z_0)$ is made then $f$
  extends to an analytic function on $\Omega$. \\
(\textbf{1 part}, except maybe if there are multiple things to prove here) \\
\textit{Solution:} \\

\newpage
\item Establish the following lemma:\\
    \begin{lemma}
      Suppose $\Omega$ is an open region and $f(z)$ is continuous on
      $\overline \Omega$.  Let $\Gamma$ be a contour in $\overline
      \Omega$.  Suppose a sequence of contours $\Gamma_n \subset
      \overline \Omega$ converge to
      $\Gamma$ in the sense that there exists parameterizations $z(t)$
      of $\Gamma$ and $z_n(t)$ of $\Gamma_n$ defined on $[a,b]$
      satisfying
      \begin{align*}
        z_n(t) \overset{n \to \infty}{\longrightarrow} z(t),  \quad \text{
        uniformly on } [a,b],\\
        z_n'(t) \overset{n \to \infty}{\longrightarrow} z'(t), \quad \text{
        uniformly on } [a,b].
      \end{align*}
      Then
      \begin{align*}
        \int_{\Gamma_n} f(z) \D z \overset{n \to
        \infty}{\longrightarrow}   \int_{\Gamma} f(z) \D z.
      \end{align*}
    \end{lemma}
    Hint: Use that $f$ is uniformly continuous on $\overline \Omega$.\\
(\textbf{1 part}, except maybe if there are multiple things to prove here) \\
\textit{Solution:} \\

\item for any $r, R > 0$, let $C = \partial \Sigma$, $\Sigma = \{z \in \mathbb C ~:~
      |\real z | \leq r \text{ and } 0 \leq -\imag z \leq R, ~ R >
    0\}$.   In this problem $\sqrt{z}$ denotes the principal branch
    with $\arg z  \in [-\pi, \pi)$.
    \begin{itemize}
    \item Show that if $f(z)$ is analytic in a region that contains $\Sigma$,
      \begin{align*}
        \oint_C f(z) \sqrt{z-1} \sqrt{z+1} \D z = 0.
      \end{align*}
(\textbf{1 part}) \\
\textit{Solution:} \\
\\
  \item Show that if $f(z)$ is analytic in a region that contains $\Sigma$
    \begin{align*}
        \oint_C \frac{f(z) \D z}{\sqrt{z-1} \sqrt{z+1}}= 0.
      \end{align*}
(\textbf{1 part}) \\
\textit{Solution:} \\
\\
    \end{itemize}
\item From A\&F: 3.1.1 b,d \\
In the following we are given sequences.
Discuss their limits and whether the convergence is uniform, in the region $\alpha \leq |z| \leq \beta$, for finite $\alpha, \beta > 0$. \\
b) $$ \bigg\{ \frac 1 {z^n} \bigg\}_{n=1}^{\infty} $$
(\textbf{2 parts}) \\
\textit{Solution:}\\

\noindent
d) $$ \bigg\{ \frac 1 {1 + (nz)^2} \bigg\}_{n=1}^{\infty} $$
(\textbf{2 parts}) \\
\textit{Solution:}\\

\item From A\&F: 3.1.2 b,d \\
For each sequence in problem 1, what can be said if \\
(a) $\alpha = 0$ \\
(b) $\alpha > 0, \quad \beta = \infty$ \\
(\textbf{4 parts} 2x2)
\textit{Solution:}
\item From A\&F: 3.1.3
Compute the integrals
$$
\lim_{n \rightarrow \infty} \int_0^1 nz^{n -1}\D z \quad \text{and} \quad \int_0^1 \lim_{n \rightarrow \infty} \left(nz^{n -1} \right) \D z
$$
and show that they are not equal.
Explain why this is not a counter example to Theorem 3.1.1. (A \&F pg. 111) \\
(\textbf{3 parts})
\textit{Solution:} \\
    
    
\textbf{There are approximately 25 things to do}
\end{enumerate}

\end{document}

%%% Local Variables:
%%% mode: latex
%%% TeX-master: t
%%% End:
