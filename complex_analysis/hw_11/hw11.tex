\documentclass[10pt]{amsart}
\usepackage[margin=1.4in]{geometry}
\usepackage[usenames,dvipsnames,cmyk]{xcolor} %load first

\usepackage{amssymb,amsmath,enumitem,url}

\newcommand{\D}{\mathrm{d}}
\newcommand{\I}{\mathrm{i}}
\DeclareMathOperator{\E}{e}
\DeclareMathOperator{\OO}{O}
\DeclareMathOperator{\oo}{o}
\DeclareMathOperator{\erfc}{erfc}
\DeclareMathOperator{\real}{Re}
\DeclareMathOperator{\imag}{Im}
\usepackage{tikz}
\usepackage[framemethod=tikz]{mdframed}
\theoremstyle{nonumberplain}

\mdtheorem[innertopmargin=5pt]{lemma}{Lemma}
\mdtheorem[innertopmargin=-5pt]{sol}{Solution}
%\newmdtheoremenv[innertopmargin=-5pt]{sol}{Solution}
\definecolor{MichiganBlue}{HTML}{00274C}
\definecolor{MichiganYellow}{HTML}{FFCB05}  
\definecolor{NicePurple}{RGB}{75,56,76} %PrincePurple
\definecolor{NiceRed}{RGB}{230,37,52}
\definecolor{MidnightBlue}{rgb}{0.1, 0.1, 0.44}
\usepackage[colorlinks=true, linkcolor=MidnightBlue, citecolor=MidnightBlue, urlcolor=MidnightBlue]{hyperref}

\begin{document}
\pagestyle{empty}

\newcommand{\mline}{\vspace{.2in}\hrule\vspace{.2in}}
\renewcommand{\vec}{\mathbf}


\title{\bf { AMATH 567 Fall 2024 \\ Homework 11 ---
    Due December 6 on Gradescope by 1:30pm} }


\maketitle

\begin{center}
  All solutions must include significant justification to receive full credit.  If you handwrite your assignment you must either do so digitally or if it is written on paper you must \emph{scan} your work.  A standard photo is not sufficient.  \vskip 4pt

  If you work with others on the homework, you must name your collaborators.
\end{center}


\mline
\begin{enumerate}[label={\bf {\arabic*}:}]

\item
  \begin{enumerate}
    \item Let $f: z \rightarrow w=f(z)$ be an analytic function on the closed disk $D\left(z_0, R\right)$ of radius $R$ centered at $z_0$. Denote the boundary of $D\left(z_0, R\right)$ by $C\left(z_0, R\right)$. Assume that $f: D\left(z_0, R\right) \rightarrow f\left(D\left(z_0, R\right)\right)$ is one-to-one and onto. Show that the inverse function to $f$ is

$$
g(w)=\frac{1}{2 \pi i} \oint_{C\left(z_0, R\right)} \frac{t f^{\prime}(t)}{f(t)-w} d t,
$$

for $w \in f\left(D\left(z_0, R\right)\right)$, not including the
boundary.
\item Use this result to calculate the Taylor series of the inverse
  function of $w=z e^z$ around $\left(z_0, w_0\right)=(0,0)$. What is
  the radius of convergence of this series? Does this radius of
  convergence correspond with what you expect from real analysis?\\

  \end{enumerate}
\item Suppose that $A \in \mathbb C^{n \times n}$ and $\vec u$ is an
  eigenvector with eigenvalue $\lambda$.  Show that if no eigenvalue
  of $A$ is on a simple contour $\Gamma$ and that $\lambda$ is in the
  interior of $\Gamma$ then
  \begin{align*}
    \vec u = \frac{1}{2 \pi \I} \oint_{\Gamma} ( z I - A)^{-1} \vec u \D z.
  \end{align*}
  Hint: Use $I = (z I - A)^{-1} (w I - A)  + (z I - A)^{-1} (z -
  w)$.\\

\item Suppose that the eigenvalues $\lambda_1,\ldots,\lambda_n$ of $A \in \mathbb C^{n \times n}$
  are real.  Show that for any $c > \lambda_n$ that
  \begin{align*}
    \E^{-c t} \E^{A t} \to 0, \quad t \to \infty.
  \end{align*}

\item  Laplace transform for systems of differential equations.  For
  this problem, recall the Laplace transform
  \begin{align*}
    \mathcal L[f](z) = \int_0^\infty f(t) \E^{-t z} \D t.
  \end{align*}
  For vector-valued functions, we consider it applied component wise.
  \begin{enumerate}
  \item Show that
    \begin{align*}
     \mathcal L[y'](z) =  z \mathcal L[y](z) - y(0).
    \end{align*}
    You assume as much as you want for $y$.
  \item If $\vec y'(t) = A \vec y(t) + \vec b(t)$, show that
    \begin{align*}
      \mathcal L [\vec y](z) = (z I - A)^{-1} (\mathcal L[\vec b](z) -
      \vec y(0)),
    \end{align*}
    whenever $z$ is not an eigenvalue of $A$.
  \item Write an integral formula for $\vec y(t)$.\\
  \end{enumerate}

\item Define $w(z) = \sqrt{z -1} \sqrt{z + 1}$.   Suppose $\rho :
  [-1,1] \to \mathbb R$ is smooth and $\rho(x) > 0$.  Consider
  \begin{align*}
    G(z) = \exp \left(  -\frac{w(z)}{2 \pi} \int_{-1}^1  \frac{\log
    \rho(x)}{x - z} \frac{\D
   x}{\sqrt{1 - x^2}}\right), \quad z \not\in (-1,1).
  \end{align*}
  Define $G_\pm (x) = \lim_{\epsilon \to 0^+} G(x \pm \I \epsilon)$.
  Show
  \begin{align*}
    G_+(x) G_-(x)= \rho(x), \quad x \in (-1,1).
  \end{align*}

\vspace{1in}

  
  On a historal note, the function $G$ is an example of a Szeg\H{o} function, named
  after G\'abor Szeg\H{o}, and
  is critical in describing the large-degree behavior of orthogonal
  polynomials.  Specifically, if $p_n$ is the degree $n$ orthonormal
  polynomial for a weight function $\rho(x)$ supported on $[-1,1]$ then
  \begin{align*}
    \lim_{n \to \infty} \frac{p_n(z)}{\varphi(z)^n} = \frac{
    \varphi(z)^{1/2}}{\sqrt{2 \pi} (z^2 - 1)^{1/4} G(z)}, \quad z \in
    \mathbb C \setminus [-1,1].
  \end{align*}
    
\end{enumerate}

  
\end{document}

%%% Local Variables:
%%% mode: latex
%%% TeX-master: t
%%% End:
