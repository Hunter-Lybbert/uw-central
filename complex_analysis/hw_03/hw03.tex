\documentclass[10pt]{amsart}
\usepackage[margin=1.4in]{geometry}
\usepackage[usenames,dvipsnames,cmyk]{xcolor} %load first

\usepackage{amssymb,amsmath,enumitem,url}

\newcommand{\D}{\mathrm{d}}
\newcommand{\I}{\mathrm{i}}
\DeclareMathOperator{\E}{e}
\DeclareMathOperator{\OO}{O}
\DeclareMathOperator{\oo}{o}
\DeclareMathOperator{\erfc}{erfc}
\DeclareMathOperator{\real}{Re}
\DeclareMathOperator{\imag}{Im}
\usepackage{tikz}
\usepackage[framemethod=tikz]{mdframed}
\theoremstyle{nonumberplain}

\mdtheorem[innertopmargin=-5pt]{sol}{Solution}
%\newmdtheoremenv[innertopmargin=-5pt]{sol}{Solution}
\definecolor{MichiganBlue}{HTML}{00274C}
\definecolor{MichiganYellow}{HTML}{FFCB05}  
\definecolor{NicePurple}{RGB}{75,56,76} %PrincePurple
\definecolor{NiceRed}{RGB}{230,37,52}
\definecolor{MidnightBlue}{rgb}{0.1, 0.1, 0.44}
\usepackage[colorlinks=true, linkcolor=MidnightBlue, citecolor=MidnightBlue, urlcolor=MidnightBlue]{hyperref}

\begin{document}
\pagestyle{empty}

\newcommand{\mline}{\vspace{.2in}\hrule\vspace{.2in}}

\noindent
\text{Hunter Lybbert} \\
\text{Student ID: 2426454} \\
\text{10-14-24} \\
\text{AMATH 567} \\

\title{\bf { Homework 3} }


\maketitle
\noindent
Collaborators*: TBD \\
\\
\tiny
\text{*Listed in no particular order. And anyone I discussed at least part of one problem with is considered a colaborator.}
\normalsize
\mline
\begin{enumerate}[label={\bf {\arabic*}:}]
\item From A\&F: 2.2.4. \\
Let $\alpha$ be a real number.
Show that the set of all values of the multivalued function $\log(z^a)$ is not necessarily the same as that of $\alpha \log z$. \\
\textit{Solution:} \\
\item Describe the Riemann surface on which the multi-valued function
  $w(z)$, defined by $w^2=\prod_{j=1}^{n=3}\left(z-a_j\right)$ is
  single-valued. What happens for $n=4,5$ ? For $n>5$ ? You may assume
  that all the $a_j$ are distinct.\\
\textit{Solution:} \\
\item From A\&F: 2.2.5a. While you're at it, also derive a formula for $\operatorname{arccot}(z)$ in terms of the logarithm.\\
Derive the following formulae: \\
a) $$\coth^{-1}(z) = \frac{1}{2}\log\frac{z + 1}{z - 1}$$
\textit{Solution:} \\
b) $$\sec h^{-1}(z) = \log \left(\frac{1 + (1 - z^2)^{\frac{1}{2}}}{z}\right)$$
\textit{Solution:} \\

\item Let
  \begin{align*}
    s(z) = z^{1/2} = \rho^{1/2} \E^{\I \theta/2}, \quad \theta \in [-\pi,\pi),
  \end{align*}
  denote the principal branch of the square root.  Show that the
  functions
  \begin{align*}
    f_1(z) = s(z^2 -1), \quad f_2(z) = s(z-1) s(z+1),
  \end{align*}
  are not equal as functions on $\mathbb C$ --- first produce plots and then use a mathematical argument.  Determine the branch cut for $f_2(z)$ (Note: My
  cartoon of what the branch cut for $f_1$ looks like in lecture was
  not accurate).  Find the relationship between $f_1(z)$ and $f_2(z)$.\\
\textit{Solution:} \\

  \item Consider the function
    \begin{align*}
     \psi(z) = \int_1^z \frac{\D w}{(w^2 - 1)^{1/2}}, \quad z \not \in
      (-\infty, 1),
    \end{align*}
    where the path of integration is a straight line from $1$ to $z$.
    \begin{itemize}
   \item  Show that
    \begin{align*}
      \psi(z) = \log \varphi(z), \quad \varphi(z) = z + (z^2 -
      1)^{1/2}, \quad z \not \in
      (-\infty, 1),
    \end{align*}
   for an appropriate choice of branch cut for $(z^2 -
   1)^{1/2}$.  Here $\log z$ denotes the principal branch.
   \item Find an expression for
   \begin{align*}
     \gamma(z) = \int_{-1}^z \frac{\D w}{(w^2 - 1)^{1/2}}, \quad z \not \in
      (-1, \infty),
   \end{align*}
   in terms of $\varphi(z)$ and the principal branch of the logarithm.  Again, the path of integration is a
   straight line.
 \end{itemize}
 \textit{Solution:} \\

 \item Show that $\varphi,$ from the previous problem, maps $\mathbb C \setminus [-1,1]$ onto the
   exterior of the unit disk, $\{ z \in \mathbb C ~:~ |z| > 1\}$.
   Furthermore
   \begin{align*}
     \frac 1 2 \left( \varphi(z) + 1/\varphi(z) \right) = z, \quad \mathbb C \setminus [-1,1].
   \end{align*}
\textit{Solution:} \\

   \item (Sharpness of the Bernstein--Walsh inequality)  The
     Bernstein--Walsh inequality states that if a polynomial $p_n$ of
     degree $n$ satisfies $\max_{-1 \leq x \leq 1} |p_n(x)| \leq 1$
     then
     \begin{align*}
       |p_n(z)| \leq |\varphi(z)|^n, \quad z \in \mathbb C \setminus [-1,1].
     \end{align*}
     Show that
     \begin{align*}
        T_n(z) = \frac 1 2 \left( \varphi(z)^n + \varphi(z)^{-n}
       \right), \quad z \in \mathbb C \setminus [-1,1]
     \end{align*}
     is a polynomial that satisfies
     \begin{align*}
       \max_{-1 \leq x \leq 1} |T_n(x)| &= 1,\\
       \lim_{n \to \infty} |T_n(z)|^{1/n} &= |\varphi(z)|,
     \end{align*}
     for any fixed $z \in \mathbb C \setminus [-1,1]$. \\
\textit{Solution:} \\
\end{enumerate}

\end{document}

%%% Local Variables:
%%% mode: latex
%%% TeX-master: t
%%% End:
