\documentclass[10pt]{amsart}
\usepackage[margin=1.4in]{geometry}
\usepackage[usenames,dvipsnames,cmyk]{xcolor} %load first
\usepackage{cancel}
\usepackage{graphicx,subfig}
\usepackage{mathtools}

\graphicspath{ {./images/} }

\usepackage{amssymb,amsmath,enumitem,url}

\newcommand{\D}{\mathrm{d}}
\newcommand{\I}{\mathrm{i}}
\DeclareMathOperator{\E}{e}
\DeclareMathOperator{\OO}{O}
\DeclareMathOperator{\oo}{o}
\DeclareMathOperator{\erfc}{erfc}
\DeclareMathOperator{\real}{Re}
\DeclareMathOperator{\imag}{Im}
\usepackage{tikz}
\usepackage[framemethod=tikz]{mdframed}
\theoremstyle{nonumberplain}

\mdtheorem[innertopmargin=5pt]{lemma}{Lemma}
\mdtheorem[innertopmargin=-5pt]{sol}{Solution}
%\newmdtheoremenv[innertopmargin=-5pt]{sol}{Solution}
\definecolor{MichiganBlue}{HTML}{00274C}
\definecolor{MichiganYellow}{HTML}{FFCB05}  
\definecolor{NicePurple}{RGB}{75,56,76} %PrincePurple
\definecolor{NiceRed}{RGB}{230,37,52}
\definecolor{MidnightBlue}{rgb}{0.1, 0.1, 0.44}
\usepackage[colorlinks=true, linkcolor=MidnightBlue, citecolor=MidnightBlue, urlcolor=MidnightBlue]{hyperref}

\begin{document}
\pagestyle{empty}

\newcommand{\mline}{\vspace{.2in}\hrule\vspace{.2in}}


\noindent
\text{Hunter Lybbert} \\
\text{Student ID: 2426454} \\
\text{11-25-24} \\
\text{AMATH 567} \\

\title{\bf { Homework 9} }


\maketitle
\noindent
Collaborators*: Cooper Simpson, Nate Ward, Laura Thomas \\
\\
\tiny
\text{*Listed in no particular order. And anyone I discussed at least part of one problem with is considered a collaborator.}
\normalsize


\mline
\begin{enumerate}[label={\bf {\arabic*}:}]
\item From A\&F: 4.1.2 only (i), i.e., only by computing residues inside.\\
Evaluate the integrals $\frac 1 {2 \I \pi} \oint_C f(z) \D z$, where C is the unit circle centered at the origin with $f(z)$ given below.
Do these problems ($i$) enclosing the singular points inside C. \\

\noindent
(a)
$$
\frac {z^2 + 1}{ z^2 - a^2 }, \quad a^2 < 1
$$
\textit{Solution:} \\
\textbf{TODO:}
\begin{align*}
\frac 1 {2 \I \pi} \oint_C \frac {z^2 + 1}{ z^2 - a^2 } \D z
\end{align*}

\noindent
(b)
$$
\frac {z^2 + 1}{ z^3 }
$$
\textit{Solution:} \\
\textbf{TODO:}
\begin{align*}
\frac 1 {2 \I \pi} \oint_C \frac {z^2 + 1}{ z^3 } \D z
\end{align*}

\noindent
(c)
$$
z^2\E^{-1/z}
$$
\textit{Solution:} \\
\textbf{TODO:}
\begin{align*}
\frac 1 {2 \I \pi} \oint_C z^2\E^{-1/z} \D z
\end{align*}

\newpage


\item From A\&F: 4.2.1(b)\\
Evaluate the following real integral
$$
\int_0^\infty \frac {\D x}{ (x^2 + a^2)^2 }, \quad a^2 > 0
$$
\textit{Solution:} \\
\textbf{TODO:}
\begin{align*}
\end{align*}

\newpage


\item  \textbf{Existence and uniqueness of polynomial interpolants.}
\begin{enumerate}
\item Suppose $(z_j)_{j = 1}^n$ are distinct points in $\mathbb C$ and suppose $f_j \in \mathbb C$ for $j = 1,\ldots,n$.
Show that there is at most one polynomial $p(z)$ of degree $n-1$ such that $p(z_j) = f_j$ for $j = 1,\ldots,n$ using Liouville's theorem.
Such a polynomial $p$ is called an \emph{interpolant}. \\

\noindent
\textit{Solution:} \\
Suppose there exists two polynomials $p_1(z)$ and $p_2(z)$ each of degree $n - 1$.
Assume both agree with $f_j$ at each $z_j$ such that
$$
p_1(z_j) = p_2(z_j) = f_j \quad \text{ for each } j = 1, ..., n.
$$
Additionally define the node polynomial $\nu (z) = \prod_{j=1}^n (z - z_j)$.
Now let's consider the function
$$
g(z) = \frac {p_1(z) - p_2(z)}{\nu(z)}.
$$
We want to utilize Liouville's theorem to conclude that $g(z)$ is constant.
In order to do this we need to show that $g(z)$ is entire and bounded.
Let's begin by demonstrating that it is bounded by taking the limit as $z\rightarrow\infty$
\begin{align*}
\lim_{z\rightarrow\infty}g(z) 
	&= \lim_{z\rightarrow\infty}\frac {p_1(z) - p_2(z)}{\nu(z)} \\
	&= \lim_{z\rightarrow\infty}\frac {p_1(z) - p_2(z)}{\prod_{j=1}^n (z - z_j)} \\
	&= \frac \infty \infty.
\end{align*}
Applying L'Hôpitals rule repeatedly we will end up with $1/z$ which goes to 0 as $z$ goes to infinity since the denominator is an $n$th degree polynomial while the numerator is a degree $n - 1$ polynomial.
Therefore,
$$
\lim_{z\rightarrow\infty}g(z) = \lim_{z\rightarrow\infty}\frac {p_1(z) - p_2(z)}{\nu(z)} = 0,
$$
which implies that $g(z)$ is bounded.
Next, we need to determine if $g(z)$ is entire.
Since polynomials are entire in the finite $z$ plane, $p_1(z) - p_2(z)$ is entire.
However, $g(z)$ overall requires a little more analysis since it has singularities where $z=z_j$.
Notice, since the expression $p_1(z) - p_2(z)$ and $\nu(z)$ are both zero at each $z_j$, then there exists a factorization of $p_1(z) - p_2(z)$ which would allows us to cancel out each of the factors in the product in the denominator.
Therefore, the singularities of $g(z)$ are removable and thus $g(z)$ is entire (or can be made entire, with the right extension at each $z_j$ as we have done in previous assignments).
Hence, by Liouville's Theorem, we can conclude that $g(z)$ is constant.
Combining with the fact that $p_1(z_j) - p_2(z_j) = 0$ for each $j = 1, ..., n$, then $g(z)$ must be 0 everywhere, thus implying $p_1(z) = p_2(z)$ everywhere.
In conclusion, since these two functions are the same therefore there is at most one polynomial $p(z)$ of degree $n - 1$ such that $p(z_j) = f_j$ for $j = 1,\ldots,n$, otherwise known at the interpolant. \\
\qed \\

\newpage

\item Define the node polynomial $\nu(z) = \prod_{j=1}^n( z - z_j)$.
Supposing that $p$ is an interpolant, as above, express $p(z)/\nu(z)$ as a rational function. 
Find an expression for $p(z)$. This shows existence.\\

\noindent
\textit{Solution:} \\
Let's look at $p(z)/\nu(z)$ and consider what happens if we subtract off a specially cooked up collection of terms including the residues $r_j$ for $j=1, ..., n$.
We can express the residues of $p(z)/\nu(z)$ as
$$
\frac 1 {2 \pi \I} \oint_C \frac {p(z)}{\nu(z)} \D z
	= \sum_{j = 0}^n {\rm Res}\bigg(\frac {p(z)}{\nu(z)}; z_j\bigg)
	= \sum_{j = 0}^n \frac {f_j} {\prod_{k \neq j}( z _k - z_j )}.
$$
Recall partial fractions is connected to the residues.
We construct the expression to subtract from $p(z)/\nu(z)$ using the partial fraction decomposition relationship to residues
\begin{align*}
& \frac {p(z)}{\nu(z)} - \sum_{j = 0}^n \frac {f_j \Big(\prod_{k \neq j}( z _k - z_j )\Big)^{-1}} {z - z_j} \\
	&\: = \frac {p(z)}{\nu(z)}
		- \frac {f_1 \Big(\prod_{k \neq 1}( z _k - z_1 )\Big)^{-1}}{z - z_1}
		- \frac {f_2 \Big(\prod_{k \neq 2}( z _k - z_2 )\Big)^{-1}}{z - z_2} 
		- ...
		- \frac {f_n \Big(\prod_{k \neq n}( z _k - z_n )\Big)^{-1}}{z - z_n}
	&= 0.
\end{align*}
\textbf{TODO: Why is this 0 besides saying it's the partial fraction decomposition?}
This expression is equal to 0 because the collection of terms we are subtracting is the partial fraction decomposition of $p(z)/\nu(z)$.
If we can show that this function is bounded and entire then it is a constant.
Therefore, we would be able to state that since it is a constant and 0 then it must be a 0 everywhere.
Thus we can say
\begin{align*}
\frac {p(z)}{\nu(z)} - \sum_{j = 0}^n \frac {f_j \Big(\prod_{k \neq j}( z _k - z_j )\Big)^{-1}} {z - z_j} &= 0 \\
\frac {p(z)}{\nu(z)} &= \sum_{j = 0}^n \frac {f_j \Big(\prod_{k \neq j}( z _k - z_j )\Big)^{-1}} {z - z_j} \\
p(z) &= \nu(z)\sum_{j = 0}^n \frac {f_j \Big(\prod_{k \neq j}( z _k - z_j )\Big)^{-1}} {z - z_j} \\
p(z) &= \prod_{j=1}^n( z - z_j)\sum_{j = 0}^n \frac {f_j \Big(\prod_{k \neq j}( z _k - z_j )\Big)^{-1}} {z - z_j} \\
p(z) &= \sum_{j = 0}^n \frac {f_j \prod_{\ell \neq j}( z - z_\ell) } {\prod_{k \neq j}( z _k - z_j )}.
\end{align*}
Therefore we have this expression for $p(z)$.
\end{enumerate}

\newpage


\item \textbf{Bernstein interpolation formula.}   Suppose that $x_1 < x_2 <
  \cdots x_n$.  And suppose that $f(z)$ is analytic in a region
  $\Omega$ that contains $[-1,1]$.  Show that for any simple contour
  $C$ inside $\Omega$ with $[-1,1]$ in its interior that
  \begin{align*}
    f(x) - p(x) = \frac{\nu(x)}{2 \pi i} \int_C \frac{f(z)}{z - x}
    \frac{\D z}{\nu(z)}, \quad x \in [-1,1],
  \end{align*}
  where $p(x_j) = f(x_j)$ for $j = 1,2,\ldots,n$, $\nu(x) =
  \prod_{j=1}^n (x - x_j)$.\\

\noindent
\textit{Solution:} \\
\textbf{TODO: the $x_j$ are in -1,1...residue on the right, and p is degree n-1}
\begin{align*}
\end{align*}

\newpage


\item  \textbf{Chebyshev polynomial interpolants.}  Recall 
\begin{align*}
\varphi(z) = z + \sqrt{z - 1} \sqrt{z +1}, \quad z \in \mathbb C \setminus [-1,1].
\end{align*}

\begin{enumerate}
\item Show that the polynomial
\begin{align*}
T_n(z) = \frac 1 2 \left( \varphi(z)^n + \varphi(z)^{-n} \right),
\end{align*}
has all of its roots $x_1 < x_2 < \cdots x_n$ within $[-1,1]$. \\

\noindent
\textit{Solution:} \\
\textbf{TODO:}
\begin{align*}
\end{align*}

\item Consider $J(w) =  1/2 (w + 1/w)$. 
Show that the image of the circle of radius $\rho > 1$ under $J$ is an ellipse $B_\rho$ that contains $[-1,1]$ in its interior.
Then show $\varphi(J(w)) = w$. \\

\noindent
\textit{Solution:} \\
\textbf{TODO: apply things from hw3 problem 7 or 8?}
\begin{align*}
\end{align*}

\item Show that if $f$ is analytic in a region that contains $B_\rho$ and its interior, and $|f(z)| \leq M$ for $z$ interior to $B_\rho$ then for $-1 \leq x \leq 1$,
\begin{align*}
|f(x) - p(x)| \leq 2 \frac{M | B_\rho| }{\pi}  (\rho^n - \rho^{-n})^{-1} (\rho + \rho^{-1} - 1)^{-1} \leq 2 \frac{M | B_\rho| }{\pi} \frac{\rho^{1-n}}{(\rho - 1)^2}.
\end{align*}
where $p(x_j) = f(x_j)$, i.e., $p$ is the interpolant of $f$ at the roots of $T_n$. 
Here $|B_\rho|$ denotes the arclength of $B_\rho$.  This shows that the exponential rate of convergence of Chebyshev interpolants is governed by the proximity of the nearest singularity of $f$. \\

\noindent
\textit{Solution:} \\
\textbf{TODO:} \
\textbf{TODO: p is the polynomial interpolant of f of degree n - 1, lots of varphi stuff hw 3 prob 6/7/8}
\begin{align*}
\end{align*}
      
\end{enumerate}
\newpage


\item Compute the following two integrals explicitly for $ z \not \in [-1,1]$:
\begin{enumerate}
\item
\begin{align*}
\frac{1}{\pi}\int_{-1}^1 \frac{1}{\sqrt{1-x} \sqrt{1 + x}} \frac{\D x}{x -z}.
\end{align*}

\textit{Solution:} \\
\textbf{TODO:}
\begin{align*}
\end{align*}

\item 
\begin{align*}
\frac{2}{\pi}\int_{-1}^1 {\sqrt{1-x} \sqrt{1 + x}}
\frac{\D x}{x -z}.
\end{align*}

\textit{Solution:} \\
\textbf{TODO:}
\begin{align*}
\end{align*}


\end{enumerate}

      
    


  
\end{enumerate}

\end{document}

%%% Local Variables:
%%% mode: latex
%%% TeX-master: t
%%% End:
